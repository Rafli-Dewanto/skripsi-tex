\chapter{PENDAHULUAN}

\setcounter{page}{1}
\pagenumbering{arabic}
\thispagestyle{plain}
\pagestyle{arabicstyle}

\section{Latar Belakang Masalah}

Meningkatnya kasus batu ginjal di Indonesia menjadi perhatian serius
di bidang kesehatan, mengingat kondisi ini dapat menyebabkan komplikasi
berat seperti infeksi, kerusakan fungsi ginjal, bahkan kematian jika
tidak ditangani secara tepat waktu.

\section{Ruang Lingkup}

Batasan-batasan tersebut adalah sebagai berikut.
\begin{enumerate}
\item Citra
\end{enumerate}

\section{Tujuan Penelitian}

Tujuan penelitian yang akan dibahas adalah sebagai berikut.
\begin{enumerate}
\item Mengembangkan model
\end{enumerate}

\section{Sistematika Penulisan}

Sistematika yang digunakan dalam penulisan ini mencakup beberapa bagian
penting, di antaranya:
\begin{enumerate}
\item PENDAHULUAN

Bab ini berisi pendahuluan yang menguraikan latar belakang masalah
yang melandasi pemilihan topik penelitian, didukung oleh data relevan
dan perbandingan dengan penelitian terdahulu, yang menyoroti kelemahan
atau perbedaan pendekatan. Ruang lingkup penelitian dibatasi secara
jelas untuk fokus pada persoalan yang dikaji. Tujuan penelitian ini
adalah untuk menjawab masalah penelitian dan menghasilkan luaran yang
diharapkan. Sistematika penulisan laporan skripsi ini akan disusun
secara naratif, dimulai dari bab pendahuluan yang menguraikan konteks
penelitian, diikuti oleh tinjauan pustaka yang membahas landasan teoretis,
metodologi penelitian yang menjelaskan pendekatan dan teknik pengumpulan
data, hasil penelitian dan pembahasan yang menganalisis temuan, hingga
kesimpulan dan saran yang merangkum hasil penelitian dan implikasinya.
\item TINJAUAN PUSTAKA

Bab ini merupakan tinjauan pustaka yang disusun untuk memberikan landasan
teoretis dan empiris yang kokoh dalam mendukung pendekatan pemecahan
masalah yang diusulkan. Bab ini menguraikan secara komprehensif hasil
penelitian sejenis yang relevan dengan tema yang dipilih, dengan menekankan
pada kontribusi penelitian-penelitian tersebut terhadap pengembangan
kerangka teoretis dan metodologis penelitian ini. Selain itu, tinjauan
pustaka ini juga berfungsi untuk memperkuat argumentasi penelitian
dengan menunjukkan bukti-bukti empiris dari penelitian sebelumnya,
serta mengidentifikasi celah penelitian yang akan diisi oleh penelitian
ini. Tinjauan pustaka ini tidak hanya merangkum literatur yang ada,
tetapi juga menganalisis dan mensintesisnya untuk membangun dasar
yang kuat bagi penelitian ini.
\item METODE PENELITIAN

Bab ini merupakan bagian perancangan yang menguraikan secara komprehensif
prosedur yang ditempuh untuk mencapai tujuan penelitian, mencakup
detail peralatan, bahan, teknik pengumpulan dan analisis data, model
pendekatan, serta rancangan penelitian. Uraian ini harus memungkinkan
replikasi penelitian, dengan langkah-langkah yang dijelaskan secara
kronologis dan sistematis. Pada penelitian di bidang ilmu komputer,
penjelasan mencakup peralatan, algoritma/metode, dan prosedur implementasi
(misalnya, Prototype, SDLC, Agile). Alur penelitian divisualisasikan
dalam bagan, dan struktur pembahasan hasil dipengaruhi oleh metode
yang dipilih.
\item HASIL DAN PEMBAHASAN

Bab ini menyajikan hasil penelitian yang diperoleh, disertai dengan
analisis mendalam untuk menjawab tujuan penelitian yang telah ditetapkan.
Pada kasus penelitian yang menghasilkan rancangan, bab ini memuat
deskripsi rinci hasil rancangan tersebut, termasuk evaluasi komprehensif
terhadap kelebihan dan keterbatasan yang teridentifikasi. Analisis
ini dilakukan secara sistematis dan objektif, dengan mengacu pada
kerangka teoretis dan metodologis yang telah ditetapkan sebelumnya.
\item PENUTUP

Bagian penutup ini terdiri dari kesimpulan, yang merangkum jawaban
atas masalah penelitian berdasarkan temuan yang diperoleh, dan saran,
yang mengusulkan pengembangan lebih lanjut dari hasil penelitian yang
telah dipaparkan dalam kesimpulan.
\end{enumerate}