\documentclass[12pt,oneside,bahasa]{book}
\usepackage{charter}
\usepackage[T1]{fontenc}
\usepackage[latin9]{inputenc}
\setcounter{secnumdepth}{3}
\setcounter{tocdepth}{3}
\usepackage{array}
\usepackage{longtable}
\usepackage{float}
\usepackage{varwidth}
\usepackage{url}
\usepackage{hyperref} 
\usepackage{graphicx}
\usepackage[a4paper]{geometry}
\geometry{verbose,tmargin=3cm,bmargin=3cm,lmargin=4cm,rmargin=3cm}
\usepackage{fancyhdr}
\pagestyle{fancy}
\usepackage{setspace}
\usepackage[authoryear]{natbib}
\usepackage{titlesec}
\usepackage{graphicx}
\usepackage{indentfirst}
\usepackage{tocloft}
\onehalfspacing

\makeatletter

% LyX specific LaTeX commands.
% Because html converters don't know tabularnewline
\providecommand{\tabularnewline}{\\}
% Variable width box for table cells
\newenvironment{cellvarwidth}[1][t]
    {\begin{varwidth}[#1]{\linewidth}}
    {\@finalstrut\@arstrutbox\end{varwidth}}

\titleformat{\chapter}[hang]
  {\normalfont\huge\bfseries\centering}{\thechapter.}{14pt}{\Huge\MakeUppercase}
\newcommand{\Judul}{Aplikasi \textit {Review} Artikel Ilmiah dengan metode \textit {Retrieval-Augmented Generation} Dan \textit {Langchain} }
\newcommand{\JudulInggris}{Scientific Article Review Application with Retrieval-Augmented Generation and Langchain method}
\newcommand{\Penulis}{Rafli Satya Dewanto}
\newcommand{\NPM}{51421208}
\newcommand{\NIRM}{98432324243}
\newcommand{\JenisTulisan}{Skripsi}
\newcommand{\Gelar}{Jenjang S1 / Setara Sarjana Muda}
\newcommand{\Fakultas}{Teknologi Industri}
\newcommand{\Faculty}{Industrial Technology}
\newcommand{\Jurusan}{Informatika}
\newcommand{\Major}{Informatics}
%%-------------
\newcommand{\Prodi}{Direktorat Program Diploma Tiga Teknologi Informasi}
\newcommand{\Tahun}{2025}
\newcommand{\Bulan}{Juli}
\newcommand{\Tanggal}{24}
\newcommand{\Kota}{JAKARTA}
%%-------------
\newcommand{\KataKunci}{\textit{Artifical Intelligence, Retrieval-Augmented Generation, Chatbot, Next.js, Gemini}}
\newcommand{\KeyWords}{Artifical Intelligence, Retrieval-Augmented Generation, Chatbot, Next.js, Gemini}
%%-------------
\newcommand{\KoordinatorPI}{Dr. Achmad Fahrurozi, S.Si., M.Si.}
%%-------------
\newcommand{\KetuaJurusan}{Prof. Dr. Lintang Yuniar Banowosari, S.Kom., M.Sc.}
\newcommand{\DosenPembimbingA}{Drs. Agus Sumin, MMSI.}
\newcommand{\DosenPembimbingB}{Dr. Ricky Agus T., S.T., S.Si., M.M.}
\newcommand{\KetuaPembimbing}{Prof. Dr. Lintang Yuniar Banowosari, S.Kom., M.Sc.}
\newcommand{\AnggotaPembimbingA}{Dr. Elfitrin Syahrul, ST., MT.}
\newcommand{\AnggotaPembimbingB}{Dr. Marliza Ganefi Gumay, S.Kom., MMSI}
%%------------
\newcommand{\KetuaUjian}{Dr. Ravi Ahmad Salim}
\newcommand{\SekUjian}{Prof. Dr. Wahyudi Priyono}
\newcommand{\AnggotaUjianA}{Prof. Dr. Lintang Yuniar Banowosari, S.Kom., M.Sc.}
\newcommand{\AnggotaUjianB}{Dr. Elfitrin Syahrul, ST., MT.}
\newcommand{\AnggotaUjianC}{Dr. Marliza Ganefi Gumay, S.Kom., MMSI}
%%-------------
\newcommand{\Ringkasan}{Tulis ringkasan skripsi, pi, atau apa dengan bahasa yang jelas, lugas dan menggambarkan secara singkat tulisan ini.  Sebaiknya tidak lebih dari 150 kata dan sudah menjelaskan dari permasalahan, pembahasan dan penutup.}
\newcommand{\JumlahPustaka}{50}
\newcommand{\JumlahHalaman}{89}
\newcommand{\JumlahHalamanDepan}{xiii}
\newcommand{\TahunPustaka}{1959-2024}
%%
%% Keterangan administratif sidang sarjana
%%
\newcommand{\TanggalSidang}{24 Juli 2025}
\newcommand{\TanggalLulus}{24 Juli 2025}
\newcommand{\TanggalSah}{}
\newcommand{\PejabatBagianSidang}{Dr. Edi Sukirman, S.Si., MM., M.I.Kom.}
\setlength{\headheight}{15pt}
%%-------------
%%
%%Untuk Kata pengantar
%%
\newcommand{\Rektor}{Prof. Dr. E.S. Margianti, SE., MM.}
\newcommand{\Dekan}{Prof. Dr. Ing. Adang Suhendra, S.Si, S.Kom., M.Sc.}
\newcommand{\KotaPenulis}{Depok}
\newcommand{\BlnThn}{24 Juli 2025}

% Mengatur judul Daftar Isi, Daftar Gambar, dan Daftar Tabel agar tetap rata tengah dengan ukuran default
\renewcommand{\cfttoctitlefont}{\hfill\Huge\bfseries}
\renewcommand{\cftloftitlefont}{\hfill\Huge\bfseries}
\renewcommand{\cftlottitlefont}{\hfill\Huge\bfseries}
\renewcommand{\cftaftertoctitle}{\hfill}
\renewcommand{\cftafterloftitle}{\hfill}
\renewcommand{\cftafterlottitle}{\hfill}
% Membuat daftar khusus untuk lampiran
\newlistof{appendices}{app}{Daftar Lampiran}
% Kode untuk mengatur judul Daftar Lampiran
\makeatletter
\renewcommand{\listofappendices}{%
  \chapter*{DAFTAR LAMPIRAN}%
  \@starttoc{app}%
}
\makeatother


\renewcommand{\cftchapleader}{\cftdotfill{\cftdotsep}}
\renewcommand{\cftsecleader}{\cftdotfill{\cftdotsep}}
\renewcommand{\cftsubsecleader}{\cftdotfill{\cftdotsep}}



% Define custom page numbering style
\usepackage{fancyhdr}
\newcommand{\appendixpagenumbering}{
    \renewcommand{\thepage}{L-\arabic{page}}
}


% Mengatur header dan footer
\fancypagestyle{romanstyle}{
    \fancyhf{}
    \fancyfoot[C]{\thepage} % Nomor halaman di tengah bawah
    \renewcommand{\headrulewidth}{0pt} % Hapus garis header
    \renewcommand{\footrulewidth}{0pt} % Hapus garis footer
}

\fancypagestyle{arabicstyle}{
    \fancyhf{}
    \fancyhead[R]{\thepage} % Nomor halaman di kanan atas
    \renewcommand{\headrulewidth}{0pt} % Hapus garis header
    \renewcommand{\footrulewidth}{0pt} % Hapus garis footer
}

\fancypagestyle{appendixstyle}{
    \fancyhf{}
    \fancyfoot[C]{\thepage} % Nomor halaman di tengah bawah
    \renewcommand{\headrulewidth}{0pt} % Hapus garis header
    \renewcommand{\footrulewidth}{0pt} % Hapus garis footer
}

\makeatother

\usepackage{babel}
\usepackage{listings}
\renewcommand{\lstlistingname}{Listing}

\begin{document}
\input{components/made-hypen.tex} 
\sloppy 
\thispagestyle{empty}

\addcontentsline{toc}{chapter}{HALAMAN JUDUL}

\pagenumbering{roman}
\pagestyle{romanstyle}

\vspace*{10mm}

\begin{center}
{\large\textbf{UNIVERSITAS GUNADARMA}}{\large\par}
\par\end{center}

\begin{center}
{\large\textbf{FAKULTAS \MakeUppercase{\Fakultas}}}{\large\par}
\par\end{center}

\vspace*{10mm}

\begin{center}
\includegraphics[width=55mm]{images/gundarlogo.png}
\par\end{center}

\vspace*{3mm}

\begin{center}
{\Large\textbf{SKRIPSI}}{\Large\par}
\par\end{center}

\vspace*{7mm}

\begin{center}
\begin{tabular}{|>{\centering}m{13.5cm}|}
\hline 
\centering{}~~\tabularnewline
\centering{}\textbf{\MakeUppercase{\Judul}}\tabularnewline
~~\tabularnewline
\centering{}%
\begin{tabular}{lcl}
Nama & : & \Penulis\tabularnewline
NPM & : & \NPM\tabularnewline
Fakultas & : & \Fakultas\tabularnewline
Program Studi & : & \Jurusan\tabularnewline
Dosen Pembimbing 1 & : & \DosenPembimbingA\tabularnewline
Dosen Pembimbing 2 & : & \DosenPembimbingB\tabularnewline
\end{tabular}\tabularnewline
~~\tabularnewline
\hline 
\end{tabular}
\par\end{center}

 \vspace{7mm}

\begin{center}
\begin{tabular}{c}
\begin{cellvarwidth}[t]
\centering
\textbf{Diajukan Guna Melengkapi Sebagian Syarat Dalam Mencapai }

\textbf{Gelar Setara Sarjana Strata Satu (S1)}
\end{cellvarwidth}\tabularnewline
\textbf{\Kota}\tabularnewline
\textbf{\Tahun}\tabularnewline
\end{tabular}
\par\end{center}

\chapter*{{\huge PERNYATAAN ORISINALITAS DAN PUBLIKASI}}

\begin{singlespace}
\addcontentsline{toc}{chapter}{PERNYATAAN ORISINALITAS DAN PUBLIKASI}
\thispagestyle{romanstyle}

Saya yang bertanda tangan di bawah ini,

\vspace*{10mm}

\hspace{-6pt}%
\begin{tabular}{>{\raggedright}p{30mm}c>{\raggedright}p{0.66\textwidth}}
Nama & : & \Penulis\tabularnewline
NPM & : & \NPM\tabularnewline
Judul PI & : & \MakeUppercase{\Judul}\tabularnewline
Tanggal Sidang & : & \TanggalSidang\tabularnewline
Tanggal Lulus & : & \TanggalLulus\tabularnewline
\end{tabular}

\vspace*{10mm}

\noindent menyatakan bahwa tulisan ini adalah merupakan hasil karya
saya sendiri dan dapat dipublikasikan sepenuhnya oleh Universitas
Gunadarma. Segala kutipan dalam bentuk apa pun telah mengikuti kaidah
dan etika yang berlaku. Semua hak cipta dari logo serta produk yang
disebut dalam buku ini adalah milik masing-masing pemegang haknya,
kecuali disebutkan lain. Mengenai isi dan tulisan merupakan tanggung
jawab Penulis, bukan Universitas Gunadarma.

Demikianlah pernyataan ini dibuat dengan sebenarnya dan dengan penuh
kesadaran.

\vspace*{15mm}
\end{singlespace}

\begin{flushright}
\begin{tabular}{c}
\KotaPenulis, \  \Tanggal\ \Bulan\ \Tahun\tabularnewline
\vspace*{25mm}\tabularnewline
\Penulis\tabularnewline
\end{tabular}
\par\end{flushright}

\chapter*{{\huge LEMBAR PENGESAHAN}}

\addcontentsline{toc}{chapter}{LEMBAR PENGESAHAN}

\begin{center}
\begin{tabular}{|c|ll|c|}
\multicolumn{4}{c}{\textbf{KOMISI PEMBIMBING}}\tabularnewline
\hline 
\textbf{NO} & \multicolumn{2}{c|}{\textbf{NAMA}} & \textbf{KEDUDUKAN}\tabularnewline
\hline 
1. & \KetuaPembimbing &  & Ketua\tabularnewline
\hline 
2. & \AnggotaPembimbingA &  & Anggota\tabularnewline
\hline 
3. & \AnggotaPembimbingB &  & Anggota\tabularnewline
\hline 
\end{tabular}
\par\end{center}

\begin{center}
\begin{tabular}{|c|ll|c|}
\multicolumn{4}{c}{\textbf{PANITIA UJIAN}}\tabularnewline
\hline 
\textbf{NO} & \multicolumn{2}{c|}{\textbf{NAMA}} & \textbf{KEDUDUKAN}\tabularnewline
\hline 
1. & \KetuaUjian &  & Ketua\tabularnewline
\hline 
2. & \SekUjian &  & Sekretaris\tabularnewline
\hline 
3. & \AnggotaUjianA &  & Anggota\tabularnewline
\hline 
4. & \AnggotaUjianB &  & Anggota\tabularnewline
\hline 
5. & \AnggotaUjianC &  & Anggota\tabularnewline
\hline 
\end{tabular}
\par\end{center}

\begin{center}
\vspace*{5mm}
\par\end{center}

\begin{center}
\begin{tabular}{ccc}
\multicolumn{3}{c}{\textbf{Mengetahui,}}\tabularnewline
\multicolumn{3}{c}{\textbf{\vspace*{5mm}}}\tabularnewline
\textbf{Dosen Pembimbing} &  & \textbf{Bagian Sidang Ujian}\tabularnewline
\textbf{\vspace*{10mm}} &  & \textbf{\vspace*{10mm}}\tabularnewline
\textbf{(\DosenPembimbingB)} &  & \textbf{(\KoordinatorPI)}\tabularnewline
\end{tabular}
\par\end{center}

\chapter*{ABSTRAK}

\begin{singlespace}
\addcontentsline{toc}{chapter}{ABSTRAK}

\noindent\Penulis. \NPM

\noindent\MakeUppercase{\Judul} \\
Skripsi, Fakultas \Fakultas, Program Studi \Jurusan, Universitas
Gunadarma, \Tahun

\noindent Kata Kunci : \KataKunci

\medskip{}

\noindent (\JumlahHalamanDepan \ + \JumlahHalaman \ + lampiran)

\bigskip{}

\emph{Deepfake}, teknologi yang memanfaatkan \emph{machine learning}
untuk memanipulasi video dan gambar, telah menjadi tantangan besar
dalam beberapa tahun terakhir.

\bigskip{}

\noindent Daftar Pustaka (\TahunPustaka)
\end{singlespace}

\chapter*{ABSTRACT}

\begin{singlespace}
\addcontentsline{toc}{chapter}{ABSTRACT}

\noindent\Penulis. \NPM

\noindent\MakeUppercase{\JudulInggris} \\
Thesis, Faculty of \Faculty, \Major\ Study Program, Gunadarma University,
\Tahun

\noindent Keywords : \KeyWords

\medskip{}

\noindent (\JumlahHalamanDepan \ + \JumlahHalaman \ + attachment)

\bigskip{}

Deepfake, a technology that uses machine learning to manipulate videos
and images, has become a major challenge in recent years.

\bigskip{}

\noindent Bibliography (\TahunPustaka)
\end{singlespace}

\chapter*{KATA PENGANTAR}

\thispagestyle{plain}
\addcontentsline{toc}{chapter}{KATA PENGANTAR}

Puji syukur ke hadirat Tuhan Yang Maha Esa atas segala rahmat dan
karunia-Nya, serta doa dan dorongan dari berbagai pihak, sehingga
skripsi yang berjudul ``\Judul'' dapat diselesaikan dengan baik.

Penyusunan skripsi ini merupakan salah satu syarat untuk mencapai
jenjang Setara Sarjana Strata Satu (S1) pada Program Studi Informatika,
Fakultas Teknologi Industri, Universitas Gunadarma.

Walaupun dalam penyusunannya menemui berbagai kendala, berkat bantuan
dan dorongan dari berbagai pihak, skripsi ini dapat diselesaikan dengan
baik. Penulis ingin mengucapkan terima kasih kepada: 
\begin{enumerate}
\item \Rektor, selaku Rektor Universitas Gunadarma.
\item \Dekan, selaku Dekan Fakultas Teknologi Industri, Universitas Gunadarma.
\item \KetuaJurusan , selaku Ketua Program Studi \Jurusan.
\item \KoordinatorPI, selaku Kepala Subbagian Sidang PI Fakultas Teknologi
Industri.
\item \DosenPembimbingA\ dan \DosenPembimbingB, sebagai Dosen Pembimbing
yang di tengah-tengah aktivitas dan kesibukannya telah membimbing
dan memberikan dorongan serta dukungan, sehingga penulisan ini dapat
diselesaikan.
\item Seluruh rekan 4IA01 di Universitas Gunadarma yang telah banyak
membantu serta memberi semangat selama proses penulisan. 
\item Semua pihak yang tidak bisa disebutkan yang telah membantu penyelesaian
skripsi ini, diucapkan juga terima kasih atas segala bantuan dan sarannya. 
\end{enumerate}
\indent

Sebagai manusia biasa yang tidak luput dari kesalahan, penulis menyadari
bahwa skripsi ini masih jauh dari sempurna dan memiliki banyak kekurangan.
Oleh karena itu, penulis dengan terbuka menerima kritik dan saran
yang bersifat membangun demi perbaikan di masa yang akan datang. Semoga
skripsi ini dapat memberikan manfaat bagi para pembaca dan menjadi
kontribusi kecil bagi kemajuan ilmu pengetahuan.

\begin{singlespace}
\vspace*{15mm}


\end{singlespace}\begin{flushright}
\begin{tabular}{c}
\KotaPenulis, \BlnThn\vspace{10pt}\tabularnewline
Penulis\tabularnewline
\vspace*{25mm}\tabularnewline
\Penulis\tabularnewline
\end{tabular}
\par\end{flushright}

\clearpage
\phantomsection 
\vspace*{7mm}

\thispagestyle{plain}
\addcontentsline{toc}{chapter}{DAFTAR ISI}
\renewcommand\contentsname{DAFTAR ISI}

\tableofcontents{}

\clearpage
\phantomsection 
\vspace*{7mm}

\thispagestyle{plain}
\addcontentsline{toc}{chapter}{DAFTAR TABEL}
\renewcommand\listtablename{DAFTAR TABEL}

\listoftables

\clearpage
\phantomsection 
\vspace*{7mm}

\thispagestyle{plain}
\addcontentsline{toc}{chapter}{DAFTAR GAMBAR}
\renewcommand\listfigurename{DAFTAR GAMBAR}

\listoffigures

\clearpage
\phantomsection 

\thispagestyle{plain}
\addcontentsline{toc}{chapter}{DAFTAR LAMPIRAN}
\listofappendices
\chapter{PENDAHULUAN}

\setcounter{page}{1}
\pagenumbering{arabic}
\thispagestyle{plain}
\pagestyle{arabicstyle}

\section{Latar Belakang Masalah}

Mahasiswa, terutama yang bukan penutur asli bahasa Inggris, sering menghadapi tantangan dalam memahami teks akademik berbahasa Inggris. Kesulitan ini disebabkan oleh kompleksitas struktur kalimat, penggunaan kosakata teknis, dan gaya penulisan yang formal. Penelitian oleh~\cite{Dardjito} menunjukkan bahwa mahasiswa di Indonesia sering merasa cemas dan kurang percaya diri saat membaca jurnal akademik dalam bahasa Inggris, yang berdampak negatif pada keterlibatan mereka dalam proses pembelajaran dan penelitian. Kesulitan ini juga dapat menghambat kemampuan mereka untuk mengembangkan pemikiran kritis dan berkontribusi dalam diskusi ilmiah di bidang studi mereka.

Sebagai solusi terhadap tantangan tersebut, pemanfaatan Artificial Intelligence (AI) dapat secara signifikan meningkatkan pengalaman dan hasil mahasiswa dalam penulisan research paper melalui beberapa mekanisme yang bermanfaat. AI-powered writing tools, seperti pemeriksa tata bahasa dan asisten gaya, memberikan feedback real-time tentang kualitas tulisan, membantu mahasiswa menyempurnakan tata bahasa dan sintaksis mereka untuk menghasilkan teks yang lebih jelas dan koheren \citep{kong2024pedagogical,nazari2021application}. Dengan menggabungkan alat tersebut ke dalam proses penulisan mereka, mahasiswa tidak hanya dapat meningkatkan aspek teknis dari pekerjaan mereka tetapi juga meningkatkan kepercayaan diri mereka dalam menyusun teks akademik \citep{nazari2021application}.

Peran AI meluas melampaui mekanisme bahasa. AI juga dapat membantu dalam melakukan tinjauan literatur dengan cepat meringkas penelitian yang ada dan mengidentifikasi sumber-sumber yang relevan, yang menyederhanakan proses penelitian akademik \citep{gupta2024artificial}. Misalnya, alat AI dapat menganalisis volume besar literatur akademik untuk menyajikan ringkasan singkat, membantu mahasiswa dalam mengumpulkan dan mensintesis informasi yang diperlukan untuk paper mereka secara efisien \citep{bulante2024ai}. Integrasi AI ini tidak hanya meningkatkan efisiensi penelitian tetapi juga membantu mahasiswa mengembangkan kemampuan membaca kritis dan analitis dengan mendorong mereka untuk terlibat dengan konten yang telah disintesis.

Penggabungan Retrieval-Augmented Generation dalam aplikasi chatbot berfungsi untuk memperluas batasan yang melekat pada model generasi konvensional. LLM tradisional sering kali kesulitan dengan akurasi faktual, terutama ketika ditanyai tentang informasi yang spesifik atau dinamis \citep{wang2024mememo,lewis2020retrieval}. Teknik RAG secara efektif mengatasi masalah ini dengan memungkinkan model mengakses dan menggabungkan pengetahuan real-time dari database eksternal, memastikan bahwa respons tidak hanya relevan secara kontekstual tetapi juga akurat dan koheren \citep{lewis2020retrieval}.

Dengan adanya aplikasi ini, mahasiswa dapat berinteraksi secara aktif dengan isi jurnal, bukan hanya membaca pasif, sehingga meningkatkan keterlibatan dan pemahaman akademik mereka. Aplikasi ini bukan bertujuan untuk menggantikan proses berpikir kritis, tetapi mendukung mahasiswa menjadi lebih mandiri dan efisien dalam memahami literatur dan mengelola catatan pribadi mereka. Dengan demikian, pengembangan aplikasi chatbot ini diharapkan mampu menjadi alat bantu yang efektif bagi mahasiswa dalam memahami isi jurnal, mencatat poin penting, dan menyusun sitasi dengan lebih efisien

\section{Rumusan Masalah}
Dalam penulisan ini terdapat beberapa rumusan masalah yang penulis akan bahas, diantaranya:
\begin{enumerate}
  \item Bagaimana merancang dan mengembangkan aplikasi chatbot yang dapat membantu mahasiswa memahami isi jurnal akademik berbahasa Inggris yang kompleks?
        \raggedright
  \item Bagaimana pemanfaatan metode Retrieval-Augmented Generation (RAG) dapat meningkatkan akurasi dan relevansi jawaban chatbot terhadap pertanyaan terkait isi jurnal?
  \item Bagaimana integrasi teknologi Langchain dapat mendukung kemampuan chatbot dalam berinteraksi secara kontekstual dan memberikan pengalaman yang lebih adaptif kepada pengguna?
  \item Bagaimana aplikasi ini dapat membantu mahasiswa dalam membuat sitasi otomatis dengan berbagai gaya referensi?
  \item Bagaimana aplikasi ini dapat menyediakan fitur anotasi PDF sebagai catatan pribadi mahasiswa dalam proses riset?
\end{enumerate}

\section{Ruang Lingkup}

Untuk menjaga fokus dan kejelasan dalam proses penelitian serta pengembangan aplikasi, maka ruang lingkup penelitian ini dibatasi pada aspek-aspek berikut:

\begin{enumerate}
  \item Aplikasi ini hanya digunakan untuk membantu mahasiswa memahami isi jurnal ilmiah yang telah diunggah dalam format PDF, dan tidak menyediakan fitur pencarian atau pengunduhan jurnal dari sumber eksternal seperti Google Scholar, Scopus, atau database jurnal lainnya.

  \item Fitur chatbot yang dikembangkan terbatas pada proses tanya-jawab berbasis dokumen yang telah diunggah oleh pengguna. Sistem tidak dirancang untuk menjawab pertanyaan di luar konteks jurnal yang diberikan.

  \item Fitur pembuatan kutipan (sitasi) otomatis hanya bekerja berdasarkan metadata jurnal yang berhasil diekstrak dari dokumen PDF.\@ Ketepatan gaya kutipan mengikuti format yang telah ditentukan sistem, yaitu APA, MLA, IEEE, dan Harvard.

  \item Fitur anotasi hanya berlaku pada dokumen PDF yang diunggah, dan hanya mencakup penandaan (highlight) dan penambahan catatan teks sebagai bentuk pencatatan pribadi pengguna. Fitur ini tidak mencakup kolaborasi atau anotasi multi-user.

  \item Penelitian ini berfokus pada pengembangan dan pengujian aplikasi secara fungsional (fungsi berjalan sesuai kebutuhan pengguna) dan usability (kemudahan penggunaan). Evaluasi tidak mencakup aspek performa teknis tingkat lanjut seperti pengukuran beban server atau stress testing.

  \item Model bahasa besar (LLM) yang digunakan adalah model yang tersedia secara komersial atau open-source melalui integrasi dengan framework Langchain. Penelitian ini tidak mencakup pelatihan model AI dari awal.
\end{enumerate}

\section{Tujuan Penelitian}

Penelitian ini bertujuan untuk merancang dan mengembangkan sebuah aplikasi chatbot berbasis jurnal ilmiah yang dapat membantu mahasiswa dalam proses riset akademik. Aplikasi ini tidak berfungsi sebagai pencari jurnal, melainkan sebagai asisten cerdas yang mendukung mahasiswa dalam memahami, mengelola, dan mencatat isi jurnal ilmiah yang telah mereka unggah. Adapun tujuan khusus dari penelitian ini adalah sebagai berikut:
\begin{enumerate}
  \item Mengembangkan chatbot interaktif yang dapat menjawab pertanyaan mahasiswa berdasarkan isi jurnal ilmiah yang mereka unggah, guna mempermudah proses pemahaman terhadap konten jurnal tersebut.
  \item Menerapkan metode Retrieval-Augmented Generation (RAG) dan framework Langchain untuk memungkinkan sistem memberikan jawaban yang kontekstual dan relevan terhadap isi dokumen yang telah diunggah.
  \item Membangun fitur pembuat sitasi otomatis dari jurnal yang diunggah oleh mahasiswa dengan dukungan berbagai gaya referensi (seperti APA, MLA, IEEE, dan Harvard), untuk mempermudah proses penulisan kutipan dalam karya ilmiah.
  \item Mengembangkan fitur anotasi pada file PDF jurnal, agar mahasiswa dapat memberikan catatan pribadi secara langsung di dalam dokumen, sebagai bagian dari proses pencatatan dan pemahaman literatur.
  \item Menyediakan antarmuka aplikasi yang mudah digunakan, sehingga dapat mendukung mahasiswa dalam kegiatan literatur review, penyusunan referensi, dan pengelolaan catatan riset secara terintegrasi.
\end{enumerate}

\section{Sistematika Penulisan}

Sistematika yang digunakan dalam penulisan ini mencakup beberapa bagian
penting, di antaranya:
\begin{enumerate}
  \item PENDAHULUAN

        Bab ini berisi pendahuluan yang menguraikan latar belakang masalah
        yang melandasi pemilihan topik penelitian, didukung oleh data relevan
        dan perbandingan dengan penelitian terdahulu, yang menyoroti kelemahan
        atau perbedaan pendekatan. Ruang lingkup penelitian dibatasi secara
        jelas untuk fokus pada persoalan yang dikaji. Tujuan penelitian ini
        adalah untuk menjawab masalah penelitian dan menghasilkan luaran yang
        diharapkan. Sistematika penulisan laporan skripsi ini akan disusun
        secara naratif, dimulai dari bab pendahuluan yang menguraikan konteks
        penelitian, diikuti oleh tinjauan pustaka yang membahas landasan teoretis,
        metodologi penelitian yang menjelaskan pendekatan dan teknik pengumpulan
        data, hasil penelitian dan pembahasan yang menganalisis temuan, hingga
        kesimpulan dan saran yang merangkum hasil penelitian dan implikasinya.
  \item TINJAUAN PUSTAKA

        Bab ini merupakan tinjauan pustaka yang disusun untuk memberikan landasan
        teoretis dan empiris yang kokoh dalam mendukung pendekatan pemecahan
        masalah yang diusulkan. Bab ini menguraikan secara komprehensif hasil
        penelitian sejenis yang relevan dengan tema yang dipilih, dengan menekankan
        pada kontribusi penelitian-penelitian tersebut terhadap pengembangan
        kerangka teoretis dan metodologis penelitian ini. Selain itu, tinjauan
        pustaka ini juga berfungsi untuk memperkuat argumentasi penelitian
        dengan menunjukkan bukti-bukti empiris dari penelitian sebelumnya,
        serta mengidentifikasi celah penelitian yang akan diisi oleh penelitian
        ini. Tinjauan pustaka ini tidak hanya merangkum literatur yang ada,
        tetapi juga menganalisis dan mensintesisnya untuk membangun dasar
        yang kuat bagi penelitian ini.
  \item HASIL DAN PEMBAHASAN

        Bab ini menyajikan hasil penelitian yang diperoleh, disertai dengan
        analisis mendalam untuk menjawab tujuan penelitian yang telah ditetapkan.
        Pada kasus penelitian yang menghasilkan rancangan, bab ini memuat
        deskripsi rinci hasil rancangan tersebut, termasuk evaluasi komprehensif
        terhadap kelebihan dan keterbatasan yang teridentifikasi. Analisis
        ini dilakukan secara sistematis dan objektif, dengan mengacu pada
        kerangka teoretis dan metodologis yang telah ditetapkan sebelumnya.
  \item PENUTUP

        Bagian penutup ini terdiri dari kesimpulan, yang merangkum jawaban
        atas masalah penelitian berdasarkan temuan yang diperoleh, dan saran,
        yang mengusulkan pengembangan lebih lanjut dari hasil penelitian yang
        telah dipaparkan dalam kesimpulan.
\end{enumerate}
\chapter{TINJAUAN PUSTAKA}

\section{Kajian Penelitian}

Kajian penelitian dilakukan untuk menelaah berbagai studi terdahulu yang berkaitan dengan pemanfaatan teknologi Artificial Intelligence (AI), khususnya Large Language Models (LLM), dalam mendukung kegiatan akademik seperti memahami literatur ilmiah, membuat sitasi, serta melakukan anotasi dokumen. Fokus utama dari kajian ini adalah studi yang membahas penerapan metode \textit{Retrieval-Augmented Generation (RAG)}, framework Langchain, dan integrasi LLM dalam konteks aplikasi berbasis chatbot untuk riset akademik.

\singlespacing
Studi-studi terdahulu yang dikaji dalam penelitian ini dipilih karena memiliki keterkaitan erat dengan topik yang diangkat, baik dari sisi teknologi maupun tujuan penggunaannya. Beberapa penelitian menunjukkan bahwa RAG efektif digunakan dalam memperbaiki akurasi jawaban model AI terhadap konteks dokumen tertentu, serta meningkatkan relevansi jawaban terhadap pertanyaan pengguna. Sementara itu, Langchain telah terbukti sebagai framework yang fleksibel dan dapat diintegrasikan dengan berbagai sistem vektor dan LLM untuk membangun aplikasi cerdas yang modular.

Selain itu, terdapat juga penelitian yang membahas pengembangan sistem pendukung pembelajaran atau riset berbasis chatbot, baik dalam konteks pendidikan formal maupun dalam proses penulisan karya ilmiah. Penelitian-penelitian ini menjadi acuan dalam perancangan sistem yang diusulkan, sekaligus sebagai pembanding untuk menilai keunikan serta kontribusi dari aplikasi yang dikembangkan dalam tugas akhir ini.

\vspace{10pt}

Tabel \ref{tab:kajian-penelitian} berikut menyajikan ringkasan hasil kajian terhadap beberapa penelitian terdahulu yang relevan, mencakup tujuan, metode, dan teknologi yang digunakan, serta kontribusi masing-masing studi terhadap pengembangan sistem yang diusulkan.

\vspace{-10pt}

\begin{table}[H]
  \caption{Rangkuman Penelitian Terdahulu}
  \label{tab:kajian-penelitian}
  \centering
  \begin{longtable}{|>{\centering\arraybackslash}m{0.5cm}|>{\raggedright\arraybackslash}p{3cm}|>{\raggedright\arraybackslash}p{3.5cm}|>{\raggedright\arraybackslash}p{5cm}|}
    \hline
    \textbf{No} & \textbf{Peneliti} & \textbf{Judul Penelitian} & \textbf{Hasil} \\
    \hline
    \endfirsthead

    \hline
    \textbf{No} & \textbf{Peneliti} & \textbf{Judul Penelitian} & \textbf{Hasil} \\
    \hline
    \endhead

    \hline
    1 & \cite{Dardjito} & \textit{Challenges in reading English academic texts for non-English major students of an Indonesian university} & Hasil utama menunjukkan bahwa tantangan terbesar bagi mahasiswa adalah ketergantungan mereka pada strategi penerjemahan kata per kata. Strategi ini, yang berakar pada keterbatasan penguasaan kosakata dan tata bahasa, justru menjadi penghalang utama dalam memahami teks secara efektif karena terjemahan harfiah sering kali tidak menghasilkan makna yang koheren. \\
    \hline
    2 & \cite{kong2024pedagogical} & \textit{A pedagogical design for self-regulated learning in academic writing using text-based generative artificial intelligence tools: 6-P pedagogy of plan, prompt, preview, produce, peer-review, portfolio-tracking} & Penelitian ini menunjukkan bahwa pendekatan 6-P pedagogy efektif dalam meningkatkan kemampuan berpikir kritis dan regulasi diri mahasiswa dalam penulisan akademik berbasis \textit{Generative AI}. Melalui enam tahap seperti prompting, preview, dan peer-review, mahasiswa didorong untuk menggunakan AI secara etis dan reflektif. Model ini juga memberi pendidik kerangka kerja untuk menjaga keseimbangan antara inovasi dan integritas akademik. \\
    \hline
    3 & \cite{Song2025Interactions} & \textit{Interactions with generative AI chatbots: unveiling dialogic dynamics, students' perceptions, and practical competencies in creative problem-solving} & Chatbot berbasis \textit{Generative AI} terbukti meningkatkan kemampuan pemecahan masalah kreatif mahasiswa, dengan interaksi berbasis pengetahuan yang lebih kompleks dan mendalam dibandingkan dengan diskusi antarmahasiswa. \\
    \hline
  \end{longtable}
\end{table}
\section{\textit{World-Wide Web}}
Sejak awal kemunculannya, internet telah berkembang secara signifikan. Awalnya dirancang untuk berbagi dokumen, internet telah berkembang menjadi jaringan kompleks yang mendukung berbagai aplikasi, layanan, dan interaksi. Franco (2021) mencatat bahwa internet lebih dari sekadar kumpulan halaman web; ini adalah platform dinamis yang memfasilitasi komunikasi, perdagangan, dan pembentukan komunitas. Memahami struktur dan fungsionalitas internet sangat penting bagi pengembang yang ingin menciptakan aplikasi yang efisien dan ramah pengguna.
\singlespacing{}
Evolusi internet dapat dibagi menjadi beberapa fase yang berbeda. Web 1.0 ditandai dengan halaman web statis, sementara Web 2.0 memperkenalkan interaktivitas dan konten yang dibuat oleh pengguna. Peralihan menuju Web 3.0, atau web semantik, bertujuan untuk meningkatkan konektivitas dan kegunaan data. Franco (2021) membahas bagaimana fase-fase ini telah mempengaruhi pendekatan pengembangan web, yang mengarah pada adopsi standar dan teknologi baru yang meningkatkan pengalaman pengguna dan aksesibilitas.
\singlespacing{}
Dalam pengembangan web, HTML, CSS, dan JavaScript adalah teknologi penting. HTML mendefinisikan struktur halaman web, CSS mengatur gaya dan tata letak elemen, dan JavaScript menambah interaktivitas dan memungkinkan pembuatan konten dinamis. Buku ini menyoroti pentingnya HTML5, yang memperkenalkan elemen dan API baru yang memfasilitasi integrasi multimedia dan meningkatkan keterlibatan pengguna. Misalnya, elemen <canvas> memungkinkan pembuatan grafik dinamis, memungkinkan pengembang untuk membangun aplikasi yang kaya secara visual.

\section{\textit{HTML}}
Sejak awal kemunculannya, internet telah berkembang secara signifikan. Awalnya dirancang untuk berbagi dokumen, internet telah berkembang menjadi jaringan kompleks yang mendukung berbagai aplikasi, layanan, dan interaksi.\@ \citet{franco2021html} mencatat bahwa internet lebih dari sekadar kumpulan halaman web; ini adalah platform dinamis yang memfasilitasi komunikasi, perdagangan, dan pembentukan komunitas. Memahami struktur dan fungsionalitas internet sangat penting bagi pengembang yang ingin menciptakan aplikasi yang efisien dan ramah pengguna.
\singlespacing{}
Evolusi internet dapat dibagi menjadi beberapa fase yang berbeda. Web 1.0 ditandai dengan halaman web statis, sementara Web 2.0 memperkenalkan interaktivitas dan konten yang dibuat oleh pengguna. Peralihan menuju Web 3.0, atau web semantik, bertujuan untuk meningkatkan konektivitas dan kegunaan data.\@ \citet{franco2021html} membahas bagaimana fase-fase ini telah mempengaruhi pendekatan pengembangan web, yang mengarah pada adopsi standar dan teknologi baru yang meningkatkan pengalaman pengguna dan aksesibilitas.
\singlespacing{}
Dalam pengembangan web, HTML, CSS, dan JavaScript adalah teknologi penting. HTML mendefinisikan struktur halaman web, CSS mengatur gaya dan tata letak elemen, dan JavaScript menambah interaktivitas dan memungkinkan pembuatan konten dinamis. Buku ini menyoroti pentingnya HTML5, yang memperkenalkan elemen dan API baru yang memfasilitasi integrasi multimedia dan meningkatkan keterlibatan pengguna. Misalnya, elemen <canvas> memungkinkan pembuatan grafik dinamis, memungkinkan pengembang untuk membangun aplikasi yang kaya secara visual.

\section{\textit{CSS (Cascading Style Sheets)}}
CSS, atau Cascading Style Sheets, adalah bahasa stylesheet yang digunakan untuk mendeskripsikan presentasi dokumen yang ditulis dalam HTML atau XML (termasuk dialek XML seperti SVG, MathML, atau XHTML). Menurut MDN Web Docs, CSS adalah teknologi landasan World Wide Web, bersama dengan HTML dan JavaScript, yang memungkinkan pemisahan konten dari desain, memungkinkan pengembang mengontrol tata letak, warna, font, dan keseluruhan presentasi visual halaman web.\@ di berbagai perangkat dan ukuran layar \citep{mozilla2025mdn}.
\singlespacing{}
CSS bekerja dengan mengasosiasikan aturan gaya dengan elemen HTML.\@ Aturan-aturan ini dapat didefinisikan dalam stylesheet eksternal, tertanam dalam dokumen HTML, atau sejajar dengan elemen HTML tertentu. Fleksibilitas ini memungkinkan pendekatan modular terhadap desain web, di mana gaya dapat digunakan kembali dan dipelihara dengan lebih mudah. Selain itu, CSS memperkenalkan konsep cascading, di mana beberapa aturan gaya dapat diterapkan pada elemen yang sama, dengan konflik diselesaikan berdasarkan kekhususan dan urutan sumber. Fitur ini memungkinkan desain canggih dan berlapis yang beradaptasi dengan berbagai konteks dan preferensi pengguna \citep{mozilla2025mdn}.
\singlespacing{}
Selain itu, CSS mendukung berbagai jenis dan fitur media, memungkinkan praktik desain responsif. Kueri media, misalnya, memungkinkan gaya beradaptasi berdasarkan karakteristik perangkat, seperti resolusi layar, orientasi, atau kedalaman warna. Kemampuan ini sangat penting untuk membuat situs web yang memberikan pengalaman pengguna yang konsisten dan optimal di berbagai perangkat, mulai dari ponsel hingga komputer desktop \citep{mozilla2025mdn}.

\subsection{\textit{TailwindCSS}}
Tailwind CSS adalah sistem CSS yang memberikan prioritas pada kelas utilitas untuk secara efisien membangun pengalaman pengguna yang dipersonalisasi. Kelas-kelas utilitas yang disediakan oleh alat ini memungkinkan gaya langsung terhadap elemen-elemen dalam HTML, menawarkan alternatif yang lebih fleksibel dan berkelanjutan dibandingkan dengan kerangka kerja CSS konvensional. Tailwind CSS bertujuan untuk menyediakan seperangkat komponen yang komprehensif untuk membangun kerangka desain. Kerangka kerja ini dapat disesuaikan dan diperluas sesuai dengan kebutuhan unik proyek Anda, semuanya di dalam kode HTML Anda. Pendekatan ini menghilangkan kebutuhan untuk membuat CSS khusus sementara tetap menawarkan fungsionalitas gaya yang kuat \citep{tailwind}.

\section{\textit{JavaScript}}
JavaScript diperkenalkan pada tahun 1995 sebagai cara untuk menambahkan program ke halaman web di browser Netscape Navigator. Bahasa ini sejak itu telah diadopsi oleh semua browser web grafis utama lainnya. Ini telah membuat aplikasi web modern menjadi mungkin, yaitu aplikasi yang dapat Anda interaksikan secara langsung tanpa perlu memuat ulang halaman untuk setiap tindakan. JavaScript juga digunakan di situs web yang lebih tradisional untuk menyediakan berbagai bentuk interaktivitas dan kecerdikan \citep{haverbeke2024eloquent}. Awalnya dirancang untuk scripting sisi klien, kemampuan JavaScript berkembang dengan diperkenalkannya Node.js, yang memungkinkan pengembang membuat aplikasi sisi server yang berbasis event dengan mudah \citep{jartarghar2022react}. Evolusi ini mengubah JavaScript dari bahasa scripting sederhana menjadi alat serbaguna untuk mengembangkan solusi web yang kompleks.
Dalam JavaScript, tipe data sangat penting untuk memahami cara kerja bahasa dan cara menangani berbagai jenis nilai.

\section{\textit{TypeScript}}
TypeScript adalah bahasa pemrograman yang didasarkan pada JavaScript dan memiliki pengetikan yang kuat. Hal ini bertujuan untuk meningkatkan dan mengoptimalkan proses pengembangan aplikasi berskala besar dengan mengkompilasi ke dalam JavaScript biasa. TypeScript menambahkan definisi tipe statis ke JavaScript, sehingga pengembang dapat mengidentifikasi dan memperbaiki masalah tipe selama kompilasi, bukan selama eksekusi. Hasilnya, kode menjadi lebih tangguh dan lebih mudah dipelihara karena sistem pemeriksaan tipe TypeScript dapat mengidentifikasi kesalahan pemrograman yang umum terjadi, seperti tipe data yang tidak konsisten dan penggunaan fungsi yang salah, pada tahap awal pengembangan. Selain itu, TypeScript menyediakan dukungan untuk fungsi JavaScript kontemporer dan dengan mudah diintegrasikan dengan basis kode JavaScript yang sudah ada sebelumnya, sehingga proses adopsi dapat dilakukan secara bertahap tanpa perlu menulis ulang secara total. Fitur ini menjadikannya alat pragmatis untuk meningkatkan skalabilitas dan ketergantungan program yang rumit sambil mempertahankan kemampuan beradaptasi dan keterusterangan yang melekat pada JavaScript \citep{typescript2025handbook}.
\section{Basisdata}
Basisdata adalah kumpulan data yang terorganisir dan terstruktur yang dirancang untuk menyimpan, mengelola, dan mengubah data secara efektif. Database memungkinkan berbagai tugas seperti pencarian, pembaruan, dan penghapusan data, serta menyediakan penyimpanan data dalam format yang mudah diakses dan dikelola. Database biasanya terdiri dari tabel yang terdiri dari baris dan kolom, dan setiap tabel berisi informasi tentang entitas tertentu. Sebuah tabel yang menyimpan informasi tentang pelanggan, misalnya, mungkin berisi nama, alamat, dan nomor telepon pelanggan. Pengguna dapat melakukan operasi SQL untuk berinteraksi dengan data yang tersimpan dalam database dengan menggunakan sistem manajemen basisdata \citep{santoso2021sql}.

\section{\textit{Entity Relationship Diagram}}
Entity Relationship Diagram (ERD) adalah komponen penting dalam bidang desain database, yang berfungsi sebagai representasi visual dari struktur data dan hubungan antara berbagai entitas dalam suatu sistem. Menurut \citet{bagui2023database}, diagram ER berperan penting dalam pemodelan data, memberikan kerangka kerja yang jelas untuk memahami bagaimana elemen data berinteraksi dan berhubungan satu sama lain dalam lingkungan database.

\begin{figure}[htbp]
  \centering
  \includegraphics[width=0.85\linewidth]{images/bab-2/erd-example.png}
  \caption{Contoh \emph{ERD}}\label{fig:erd-example}\citep{bagui2023database}
\end{figure}

Elemen utama diagram ER mencakup entitas, atribut, dan hubungan. Entitas didefinisikan sebagai objek atau konsep yang memiliki keberadaan berbeda dan dapat diidentifikasi dalam domain yang diminati. Misalnya, dalam database universitas, entitas mungkin menyertakan ``Mahasiswa'', ``Kursus'', dan ``Instruktur''. Setiap entitas dicirikan oleh atributnya, yang merupakan properti spesifik atau rincian yang menggambarkan entitas tersebut. Misalnya, entitas ``Pelajar'' mungkin memiliki atribut seperti ``ID Pelajar'', ``Nama'', dan ``Tanggal Lahir'' \citep{bagui2023database}.

Relasi dalam diagram ER menggambarkan bagaimana entitas saling berhubungan. Mereka dapat diklasifikasikan berdasarkan jumlah entitas yang terlibat: hubungan unary (satu entitas), biner (dua entitas), dan ternary (tiga entitas). Contoh umum dari hubungan biner adalah antara \"Siswa\" dan \"Kursus\", yang menunjukkan bahwa seorang siswa dapat mendaftar di beberapa kursus, sementara setiap kursus dapat memiliki beberapa siswa yang terdaftar \citep{bagui2023database}. Hubungan dua arah ini penting untuk memodelkan interaksi dalam database secara akurat.

Model ER sangat dihargai karena sifatnya yang intuitif, sehingga dapat diakses oleh pemangku kepentingan teknis dan non-teknis. Ini memberikan cara mudah untuk memvisualisasikan kebutuhan dan hubungan data, yang sangat penting untuk desain database yang efektif \citep{bagui2023database}. Namun, terlepas dari kelebihannya, beberapa pendidik mencatat bahwa siswa sering kesulitan dalam menerapkan konsep ER pada skenario dunia nyata, sehingga menyoroti perlunya contoh dan latihan praktis dalam pengajaran \citep{bagui2023database}.

Kesimpulannya, diagram ER adalah alat dasar dalam desain database, memfasilitasi pemahaman struktur dan hubungan data. Mereka berfungsi sebagai jembatan antara desain konseptual database dan implementasi fisiknya, memastikan bahwa database akhir memenuhi kebutuhan informasi penggunanya.

\section{\textit{SQL}}
Structured Query Language adalah bahasa yang digunakan untuk mengelola dan memanipulasi data dalam sistem manajemen basis data relasional. SQL memungkinkan pengguna untuk melakukan berbagai operasi, seperti pengambilan data, pembaruan, decreasing, dan penghapusan data, serta mendefinisikan struktur data dan mengelola akses ke data tersebut. SQL merupakan standar industri yang diadopsi secara luas untuk interaksi dengan basis data \citep{santoso2021sql}.
\citet{santoso2021sql} menjelaskan berbagai jenis perintah SQL yang dikelompokkan ke dalam beberapa kategori utama, yaitu DDL, DML, DCL, dan lainnya. Berikut adalah penjelasan singkat mengenai masing-masing kategori:
\begin{enumerate}
  \item DDL (Data Definition Language): DDL digunakan untuk mendefinisikan struktur database. Perintah-perintah dalam DDL termasuk:
        \begin{itemize}
          \item ``CREATE'': untuk membuat objek database seperti tabel, indeks, dan skema.
          \item ``ALTER'': untuk mengubah struktur objek yang sudah ada.
          \item ``DROP'': untuk menghapus objek dari database.
        \end{itemize}
        DDL berfokus pada bagaimana data disimpan dan diorganisir dalam database.
  \item DML (Data Manipulation Language): DML digunakan untuk mengelola data dalam objek database. Perintah-perintah dalam DML termasuk:
        \begin{itemize}
          \item ``SELECT'': untuk mengambil data dari tabel.
          \item ``INSERT'': untuk menambahkan data baru ke dalam tabel.
          \item ``UPDATE'': untuk memperbarui data yang sudah ada.
        \end{itemize}
        DML berfokus pada manipulasi data yang ada dalam database.
  \item DCL (Data Control Language): DCL digunakan untuk mengontrol akses ke data dalam database. Perintah-perintah dalam DCL termasuk:
        \begin{itemize}
          \item ``GRANT'': untuk memberikan hak akses kepada pengguna atau peran tertentu.
          \item ``REVOKE'': untuk mencabut hak akses yang telah diberikan.
        \end{itemize}
        DCL berfokus pada keamanan dan kontrol akses terhadap data.
  \item TCL (Transaction Control Language): TCL digunakan untuk mengelola transaksi dalam database. Perintah-perintah dalam TCL termasuk:
        \begin{itemize}
          \item ``COMMIT'': untuk menyimpan semua perubahan yang dilakukan dalam transaksi.
          \item ``ROLLBACK'': untuk membatalkan perubahan yang belum disimpan.
          \item ``SAVEPOINT'': untuk menetapkan titik pemulihan dalam transaksi.
        \end{itemize}
        TCL berfokus pada pengelolaan transaksi untuk memastikan integritas data.
\end{enumerate}

\section{\textit{PostgreSQL}}
PostgreSQL adalah sistem basis data objek-relasional sumber terbuka yang kuat, yang memperluas bahasa SQL dengan berbagai fitur untuk menangani beban kerja data yang kompleks dengan aman dan efisien. Pengembangan PostgreSQL dimulai pada tahun 1986 sebagai bagian dari proyek POSTGRES di Universitas California, Berkeley, dan telah mendapatkan lebih dari 35 tahun pengembangan aktif pada platform intinya.
\singlespacing{}
Dikenal karena arsitekturnya yang andal, integritas data, fitur-fitur yang luas, dan fleksibilitas, PostgreSQL didukung oleh komunitas sumber terbuka yang berdedikasi untuk memberikan solusi inovatif dan berkinerja tinggi secara konsisten. PostgreSQL kompatibel dengan semua sistem operasi utama, telah mematuhi standar ACID sejak tahun 2001, dan menawarkan ekstensi yang kuat seperti PostGIS yang populer untuk data geospasial. Oleh karena itu, tidak mengherankan bahwa PostgreSQL telah menjadi basis data relasional sumber terbuka pilihan bagi banyak individu dan organisasi \citep{postgresql2025}.

\section{\textit{NoSQL}}
NoSQL (sering diartikan sebagai ``Not Only SQL'') mengacu pada kelas teknologi penyimpanan data non-relasional yang dirancang untuk mengatasi keterbatasan pada basis data relasional tradisional, terutama dalam konteks aplikasi modern, bervolume tinggi, dan berubah dengan cepat. Tidak seperti sistem berbasis SQL, basis data NoSQL tidak memerlukan skema yang kaku dan dapat secara efisien menangani data semi-terstruktur atau tidak terstruktur \citep[hal.~3]{suehring2021redis}.
\singlespacing{}
Ada empat jenis utama basis data NoSQL{:} penyimpanan \emph{key-value}, penyimpanan dokumen, penyimpanan \emph{wide-column}, dan basis data grafik. Masing-masing dioptimalkan untuk kasus penggunaan yang berbeda-misalnya, penyimpanan \emph{key-value} ideal untuk pencarian cepat, sedangkan basis data grafik cocok untuk aplikasi dengan hubungan yang kompleks seperti jejaring sosial \citep[hal.~4-5]{suehring2021redis}.
Basis data NoSQL sangat berguna ketika aplikasi membutuhkan:
\begin{itemize}
  \item Performa tinggi di bawah beban data berskala besar,
  \item Fleksibilitas skema,
  \item Skalabilitas horizontal melalui sharding,
  \item Ketersediaan tinggi dan toleransi kesalahan \citep[hal.~6-7]{suehring2021redis}.
\end{itemize}


Tidak seperti basis data relasional yang biasanya membutuhkan gabungan yang rumit dan skema yang dinormalisasi, NoSQL menawarkan model yang lebih ramah pengembang. Hal ini penting untuk praktik pengembangan yang gesit di mana model data dapat berkembang dari waktu ke waktu \citep[hal.~6]{suehring2021redis}.

\subsection{Redis}
Redis adalah database NoSQL in-memory sumber terbuka yang menawarkan performa luar biasa cepat dengan menyimpan data terutama di RAM, dengan disk persistence opsional untuk durabilitas \citep[p.~7]{suehring2021redis}. Meskipun Redis sering dikategorikan sebagai key-value store, sebenarnya Redis adalah database multi-model, yang mampu menyimpan dan memanipulasi berbagai struktur data seperti:
\begin{itemize}
  \item Strings
  \item Lists
  \item Sets
  \item Hashes
  \item Sorted Sets
  \item Streams
  \item Geospatial data
  \item Bitmaps dan HyperLogLogs \citep[pp.~32-38]{suehring2021redis}
\end{itemize}
\singlespacing{}
Redis mendukung messaging Pub/Sub, analitik real-time, manajemen sesi, dan pelacakan leaderboard, menjadikannya ideal untuk aplikasi web dan seluler modern yang menuntut interaktivitas real-time \citep[pp.~11-15]{suehring2021redis}.
\singlespacing{}
Selain itu, Modul Redis memperluas kemampuannya untuk mencakup fitur-fitur seperti pencarian teks lengkap (melalui RediSearch), analisis time-series (RedisTimeSeries), dan querying graph (RedisGraph), memungkinkan Redis untuk mendukung pola akses data multi-model tanpa lapisan kompleksitas tambahan \citep[pp.~41-43]{suehring2021redis}. Karena sifat in-memory dan struktur data yang dioptimalkan, Redis menawarkan waktu respons sub-millisecond dan throughput yang luar biasa, sehingga cocok tidak hanya untuk caching tetapi juga sebagai database utama dalam skenario kinerja tinggi \citep[p.~12]{suehring2021redis}.
\singlespacing{}
Redis juga unggul dalam high availability dan kasus penggunaan sistem terdistribusi, mendukung replikasi, clustering, dan mekanisme automatic failover, sehingga menjadikannya siap produksi untuk aplikasi-aplikasi kritis \citep[p.~46]{suehring2021redis}.
\section{\textit{Unified Modelling Language}}
Unified Modeling Language atau UML adalah bahasa pemodelan standar yang digunakan dalam rekayasa perangkat lunak untuk menspesifikasi, memvisualisasikan, mengembangkan, dan mendokumentasikan artefak sistem perangkat lunak. UML menyediakan serangkaian teknik notasi grafis untuk membuat model visual dari sistem perangkat lunak berorientasi objek. UML mencakup berbagai jenis diagram, termasuk diagram struktural (seperti diagram kelas dan diagram komponen) dan diagram perilaku (seperti diagram ``use case'' dan ``sequence diagram''), yang membantu dalam memahami arsitektur dan perilaku sistem \citep{huzar2005uml}.
\singlespacing{}
UML banyak diadopsi dalam industri karena kemampuannya memfasilitasi komunikasi antar pemangku kepentingan, termasuk pengembang, desainer, dan analis bisnis, dengan menyediakan bahasa umum untuk memodelkan sistem yang kompleks \citep{huzar2005uml}.
\singlespacing{}
UML terdiri dari sejumlah diagram yang telah distandardisasi untuk pengembangan perangkat lunak berbasis objek, menjadikannya alat berharga untuk pemodelan dan desain sistem \citep{weriza2022development}. Dalam penulisan ini penulis menggunakan 2 macam jenis di antaranya:

\subsection{Use Case Diagram}
Diagram ``use case'' adalah jenis diagram perilaku dalam Unified Modeling Language (UML) yang merepresentasikan kebutuhan fungsional dari suatu sistem. Diagram ini menggambarkan interaksi antara pengguna (aktor) dan sistem itu sendiri, menangkap berbagai cara di mana sistem dapat digunakan. Diagram ``use case'' membantu mengidentifikasi fungsionalitas sistem dan hubungan antara berbagai ``use case'', memberikan gambaran umum tentang perilaku sistem.
\singlespacing{}
Dalam diagram ``use case'', aktor digambarkan sebagai figur tongkat, sementara ``use case'' diwakili sebagai oval. Garis menghubungkan aktor dengan ``use case'' yang mereka interaksikan, menunjukkan hubungan dan interaksi. Representasi visual ini membantu dalam memahami kebutuhan sistem dan berfungsi sebagai alat komunikasi di antara para pemangku kepentingan \citep{huzar2005uml}.
\singlespacing{}
Untuk memahami diagram kasus pengguna, penting untuk mengenali simbol-simbol yang digunakan. Simbol-simbol dalam diagram kasus pengguna termasuk:
\begin{itemize}
  \item Aktor, yang digambarkan dengan ikon manusia kecil, yang mewakili pengguna atau sistem lain yang berinteraksi dengan sistem yang dimodelkan.
  \item Kasus penggunaan, yang merepresentasikan fungsi atau layanan yang disediakan oleh sistem, digambarkan dengan elips.
  \item Hubungan antara aktor dan kasus penggunaan digambarkan dengan garis-garis yang menunjukkan interaksi.
  \item Hubungan ``extend'' dan ``include'' yang menunjukkan variasi atau ketergantungan dalam fungsionalitas.
\end{itemize}
Dengan memahami simbol-simbol ini, pengembang dapat lebih efektif dalam mendokumentasikan dan menganalisis kebutuhan sistem \citep{huzar2005uml}.

\begin{figure}[htbp]
  \centering
  \includegraphics[width=0.85\linewidth]{images/bab-2/uc.png}
  \caption{Simbol ``use case'' diagram}\label{fig:use-case-symbols}\citep{ali2019}
\end{figure}

Selain untuk desain, UML juga digunakan untuk pengujian. Telah dilaporkan bahwa baik diagram UML tunggal maupun ganda digunakan dalam menghasilkan kasus uji, menunjukkan fleksibilitas UML dalam berbagai tahapan pengembangan perangkat lunak. Selain itu, UML telah diperluas dan disesuaikan dalam berbagai cara, seperti dalam pengembangan alat untuk klasifikasi otomatis gambar web dan meningkatkan spesifikasi kebutuhan dalam desain sistem \citep{weriza2022development}.

\subsection{Activity Diagram}
Dalam pengembangan perangkat lunak, diagram aktivitas adalah jenis diagram yang digunakan dalam Unified Modeling Language (UML) untuk menggambarkan aliran aktivitas atau tindakan dalam suatu sistem. Diagram UML berfungsi sebagai satu set notasi universal yang memfasilitasi deskripsi konsep dan kolaborasi di antara pengembang perangkat lunak, arsitek, dan desainer \citep{Shcherban2021}.

\begin{figure}[htbp]
  \centering
  \includegraphics[width=0.85\linewidth]{images/bab-2/act.png}
  \caption{Contoh \emph{Activity Diagram}}\label{fig:activity-diagram-example}\citep{jibrilActivityDiagram}
\end{figure}

Diagram aktivitas UML dilengkapi dengan berbagai simbol yang digunakan untuk merepresentasikan aktivitas, keputusan, ``merge'', dan aliran dalam suatu sistem. Simbol-simbol ini termasuk:
\begin{itemize}
  \item Oval untuk aktivitas.
  \item Berlian untuk keputusan.
  \item Garis panah untuk aliran kendali.
  \item Simbol lain seperti ``swimlanes'' untuk membagi tanggung jawab antara berbagai aktor atau bagian dari sistem.
\end{itemize}
Penggunaan simbol-simbol ini memungkinkan diagram untuk memberikan representasi visual yang jelas dan terstruktur dari proses yang kompleks, sehingga memudahkan pemahaman dan komunikasi antar anggota tim pengembangan. Dengan memahami dan menggunakan simbol-simbol ini dengan benar, desainer dapat memastikan bahwa diagram aktivitas mereka secara akurat mencerminkan logika dan alur kerja dari sistem yang sedang dikembangkan \citep{Casati2002}.

\begin{figure}[htbp]
  \centering
  \includegraphics[width=0.85\linewidth]{images/bab-2/act-symbols.png}
  \caption{Simbol-simbol diagram aktivitas}\label{fig:activity-diagram-symbols}\citep{Casati2002}
\end{figure}

Diagram UML, termasuk diagram aktivitas, memainkan peran penting dalam pengembangan perangkat lunak dengan menggambarkan struktur statis dan perilaku dinamis dari sistem perangkat lunak. Diagram ini penting untuk pemodelan orkestrasi layanan, mengekspresikan skenario desain perangkat lunak, dan membantu dalam fase awal siklus hidup pengembangan perangkat lunak \citep{modi2021tool,flores2022empirical}. Selain itu, diagram UML adalah alat standar yang telah diterima secara luas di industri untuk mengembangkan sistem perangkat lunak berbasis objek \citep{weriza2022development}.

\subsection{Sequence Diagram}
Diagram urutan adalah komponen penting dari Unified Modeling Language (UML) yang menyediakan representasi visual tentang bagaimana objek-objek berinteraksi dalam skenario tertentu seiring waktu. Diagram ini diklasifikasikan sebagai diagram interaksi dan sangat efektif dalam menggambarkan perilaku dinamis dari suatu sistem dengan merinci urutan pesan yang dipertukarkan antara objek-objek \citep{huzar2005uml}.

\begin{figure}[htbp]
  \centering
  \includegraphics[width=0.85\linewidth]{images/bab-2/sequence.png}
  \caption{Contoh diagram urutan}\label{fig:sequence-diagram-example}\citep{huzar2005uml}
\end{figure}
\singlespacing{}
Dalam diagram urutan, sumbu vertikal mewakili garis waktu, sedangkan sumbu horizontal mencantumkan objek-objek yang terlibat dalam interaksi. Setiap objek digambarkan sebagai ``lifeline'', yaitu garis putus-putus vertikal yang memanjang ke bawah untuk menunjukkan keberadaan objek seiring waktu. Pesan yang dipertukarkan antara objek-objek diwakili oleh panah horizontal, yang diberi anotasi dengan nama pesan dan parameter yang relevan. Urutan pesan ditunjukkan oleh penempatannya sepanjang sumbu vertikal, memungkinkan pemahaman yang jelas tentang aliran kontrol dan data \citep{huzar2005uml}.
\singlespacing{}
Diagram urutan memiliki beberapa tujuan penting dalam pemodelan sistem. Mereka membantu memperjelas interaksi yang diperlukan untuk memenuhi suatu ``use case'' tertentu, sehingga sangat berharga bagi pengembang dan pemangku kepentingan. Dengan memvisualisasikan urutan operasi, diagram ini memfasilitasi komunikasi antar anggota tim dan memberikan pemahaman yang jelas tentang bagaimana berbagai komponen sistem berkolaborasi untuk mencapai fungsionalitas yang diinginkan \citep{huzar2005uml}.
\singlespacing{}
Selain itu, diagram urutan dapat digunakan untuk mengidentifikasi potensi masalah dalam aliran interaksi, seperti pesan yang hilang atau urutan yang salah, sehingga membantu dalam penyempurnaan desain sistem. Diagram ini melengkapi diagram UML lainnya, seperti diagram ``use case'', dengan memberikan pandangan yang lebih rinci tentang interaksi yang terjadi dalam konteks suatu ``use case'' \citep{huzar2005uml}.
\subsection{Application Programming Interface}

Application Programming Interface (API) didefinisikan sebagai sekumpulan protokol dan alat yang memungkinkan berbagai aplikasi perangkat lunak untuk berkomunikasi dan berinteraksi satu sama lain. Menurut \citet{richardson2013restful}, API berfungsi sebagai perantara yang memungkinkan pengembang mengakses fungsionalitas suatu layanan atau aplikasi tanpa perlu memahami cara kerja internalnya. Abstraksi ini memfasilitasi integrasi sistem yang beragam, memungkinkan pembuatan aplikasi kompleks yang dapat memanfaatkan layanan yang ada secara efisien.
\singlespacing{}
API dapat dikategorikan ke dalam berbagai jenis, termasuk \textit{web API} yang menggunakan protokol HTTP untuk komunikasi melalui internet, dan \textit{library API} yang dirancang untuk digunakan dalam bahasa pemrograman tertentu. Desain API sangat penting karena harus intuitif dan terdokumentasi dengan baik untuk memastikan kemudahan penggunaan bagi pengembang. Selain itu, API yang dirancang dengan baik dapat meningkatkan interoperabilitas aplikasi, memungkinkan mereka bekerja bersama dengan lancar dan berbagi data secara efektif.
\singlespacing{}
Richardson dan Amundsen menekankan bahwa ``API sangat penting untuk memungkinkan pengembangan aplikasi yang dapat berinteraksi dengan layanan lain, sehingga mendorong inovasi dan kolaborasi dalam pengembangan perangkat lunak'' \citep{richardson2013restful}.
\section{OpenAI}
OpenAI adalah perusahaan riset dan pengembangan Artificial Intelligence (AI) yang berfokus pada pengembangan sistem AI yang aman dan bermanfaat bagi umat manusia. Salah satu produk utamanya adalah model bahasa generatif seperti GPT (\textit{Generative Pre-trained Transformer}), yang telah digunakan secara luas dalam berbagai aplikasi, mulai dari chatbot, penulisan otomatis, hingga sistem tanya-jawab berbasis dokumen. OpenAI juga menyediakan antarmuka pemrograman aplikasi (API) yang memungkinkan pengembang untuk mengintegrasikan model AI ini ke dalam aplikasi mereka \citep{openai2025api}.

\subsection{Retrieval-Augmented Generation}
\textit{Retrieval-Augmented Generation} (RAG) adalah sebuah kerangka kerja yang mengatasi keterbatasan model AI generatif tradisional dengan memungkinkan mereka untuk memberikan respons yang lebih akurat dan relevan secara kontekstual. Model generatif tradisional hanya menghasilkan output berdasarkan data pelatihan, yang dapat menyebabkan ketidakakuratan atau halusinasi saat menjawab pertanyaan di luar cakupan data tersebut. RAG mengatasi masalah ini dengan menggabungkan pendekatan berbasis \textit{retrieval} dengan model generatif \citep{rothman2024rag}.
\singlespacing{}
Framework RAG beroperasi dalam dua fase utama: \textit{retrieval} (pengambilan) dan \textit{generation} (generasi).

\begin{itemize}
    \item \textbf{Fase Retrieval:} Dalam fase ini, sistem mencari sumber eksternal secara real-time untuk informasi yang relevan terhadap kueri. Data yang diperoleh akan memperkaya input pengguna dan menjadi dasar untuk proses generasi jawaban. RAG dapat digunakan untuk berbagai jenis data, seperti teks, gambar, maupun audio.
    \item \textbf{Fase Generation:} Informasi hasil \textit{retrieval} kemudian dikombinasikan dengan input pengguna oleh model AI generatif untuk menghasilkan jawaban yang lebih akurat dan relevan.
\end{itemize}

\paragraph{Aspek dan Komponen Utama RAG}

\textbf{Konfigurasi RAG:}
\begin{itemize}
    \item \textit{Na\"ive RAG:} Menggunakan metode sederhana seperti pencarian kata kunci.
    \item \textit{Advanced RAG:} Menggabungkan teknik lanjutan seperti \textit{vector search} dan \textit{index-based retrieval}.
    \item \textit{Modular RAG:} Mengintegrasikan seluruh pendekatan sebelumnya dengan tambahan \textit{machine learning} dan algoritma kompleks.
\end{itemize}

\textbf{Ekosistem RAG:} 
\begin{itemize}
    \item \textbf{Data (D):} Mencakup pengumpulan, pemrosesan, penyimpanan, dan pengambilan data dari berbagai format.
    \item \textbf{Generator (G):} Bertanggung jawab atas \textit{prompt engineering} dan generasi jawaban.
    \item \textbf{Evaluator (E):} Mengevaluasi kualitas output menggunakan metrik seperti \textit{cosine similarity} dan umpan balik pengguna.
    \item \textbf{Trainer (T):} Terlibat dalam \textit{pre-training} dan \textit{fine-tuning} model generatif.
\end{itemize}

\textbf{RAG vs. Fine-tuning:} 
\begin{itemize}
    \item \textit{Parametric knowledge} mengacu pada pengetahuan yang tertanam dalam parameter model dan dapat disempurnakan melalui \textit{fine-tuning}.
    \item \textit{Non-parametric knowledge} merujuk pada data eksplisit yang disimpan dan dikueri secara langsung oleh sistem, menjadi fokus utama pendekatan RAG.
\end{itemize}

\textbf{Ketertelusuran (Traceability):} 
Keunggulan utama RAG adalah kemampuannya dalam melacak setiap output ke sumber dokumen asli, meningkatkan transparansi serta meminimalkan ketidakakuratan dan bias \citep{rothman2024rag}.

\subsection{LangChain}
LangChain adalah pustaka \textit{open-source} yang dirancang untuk memudahkan pengembangan aplikasi berbasis \textit{Large Language Models} (LLM), seperti chatbot dan agen AI, dengan menggabungkan teknik \textit{prompt engineering} serta integrasi ke sumber data eksternal dan alat bantu lainnya. LangChain menyediakan abstraksi modular dalam bentuk fungsi dan kelas yang fleksibel untuk membangun aplikasi LLM yang kompleks, baik dengan model dari penyedia komersial seperti OpenAI maupun model \textit{open-source} seperti Llama dan Gemma \citep{oshin2024learning}.

\begin{figure}[htbp]
  \centering
  \includegraphics[width=0.85\linewidth]{images/bab-2/llm-process.png}
  \caption{Empat langkah utama untuk memproses awal dokumen untuk penggunaan LLM}\label{fig:langchain-pipeline}\citep{bagui2023database}
\end{figure}
\singlespacing{}
LangChain memungkinkan pengembang tidak hanya mengirim \textit{prompt} dan menerima respons dari model bahasa, tetapi juga melakukan teknik lanjutan seperti \textit{Retrieval-Augmented Generation (RAG)}, \textit{tool calling}, dan \textit{memory management}. Dengan fitur ini, sistem dapat melakukan \textit{chain-of-thought reasoning}, mengambil informasi dari dokumen eksternal, menggunakan kalkulator, dan mengingat konteks percakapan sebelumnya.
\singlespacing{}
Salah satu kekuatan utama LangChain adalah kemampuannya untuk membungkus seluruh proses pembangunan aplikasi LLM menjadi alur kerja yang terstandarisasi dan fleksibel. Selain itu, LangChain juga mendukung pemantauan dan pengujian sistem melalui platform seperti LangSmith dan LangServe.
\singlespacing{}
Dengan jutaan unduhan bulanan dan kontribusi aktif dari komunitas, LangChain menjadi pustaka terkemuka dalam pengembangan aplikasi AI generatif dan sangat relevan digunakan dalam pengembangan aplikasi pada penelitian ini \citep{oshin2024learning}.
\section{Node.js}

Node.js beroperasi sebagai lingkungan \textit{runtime} JavaScript yang bersifat asinkron dan berbasis peristiwa, yang secara khusus dirancang untuk mengembangkan aplikasi jaringan yang dapat diskalakan. Sebagai contoh, dalam skenario dasar ``\textit{Hello World}'', beberapa koneksi dapat dikelola secara bersamaan. Fungsi \textit{callback} akan dipicu untuk setiap koneksi, namun jika tidak ada tugas aktif, Node.js akan tetap dalam keadaan siaga.

\begin{figure}[htbp]
  \centering
  \includegraphics[width=0.85\linewidth]{images/bab-2/nodejs.png}
  \caption{Empat langkah utama untuk memproses awal dokumen untuk penggunaan LLM}\label{fig:nodejs-server}\citep{nodejs2025about}
\end{figure}

Pendekatan ini sangat berbeda dari model konkurensi tradisional yang bergantung pada \textit{thread} sistem operasi. Sistem berbasis \textit{thread} sering kali tidak efisien dan lebih sulit untuk dikelola. Node.js mengatasi masalah ini dengan menghindari penggunaan penguncian (\textit{locking}) secara keseluruhan, sehingga meminimalkan kemungkinan terjadinya kebuntuan proses. 
\singlespacing{}
Sebagian besar fungsi di Node.js tidak melakukan operasi I/O secara langsung, yang berarti prosesnya jarang mengalami pemblokiran, kecuali jika menggunakan metode sinkron dari pustaka standar Node.js. Karakteristik non-blocking ini menjadikan Node.js sangat praktis dalam pengembangan sistem yang skalabel \citep{nodejs2025about}.

\subsection{NPM}
Node Package Manager (npm) adalah utilitas penting dalam ekosistem JavaScript, yang bertindak sebagai manajer paket utama untuk lingkungan runtime Node.js. Ini menyederhanakan proses penginstalan, pengelolaan, dan pertukaran kode JavaScript, sehingga proses pengembangan menjadi lebih efisien. Repositori npm memiliki koleksi paket yang sangat banyak, berfungsi sebagai sumber daya yang berharga bagi para pengembang untuk mencari potongan kode dan pustaka yang dapat digunakan kembali. Selain itu, npm menyediakan dukungan untuk semantic versioning, sebuah sistem yang memastikan konsistensi dan kompatibilitas antara dependensi proyek. Npm meningkatkan produktivitas dan meminimalkan kemungkinan konflik atau kesalahan dalam kode dengan mengotomatiskan beberapa proses yang terkait dengan manajemen ketergantungan. Pentingnya manajemen paket ini tidak hanya untuk Node.js, karena juga digunakan dalam alur kerja pengembangan front-end, termasuk yang melibatkan kerangka kerja seperti React dan Angular \citep{nodejs2025npm}.

\section{\textit{Library} yang digunakan}
Dalam pengembangan aplikasi web, penggunaan library sangatlah penting untuk mempercepat proses pengembangan dan meningkatkan produktivitas. Library menyediakan kumpulan fungsi dan metode yang siap digunakan, sehingga pengembang tidak perlu menulis kode dari awal untuk setiap fitur yang ingin ditambahkan.

\subsection{\textit{React.js}}
React.js adalah toolkit JavaScript sumber terbuka terkenal yang unggul dalam mengembangkan single-page application untuk aplikasi web \citep{react2025learn}. React.js menggunakan struktur berbasis komponen untuk membangun antarmuka pengguna aplikasi web secara efisien dan dinamis \citep{oghlukyan2022information}. Library ini memungkinkan pengembang untuk membangun antarmuka pengguna yang dinamis dan adaptif dengan memecah UI menjadi komponen yang dapat digunakan kembali dan dikelola secara mandiri \citep{react2025learn}. React.js memungkinkan pengembang untuk secara efektif memperbarui dan menampilkan komponen seiring dengan perubahan state aplikasi, sehingga meningkatkan efisiensi dan pengalaman pengguna \citep{react2025learn}.
\singlespacing{}
Virtual DOM adalah aspek penting dari React.js karena memungkinkan library ini meningkatkan proses pembaruan dengan secara selektif me-render ulang hanya komponen yang mengalami perubahan, bukan me-render ulang seluruh halaman \citep{oghlukyan2022information}. Strategi ini secara signifikan meningkatkan kecepatan dan efisiensi aplikasi web yang dibangun dengan React.js.
\singlespacing{}
Lebih lanjut, React.js menyederhanakan proses pembangunan aplikasi satu halaman dengan memungkinkan pembuatan konten dinamis yang dapat diperbarui tanpa perlu memuat ulang halaman secara keseluruhan \citep{oghlukyan2022information}. React.js menyederhanakan proses pengembangan dan memungkinkan pembuatan aplikasi web yang kompleks dengan secara efektif menangani komponen UI dan status \citep{react2025learn}.

\subsection{Next.js}
Next.js adalah sebuah framework yang yang secara khusus dibuat untuk membangun aplikasi web full-stack. Framework ini memanfaatkan komponen-komponen React untuk membuat antarmuka pengguna dan menyediakan fitur-fitur tambahan serta pengoptimalan yang lebih dari apa yang ditawarkan oleh React konvensional. Dengan mengabstraksi dan secara otomatis mengkonfigurasi tooling penting untuk React, seperti bundling dan compiling, framework ini merampingkan proses pengembangan. Hal ini memungkinkan para pengembang untuk fokus pada pembuatan aplikasi daripada berurusan dengan tugas-tugas konfigurasi yang membosankan. Framework ini memfasilitasi pengembangan aplikasi React yang interaktif, dinamis, dan berkinerja tinggi baik untuk pengembang individu maupun tim yang lebih besar. Selain itu, Next.js menggabungkan rendering sisi server dan pembuatan situs statis, yang menghasilkan peningkatan efisiensi dan pengoptimalan mesin pencari. Hal ini menjadikannya pilihan serbaguna untuk pengembangan web kontemporer \citep{nextjs2025documentation}.

\section{Software Development Life Cycle}
\emph{Software development life cycle} (SDLC) adalah proses terstruktur yang digunakan untuk merencanakan, membuat, menguji, dan menerapkan aplikasi perangkat lunak. SDLC berfungsi sebagai kerangka kerja yang menguraikan berbagai tahap yang terlibat dalam pengembangan perangkat lunak, memastikan bahwa perangkat lunak berkualitas tinggi disampaikan dengan efisien dan efektif. SDLC biasanya terdiri dari beberapa fase, termasuk analisis kebutuhan, desain, pengembangan, pengujian, penerapan, dan pemeliharaan \citep{chan2020devops}.
\singlespacing{}
Pada fase analisis kebutuhan, pemangku kepentingan mengumpulkan dan menganalisis kebutuhan serta harapan pengguna untuk mendefinisikan fungsionalitas perangkat lunak. Setelah itu, fase desain dilanjutkan, di mana arsitektur dan antarmuka pengguna perangkat lunak direncanakan. Fase pengembangan melibatkan pengkodean perangkat lunak yang sebenarnya, sementara fase pengujian memastikan bahwa perangkat lunak bebas dari cacat dan memenuhi persyaratan yang ditentukan. Setelah pengujian selesai, perangkat lunak diterapkan ke lingkungan produksi, dan pemeliharaan yang berkelanjutan dilakukan untuk menangani masalah yang muncul setelah penerapan \citep{chan2020devops}.
\singlespacing{}
SDLC sangat penting untuk mengelola kompleksitas pengembangan perangkat lunak, karena ia menyediakan peta jalan yang jelas bagi tim untuk diikuti, memfasilitasi manajemen proyek yang lebih baik dan komunikasi di antara pemangku kepentingan \citep{chan2020devops}.

\subsection{Model iteratif}

Model iteratif adalah salah satu model SDLC yang mengombinasikan proses-proses pada model waterfall dan model prototype. Model inkremental akan menghasilkan versi-versi perangkat lunak yang sudah mengalami penambahan fungsi.

\begin{figure}[htbp]
  \centering
  \includegraphics[width=0.85\linewidth]{images/bab-2/sdlc-iterative.png}
  \caption{Ilustrasi Model Iteratif}\label{fig:iterative-model}\citep{sukamto}
\end{figure}

\section{Struktur Navigasi}
Struktur navigasi situs web memainkan peran penting dalam meningkatkan pengalaman pengguna dengan memandu pengunjung ke informasi yang mereka cari. Menurut \citet{dewiyana2018website}, sistem navigasi yang terorganisir dengan baik sangat penting bagi pengguna untuk memahami lokasi mereka saat ini dalam situs web, mengidentifikasi langkah selanjutnya, dan mencapai tujuan mereka secara efektif. Ketika navigasi tidak konsisten atau dirancang dengan buruk, pengguna mungkin kesulitan menemukan informasi yang mereka butuhkan, sehingga menyebabkan frustrasi dan berpotensi menurunkan lalu lintas situs web.\@ \citet{dewiyana2018website} menyoroti beberapa elemen kunci yang berkontribusi terhadap navigasi yang efektif, termasuk penggunaan rambu-rambu dan alat pencari jalan. Penunjuk arah, seperti judul halaman dan logo, membantu pengguna mengorientasikan diri mereka dalam situs web, sementara strategi pencarian arah, termasuk penandaan yang jelas dan panduan lingkungan, membantu pengguna dalam menavigasi menuju konten yang mereka inginkan. Studi ini menekankan bahwa struktur navigasi yang seragam dapat mempengaruhi jumlah pengunjung situs web secara signifikan, karena meningkatkan kegunaan dan aksesibilitas.
\singlespacing{}
Lebih lanjut \citet{dewiyana2018website} mengkategorikan struktur navigasi ke dalam berbagai model, seperti model linier, hierarki, dan model spoke-and-hub. Setiap model menawarkan pendekatan berbeda dalam mengatur konten, yang dapat memengaruhi cara pengguna berinteraksi dengan situs web. Misalnya, model hierarki memungkinkan pengguna bernavigasi dari beranda utama ke subhalaman, menciptakan jalur yang jelas untuk eksplorasi. Sebaliknya, model linier menyajikan informasi secara berurutan, memandu pengguna melalui aliran yang telah ditentukan.
\singlespacing{}
Struktur navigasi merupakan komponen penting dari desain situs web yang secara langsung berdampak pada kepuasan dan keterlibatan pengguna.\@ \citet{dewiyana2018website} menegaskan bahwa dengan menerapkan strategi navigasi yang efektif, situs web dapat meningkatkan fungsionalitanya secara keseluruhan dan melayani pengunjungnya dengan lebih baik.

\subsection{Struktur Navigasi \textit{Composite}}
Struktur navigasi mengacu pada desain dan organisasi tata letak suatu situs web yang memfasilitasi eksplorasi pengguna dan pengambilan informasi. Struktur ini mencakup hubungan dan jalur antara berbagai elemen situs web, memungkinkan pengguna mengakses konten dengan efisien. Efektivitas struktur navigasi diukur berdasarkan seberapa baik struktur tersebut memungkinkan pengguna mencapai tujuan mereka, dengan indikator seperti konteks, tautan, kustomisasi, konten, dan peta situs menjadi penting dalam penilaian ini. Sistem navigasi yang terstruktur dengan baik tidak hanya meningkatkan pengalaman pengguna tetapi juga memainkan peran penting dalam menentukan kesuksesan keseluruhan situs web dengan membimbing pengguna secara lancar melalui kontennya \citep{dewiyana2018website}.

\begin{figure}[htbp]
  \centering
  \includegraphics[width=0.85\linewidth]{images/bab-2/composite-sitemap.png}
  \caption{Struktur navigasi \emph{composite}}\label{fig:composite-sitemap}\citep{perrina2021rancang}
\end{figure}

Selain itu, pengaturan konten dan konsep dalam aplikasi web memainkan peran penting dalam menentukan struktur navigasi keseluruhan. Penataan ini memfasilitasi presentasi pola penggunaan utama bagi pengguna dan organisasi hierarkis elemen-elemen situs web, sehingga meningkatkan pengalaman navigasi yang intuitif \citep{ceci2020closed}.

\section{\textit{Visual Studio Code}}

\emph{Visual Studio Code} (VS Code) adalah alat perangkat lunak yang dibuat oleh \emph{Microsoft} untuk mengedit \emph{source code}. Situs web resmi menggambarkannya sebagai editor kode yang efisien dan memiliki fitur untuk debugging, menjalankan tugas, dan kontrol versi. Ini banyak digunakan untuk pembuatan aplikasi web dan cloud. 
\singlespacing{}
Dalam \emph{Integrated Development Environment} (IDE), \emph{Visual Studio Code} (VS Code) telah memperoleh kepopuleran yang signifikan di kalangan pengembang perangkat lunak. Elemen inti yang berkontribusi pada fungsionalitas luas VS Code adalah arsitektur ekstensibilitasnya. Arsitektur ini memfasilitasi penggabungan ekstensi, yang merupakan komponen perangkat lunak modular yang meningkatkan kemampuan VS Code. Ekstensi ini dapat memperkenalkan dukungan untuk bahasa pemrograman baru, mengintegrasikan alat debugging, dan menyediakan berbagai utilitas yang menyederhanakan proses pengembangan perangkat lunak.\@ \citep{microsoft2025visual}

\section{\textit{Deployment}}

Deployment adalah fase penting dalam Software Development Life Cycle (SDLC) yang melibatkan pelepasan perangkat lunak yang dikembangkan ke environment production yang dapat diakses dan digunakan oleh end-user. Fase ini mengikuti pengujian ekstensif untuk memastikan bahwa perangkat lunak memenuhi standar dan fungsi yang diperlukan sebagaimana dimaksud. Strategi penerapan yang efektif sangat penting untuk meminimalkan waktu henti dan memastikan kelancaran transisi dari pengembangan ke produksi \citep{chan2020devops}.
\singlespacing{}
Selama fase deployment, berbagai aktivitas dilakukan, termasuk instalasi perangkat lunak di server, konfigurasi lingkungan, dan migrasi data jika diperlukan. Penting juga untuk memantau proses penerapan untuk mengatasi masalah apa pun yang mungkin timbul dengan cepat, memastikan bahwa perangkat lunak beroperasi dengan benar di lingkungan langsung \citep{chan2020devops}. Chan (2020) berpendapat bahwa deployment yang sukses tidak hanya melibatkan pelaksanaan teknis tetapi juga memerlukan perencanaan dan koordinasi yang cermat di antara anggota tim agar selaras dengan tujuan bisnis dan kebutuhan pengguna \citep{chan2020devops}.

\subsection{\textit{Vercel}}
\textit{Vercel} adalah platform cloud untuk situs statis dan fungsi tanpa server, yang dikenal karena fokusnya pada penerapan cepat dan integrasi tanpa batas dengan alur kerja pengembangan modern. Sebelumnya dikenal sebagai \emph{Zeit}, \emph{Vercel} menyederhanakan proses penerapan aplikasi web, terutama yang dibangun menggunakan kerangka kerja seperti \emph{Next.js}, \emph{React}, \emph{Angular}, dan \emph{Vue.js}. Ia menawarkan jaringan edge global yang memastikan pengiriman konten berkinerja tinggi, mengurangi latensi bagi pengguna di seluruh dunia \citep{vercel2025platform}.
\singlespacing{}
Salah satu fitur menonjol Vercel adalah dukungannya terhadap fungsi tanpa server (serverless), yang memungkinkan pengembang menerapkan logika backend tanpa mengelola infrastruktur server tradisional. Arsitektur tanpa server ini diskalakan secara otomatis berdasarkan permintaan, mengoptimalkan efisiensi biaya, dan menyederhanakan alur kerja penerapan. Vercel juga menyediakan platform terpadu di mana pengembang dapat mengelola penerapan frontend dan backend, menyederhanakan siklus pengembangan mulai dari perubahan kode hingga penerapan produksi \citep{vercel2025platform}.
\singlespacing{}
Selain deployment, Vercel menawarkan fitur kolaborasi yang memfasilitasi alur kerja tim, termasuk pratinjau penerapan untuk berbagi aplikasi yang sedang dalam proses dengan pemangku kepentingan. Ini terintegrasi dengan baik dengan sistem kontrol versi seperti Git, memungkinkan penerapan otomatis setelah perubahan kode, yang meningkatkan produktivitas pengembang dan mempercepat waktu pemasaran aplikasi.\@ \citep{vercel2025platform}
\section{\textit{Agile Testing}}

Pengujian Agile merupakan bagian integral dari proses pengembangan perangkat lunak Agile, yang menekankan pentingnya umpan balik dan kolaborasi berkelanjutan di antara anggota tim. Tidak seperti metode pengujian tradisional, yang sering terjadi di akhir siklus pengembangan, pengujian Agile dilakukan di se-luruh proses pengembangan. Pendekatan ini memungkinkan deteksi dini cacat dan memastikan bahwa produk perangkat lunak selaras dengan persyaratan dan harapan pengguna.
\singlespacing{}
Dalam metodologi Agile, pengujian bukanlah fase terpisah tetapi tertanam dalam siklus pengembangan, yang sering disebut sebagai iterasi atau sprint. Proses iteratif ini memungkinkan tim untuk beradaptasi dengan persyaratan yang berubah dan memasukkan umpan balik pengguna dengan segera. Keterlibatan berbagai pemangku kepentingan, termasuk pengembang, penguji, dan pengguna akhir, men-dorong lingkungan kolaboratif yang meningkatkan kualitas produk akhir \citep{pandit2015agileuat}. \textit{User Acceptance Testing} (UAT) disebutkan sebagai salah satu metode dalam pengujian Agile. Menurut \citet{pandit2015agileuat}, UAT umumnya dilakukan secara manual dan tidak disarankan untuk diotomatisasi. Kerangka kerja UAT tersedia untuk metodologi Agile seperti Scrum.
\singlespacing{}
\textit{User Acceptance Testing} (UAT) memainkan peran penting dalam pengujian Agile, karena ini adalah fase di mana pengguna akhir memvalidasi perangkat lunak terhadap kebutuhan mereka. UAT biasanya dilakukan secara manual dan tidak disarankan untuk diotomatisasi, karena berfokus pada upaya memastikan bahwa perangkat lunak sesuai dengan tujuan dari perspektif pengguna \citep{pandit2015agileuat}. Dengan mengintegrasikan UAT ke dalam kerangka kerja Agile, tim dapat memastikan bahwa perangkat lunak tidak hanya memenuhi spesifikasi teknis tetapi juga memberikan nilai kepada pengguna.

\subsection{\textit{User Acceptance Testing}}
\textit{User Acceptance Testing} (UAT) adalah tahapan penting dalam software development life cycle yang bertujuan untuk memastikan bahwa sistem yang dikembangkan memenuhi kebutuhan dan harapan pengguna akhir sebelum diimplementasikan dalam lingkungan bisnis yang sesungguhnya. UAT sering kali menjadi langkah terakhir dalam proses pengujian perangkat lunak sebelum sistem tersebut dirilis ke pengguna akhir.
\singlespacing{}
Menurut \citet{hambling2013user}, UAT merupakan pengujian yang dilakukan oleh pengguna akhir untuk memverifikasi bahwa sistem informasi baru bekerja sesuai dengan tujuan awal dan memenuhi persyaratan bisnis yang telah ditetapkan.
Dari definisi ini, terdapat tiga aspek penting yang perlu diperhatikan dalam pelaksanaan UAT:\@
\begin{enumerate}
  \item Pengujian Formal: UAT memerlukan pengujian formal, yang berarti bahwa pengujian harus dirancang dan dilaksanakan dengan cara yang terstruktur untuk memberikan bukti objektif mengenai keabsahan sistem. Ini mencakup penyiapan skenario pengujian yang sesuai dengan kebutuhan pengguna dan standar bisnis yang berlaku.
  \item Kebutuhan Pengguna dan Proses Bisnis: UAT tidak hanya fokus pada pengujian berdasarkan spesifikasi teknis, tetapi juga harus memperhatikan kebutuhan pengguna dan proses bisnis yang ada. Pengujian ini bertujuan untuk memastikan bahwa sistem dapat mendukung aktivitas sehari-hari pengguna dan membantu mencapai tujuan bisnis.
  \item Kriteria Penerimaan: Kriteria penerimaan merupakan standar yang harus dipenuhi oleh sistem agar dapat diterima oleh pengguna akhir. Kriteria ini biasanya mencakup aspek fungsionalitas, kinerja, keamanan, dan kemudahan penggunaan. UAT bertujuan untuk memastikan bahwa semua kriteria ini terpenuhi sebelum sistem diimplementasikan.
\end{enumerate}
\singlespacing{}
Pelaksanaan UAT yang efektif melibatkan partisipasi aktif dari pengguna akhir, pemangku kepentingan bisnis, dan tim pengembang. Proses ini meliputi penentuan skenario pengujian, eksekusi pengujian, pencatatan hasil pengujian, dan penyelesaian masalah yang ditemukan. Hasil dari UAT menjadi dasar bagi keputusan apakah sistem siap untuk diterapkan dalam lingkungan produksi atau memerlukan perbaikan lebih lanjut. Dengan demikian, UAT memainkan peran krusial dalam menjamin bahwa sistem yang dikembangkan benar-benar dapat memenuhi kebutuhan pengguna dan mendukung tujuan bisnis secara efektif, sebelum diluncurkan ke lingkungan produksi \citep{hambling2013user}
\chapter{METODE PENELITIAN}

\section{Analisis Kebutuhan}

Tahap awal ini dilakukan untuk mengidentifikasi kebutuhan pengguna dan sistem. Fokus utama adalah permasalahan yang dihadapi mahasiswa dalam memahami jurnal ilmiah, serta fitur yang diperlukan seperti tanya jawab berbasis dokumen, sitasi otomatis, dan anotasi PDF. Dari tahap ini diturunkan kebutuhan fungsional dan non-fungsional sebagai dasar pengembangan sistem.

\subsection{Spesifikasi Perangkat Keras}

Tabel 3.1 menunjukkan informasi lengkap seluruh perangkat keras yang
dipergunakan dalam penelitian ini.

\begin{table}[H]
  \caption{Spesifikasi Perangkat Keras yang Digunakan}

  \centering{}%
  \begin{tabular}{|l|l|}
    \hline
    \textbf{Komponen} & \textbf{Nama}\tabularnewline
    \hline
    \hline
    CPU               & Apple Chip M1\tabularnewline
    \hline
    GPU               & Apple Chip M1\tabularnewline
    \hline
    RAM               & 8GB\tabularnewline
    \hline
    \emph{Chipset}    & M1\tabularnewline
    \hline
    SSD               & 256GB\tabularnewline
    \hline
  \end{tabular}
\end{table}


\subsection{Spesifikasi Perangkat Lunak}

Tabel 3.2 menunjukkan informasi lengkap mengenai seluruh perangkat
lunak yang dipergunakan dalam penelitian ini.

\begin{table}[H]
  \caption{Spesifikasi Perangkat Lunak yang Digunakan}

  \centering{}%
  \begin{tabular}{|l|l|}
    \hline
    \textbf{Aplikasi}          & \textbf{Nama}\tabularnewline
    \hline
    \hline
    OS                         & Macintosh\tabularnewline
    \hline
    \emph{Virtual Environment} & \emph{Node Package Mananger}\tabularnewline
    \hline
    Bahasa Pemrograman         & \emph{TypeScript}\tabularnewline
    \hline
    Editor Teks                & \emph{VS Code}\tabularnewline
    \hline
    \emph{Terminal Emulator}   & \emph{Ghostty}\tabularnewline
    \hline
    \emph{Framework}           & \emph{Next.js}\tabularnewline
    \hline
    Penelusur Web              & \emph{Arc}\tabularnewline
    \hline
    \emph{Version Control}     & \emph{Git}\tabularnewline
    \hline
  \end{tabular}
\end{table}

\section{Analisis Kebutuhan}
Tahap awal ini dilakukan untuk mengidentifikasi kebutuhan pengguna dan sistem. Fokus utama adalah permasalahan yang dihadapi mahasiswa dalam memahami jurnal ilmiah, serta fitur yang diperlukan seperti tanya jawab berbasis dokumen, sitasi otomatis, dan anotasi PDF. Dari tahap ini diturunkan kebutuhan fungsional dan non-fungsional sebagai dasar pengembangan sistem.

\section{Tools dan Peralatan Penelitian}
Penelitian ini menggunakan kombinasi perangkat lunak dan layanan AI untuk membangun sistem. Beberapa alat dan teknologi utama meliputi:
\begin{itemize}
  \item \textbf{Bahasa pemrograman:} JavaScript (Node.js), TypeScript (Next.js).
  \item \textbf{Framework AI:} Langchain untuk integrasi LLM dan retrieval.
  \item \textbf{Model AI:} OpenAI GPT-4 melalui API resmi.
  \item \textbf{Basis data:} Supabase PostgreSQL untuk data struktural, dan Chroma sebagai vektor store.
  \item \textbf{\emph{Tools} pendukung}:
        \begin{itemize}
          \item \emph{Postman} untuk API \emph{testing}.
          \item \emph{Visual Studio Code} untuk \emph{development}.
          \item \emph{GitHub} untuk kontrol versi.
        \end{itemize}
\end{itemize}

\section{Perancangan Sistem}
Pada tahap desain, dilakukan:
\begin{itemize}
  \item Perancangan arsitektur sistem yang mengintegrasikan \textit{front-end}, \textit{back-end}, dan layanan AI eksternal.
  \item Perancangan alur proses RAG menggunakan \textit{Langchain}, yang menghubungkan retriever, dokumentasi, dan model LLM.
  \item Desain tampilan antarmuka pengguna (\textit{user interface}) untuk memudahkan mahasiswa dalam mengunggah artikel ilmiah, melakukan interaksi tanya-jawab, membuat sitasi, dan memberi anotasi pada PDF.
  \item Perancangan struktur database menggunakan \emph{PostgreSQL} dan \emph{Redis} untuk menyimpan metadata, histori interaksi, dan catatan pengguna.
\end{itemize}

\subsection{Arsitektur Sistem}
Sistem dirancang dalam arsitektur berbasis layanan terpisah (\textit{modular}) untuk memastikan fleksibilitas dan skalabilitas. Gambar~\ref{fig:arsitektur-sistem} menyajikan arsitektur umum sistem yang dikembangkan.

\begin{figure}[H]
  \centering
  \includegraphics[width=0.9\textwidth]{images/system-arch.png}
  \caption{Arsitektur sistem chatbot dengan RAG dan Langchain}
  \label{fig:arsitektur-sistem}
\end{figure}

\subsection{Use Case Diagram}
Bagian ini menjelaskan fungsionalitas sistem Chatbot Journal dari perspektif pengguna, yang digambarkan melalui use case diagram. Use case diagram ini mengidentifikasi aktor yang berinteraksi dengan sistem dan berbagai fungsi atau layanan yang disediakan oleh sistem.

\begin{figure}[H]
  \centering
  \includegraphics[width=0.9\textwidth]{images/bab-3/usecase.jpg}
  \caption{Use Case Diagram Aplikasi Chatbot Akademik}
  \label{fig:usecase}
\end{figure}
\subsection{Aktor}
Dalam use case diagram ini, terdapat satu aktor utama yaitu \textbf{Pengguna}. Aktor ini merepresentasikan individu yang akan berinteraksi langsung dengan sistem \textit{Journal Chatbot} untuk memenuhi berbagai kebutuhannya terkait manajemen jurnal dan interaksi dengan chatbot.

\subsection{Use Case}
Berikut adalah penjelasan rinci mengenai setiap use case yang terlibat dalam sistem \textit{Journal Chatbot}:

\begin{enumerate}
    \item \textbf{Login}
    \begin{itemize}
        \item \textbf{Deskripsi:} Use case ini memungkinkan pengguna untuk masuk ke dalam sistem dengan menggunakan kredensial yang valid. Ini adalah use case dasar yang harus dilakukan sebelum pengguna dapat mengakses fitur-fitur lain yang memerlukan otentikasi.
        \item \textbf{Relasi:} Use case \textit{Login} di-\textit{include} oleh \textit{Chat dengan Chatbot} dan \textit{Melihat History Chat}, yang berarti setiap kali pengguna ingin melakukan chat atau melihat riwayat chat, mereka harus terlebih dahulu berhasil login.
    \end{itemize}

    \item \textbf{Membuat Sitasi}
    \begin{itemize}
        \item \textbf{Deskripsi:} Use case ini memungkinkan pengguna untuk menghasilkan sitasi dari sumber jurnal atau dokumen yang telah diunggah.
        \item \textbf{Relasi:}
        \begin{itemize}
            \item \textit{Extend Salin Sitasi}: Pengguna dapat memilih untuk menyalin sitasi yang telah dibuat ke clipboard untuk penggunaan lebih lanjut.
            \item \textit{Extend Export Sitasi}: Pengguna dapat memilih untuk mengekspor sitasi ke format lain (misalnya, Bib\TeX, RIS, atau plain text) untuk pengelolaan referensi.
        \end{itemize}
    \end{itemize}

    \item \textbf{Chat dengan Chatbot}
    \begin{itemize}
        \item \textbf{Deskripsi:} Use case ini merupakan inti dari sistem, di mana pengguna dapat berinteraksi secara langsung dengan chatbot untuk mendapatkan informasi, menjawab pertanyaan, atau melakukan tugas-tugas terkait jurnal.
        \item \textbf{Relasi:}
        \begin{itemize}
            \item \textit{Include Login}: Pengguna harus login sebelum dapat berinteraksi dengan chatbot.
            \item \textit{Extend Upload PDF}: Saat berinteraksi dengan chatbot, pengguna memiliki opsi untuk mengunggah dokumen PDF yang kemudian dapat dianalisis atau digunakan oleh chatbot dalam percakapan.
            \item \textit{Extend Anotasi PDF}: Setelah mengunggah PDF, pengguna dapat melakukan anotasi pada dokumen tersebut, seperti menyorot teks, menambahkan catatan, atau menandai bagian-bagian penting. Fitur ini memperkaya interaksi dengan chatbot karena chatbot dapat merujuk pada anotasi pengguna.
        \end{itemize}
    \end{itemize}

    \item \textbf{Melihat History Chat}
    \begin{itemize}
        \item \textbf{Deskripsi:} Use case ini memungkinkan pengguna untuk melihat dan mengelola riwayat percakapan mereka dengan chatbot. Ini sangat penting untuk melacak interaksi sebelumnya dan melanjutkan percakapan.
        \item \textbf{Relasi:}
        \begin{itemize}
            \item \textit{Include Login}: Pengguna harus login untuk dapat mengakses riwayat chat mereka.
            \item \textit{Extend Share Chat}: Pengguna dapat memilih untuk membagikan riwayat chat tertentu dengan pihak lain (misalnya, melalui email atau aplikasi pesan).
            \item \textit{Extend Edit Nama Chat}: Pengguna dapat mengubah nama atau judul dari riwayat chat tertentu untuk mempermudah identifikasi dan pengelolaan.
            \item \textit{Extend Hapus Chat}: Pengguna dapat menghapus riwayat chat yang tidak lagi dibutuhkan untuk menjaga kebersihan data.
        \end{itemize}
    \end{itemize}
\end{enumerate}

\subsection{Relasi Antar Use Case}
\begin{itemize}
    \item \textbf{Relasi \textit{include}}: Menunjukkan bahwa suatu use case menyertakan fungsionalitas dari use case lain secara wajib. Dalam diagram ini, \textit{Login} di-\textit{include} oleh \textit{Chat dengan Chatbot} dan \textit{Melihat History Chat}, yang menekankan bahwa otentikasi adalah prasyarat untuk kedua fungsi tersebut.
    
    \item \textbf{Relasi \textit{extend}}: Menunjukkan bahwa suatu use case dapat memperluas fungsionalitas use case lain dalam kondisi tertentu. Contohnya, \textit{Salin Sitasi} dan \textit{Export Sitasi} adalah opsi tambahan yang tersedia setelah \textit{Membuat Sitasi}. Demikian pula, \textit{Upload PDF} dan \textit{Anotasi PDF} memperluas fungsionalitas \textit{Chat dengan Chatbot}, sementara \textit{Share Chat}, \textit{Edit Nama Chat}, dan \textit{Hapus Chat} memperluas \textit{Melihat History Chat}. Relasi ini menunjukkan fleksibilitas sistem dalam menawarkan fitur tambahan sesuai kebutuhan pengguna.
\end{itemize}

\subsection{Activity Diagram}

Activity Diagram adalah representasi grafis dari alur kerja atau kegiatan di dalam suatu sistem, yang memperlihatkan berbagai langkah yang diperlukan untuk membangun aplikasi chatbot jurnal ini. Berikut merupakan penjelasan lengkap tentang setiap fungsi yang dijalankan sistem ini.

\begin{figure}[H]
  \centering
  \includegraphics[width=0.9\textwidth]{images/bab-3/sitemap.jpg}
  \caption{Activity Diagram Interaksi Chatbot dengan RAG}
  \label{fig:activity}
\end{figure}

Activity diagram di atas menggambarkan alur proses interaksi pengguna dengan aplikasi. Proses dimulai ketika pengguna membuka web app. Sistem kemudian memeriksa apakah pengguna sudah memiliki akun. Jika belum, sistem akan menampilkan halaman sign up; jika sudah, sistem menampilkan halaman login. Pengguna kemudian mengisi kredensial yang dikirimkan ke sistem untuk proses otentikasi melalui database. Jika otentikasi berhasil, sistem akan menampilkan halaman utama. Jika gagal, sistem akan menampilkan pesan kesalahan.
\singlespacing{}
Setelah berhasil masuk, pengguna dapat memilih apakah ingin mengunggah file PDF atau tidak. Jika tidak, pengguna hanya perlu mengirimkan chat. Namun, jika ingin mengunggah PDF, pengguna akan mengirimkan chat bersamaan dengan file tersebut. File PDF akan diunggah ke object storage, kemudian payload yang berisi chat dan file akan diproses menggunakan integrasi OpenAI.\@ Setelah diproses, hasil response dikirimkan kembali ke sistem dan sistem mengirimkannya ke pengguna dalam bentuk stream teks. Di sisi lain, chat juga disimpan ke dalam database sebagai arsip interaksi.

\subsection{Sequence Diagram}

Gambar~\ref{fig:sequence-login-chat} menunjukkan sequence diagram yang menggambarkan alur interaksi antara pengguna, sistem, dan basis data saat menggunakan aplikasi chatbot jurnal. Diagram ini menjelaskan urutan proses dari mulai membuka aplikasi hingga pengguna menerima respons dari sistem berdasarkan dokumen yang telah diunggah.

\begin{figure}[H]
    \centering
    \includegraphics[width=0.9\textwidth]{images/bab-3/sequence.png}
    \caption{Sequence Diagram Interaksi User dan Sistem}
    \label{fig:sequence-login-chat}
\end{figure}

Berdasarkan Gambar~\ref{fig:sequence-login-chat}, berikut adalah tahapan proses yang terjadi:

\begin{enumerate}
  \item \textbf{Pengguna membuka aplikasi web}: Proses dimulai ketika pengguna mengakses aplikasi melalui browser.
  
  \item \textbf{Cek akun dan otentikasi}: Sistem memeriksa apakah pengguna sudah memiliki akun. Jika ya, pengguna mengisi kredensial (email dan password) lalu sistem mengirimkan kredensial tersebut ke backend untuk proses otentikasi.
  
  \item \textbf{Respons otentikasi}: Jika kredensial valid, sistem akan menampilkan halaman utama. Jika tidak, sistem menampilkan pesan error dan meminta pengguna untuk mencoba kembali.
  
  \item \textbf{Registrasi pengguna baru}: Jika pengguna belum memiliki akun, sistem mengarahkan ke halaman sign-up.
  
  \item \textbf{Pengiriman pesan dan dokumen}: Setelah berhasil masuk, pengguna dapat melakukan percakapan dengan chatbot. Jika dibutuhkan, pengguna juga dapat mengunggah file PDF berisi jurnal ilmiah.
  
  \item \textbf{Upload PDF ke Object Storage}: Jika terdapat PDF yang dikirimkan, sistem akan mengunggah file tersebut ke layanan object storage (misalnya Vercel Blob atau storage berbasis cloud lainnya).
  
  \item \textbf{Pemrosesan payload}: Sistem kemudian menggabungkan pertanyaan dan dokumen yang telah diunggah sebagai konteks untuk dikirim ke model LLM (OpenAI GPT-4) melalui integrasi Langchain.
  
  \item \textbf{Penyimpanan riwayat chat}: Setelah mendapatkan hasil respons dari model AI, sistem menyimpan riwayat percakapan (chat) ke basis data untuk kepentingan histori pengguna.
  
  \item \textbf{Streaming respons ke pengguna}: Respons dari model dikirimkan secara bertahap dalam bentuk streaming teks ke pengguna, untuk pengalaman percakapan yang lebih cepat dan real-time.
\end{enumerate}

Diagram ini memvisualisasikan bagaimana sistem menangani proses otentikasi, pengelolaan dokumen, dan integrasi dengan layanan AI secara terstruktur dan efisien. Alur ini juga menunjukkan bahwa sistem bersifat stateless pada sisi server, karena menggunakan pendekatan API tanpa backend server monolitik.

\subsection{Entity Relationship Diagram (ERD)}

Entity Relationship Diagram (ERD) digunakan untuk menggambarkan struktur dan relasi antar tabel dalam basis data aplikasi chatbot berbasis \textit{Retrieval-Augmented Generation (RAG)}. Diagram ini membantu dalam merancang basis data yang efisien dan sesuai dengan kebutuhan fungsional aplikasi.

\begin{figure}[H]
  \centering
  \includegraphics[width=0.95\linewidth]{images/bab-3/erd-skripsi.png}
  \caption{Entity Relationship Diagram Aplikasi}
  \label{fig:erd}
\end{figure}

\noindent Penjelasan masing-masing entitas dalam ERD adalah sebagai berikut:

\begin{itemize}
  \item \textbf{User} \\
  Entitas \texttt{User} menyimpan data pengguna yang terautentikasi. Atribut yang dimiliki:
  \begin{itemize}
    \item \texttt{id}: UUID sebagai \textit{primary key}
    \item \texttt{email}: Alamat email pengguna
    \item \texttt{password}: Kata sandi pengguna (dalam bentuk terenkripsi)
  \end{itemize}
  Relasi: Satu pengguna dapat memiliki banyak \texttt{Chat}, \texttt{Document}, dan \texttt{Citations}.

  \item \textbf{Chat} \\
  Entitas ini merepresentasikan sesi percakapan antara pengguna dan sistem chatbot. Atribut:
  \begin{itemize}
    \item \texttt{id}: UUID sebagai \textit{primary key}
    \item \texttt{createdAt}: Timestamp waktu pembuatan chat
    \item \texttt{userId}: FK mengacu ke \texttt{User}
    \item \texttt{title}: Judul percakapan
    \item \texttt{visibility}: Status visibilitas percakapan (privat/publik)
  \end{itemize}
  Relasi: Satu \texttt{Chat} memiliki banyak \texttt{Message}.

  \item \textbf{Message} \\
  Menyimpan seluruh pesan yang terkandung dalam satu sesi chat. Atribut:
  \begin{itemize}
    \item \texttt{id}: UUID sebagai \textit{primary key}
    \item \texttt{chatId}: FK ke \texttt{Chat}
    \item \texttt{role}: Peran pengirim pesan (user/system)
    \item \texttt{parts}: Konten pesan dalam format JSON
    \item \texttt{attachments}: Lampiran terkait dalam format JSON
    \item \texttt{createdAt}: Timestamp pembuatan pesan
  \end{itemize}

  \item \textbf{Document} \\
  Entitas ini menyimpan data dokumen PDF yang diunggah oleh pengguna. Atribut:
  \begin{itemize}
    \item \texttt{id}: UUID sebagai \textit{primary key}
    \item \texttt{createdAt}: Waktu unggahan dokumen
    \item \texttt{title}: Judul dokumen
    \item \texttt{content}: Isi teks hasil ekstraksi PDF
    \item \texttt{userId}: FK ke \texttt{User}
    \item \texttt{text}: Versi teks tambahan atau metadata (opsional)
  \end{itemize}

  \item \textbf{Citations} \\
  Menyimpan kutipan atau sitasi yang dibuat pengguna berdasarkan dokumen yang telah diunggah. Atribut:
  \begin{itemize}
    \item \texttt{id}: UUID sebagai \textit{primary key}
    \item \texttt{userId}: FK ke \texttt{User}
    \item \texttt{doi}: Digital Object Identifier dari jurnal
    \item \texttt{style}: Gaya kutipan (APA, IEEE, Harvard, dll)
    \item \texttt{content}: Format sitasi lengkap
    \item \texttt{created\_at}: Waktu pembuatan sitasi
  \end{itemize}
\end{itemize}
\section{Implementasi Sistem}

Tahap implementasi merupakan proses penting dalam membangun aplikasi sesuai rancangan dan spesifikasi teknis. Sistem ini dikembangkan menggunakan pendekatan \textit{fullstack} dengan framework Next.js 15 (App Router) dan database PostgreSQL yang dihosting melalui layanan Neon. Tidak ada \textit{back-end} terpisah, seluruh \textit{server logic} ditulis dalam API Routes di dalam struktur Next.js. Implementasi juga melibatkan integrasi Langchain, Retrieval-Augmented Generation (RAG), OpenAI API, serta fitur anotasi PDF menggunakan PDF.js. Berikut adalah langkah-langkah implementasi yang dilakukan:

\subsection{Persiapan Database PostgreSQL (Neon)}

\begin{itemize}
  \item Membuat akun dan proyek baru di \url{https://neon.tech}, kemudian membuat \texttt{branch}, \texttt{database}, dan \texttt{user}.
  \item Menyimpan kredensial \texttt{host}, \texttt{database}, \texttt{user}, dan \texttt{password} ke dalam file \texttt{.env} proyek.
  \item Menyusun skema database berdasarkan ERD menggunakan perintah SQL secara manual atau melalui migration tool seperti Prisma.
  \item Tabel-tabel utama meliputi: \texttt{users}, \texttt{chats}, \texttt{messages}, \texttt{documents}, dan \texttt{citations}.
\end{itemize}

\subsection{Inisialisasi Proyek Next.js 15}

\begin{itemize}
  \item Menginisialisasi proyek dengan perintah:
  \begin{verbatim}
  npx create-next-app@latest ai-journal-assistant --typescript --app
  \end{verbatim}
  \item Mengaktifkan Tailwind CSS dan PostCSS untuk styling antarmuka.
  \item Menyiapkan folder \texttt{app/api/} untuk menyimpan seluruh logika server-side.
\end{itemize}

\subsection{Integrasi PostgreSQL dengan ORM}

\begin{itemize}
  \item Menggunakan Prisma sebagai ORM untuk menghubungkan aplikasi dengan database PostgreSQL.
  \item Menjalankan inisialisasi Prisma:
  \begin{verbatim}
  npx prisma init
  \end{verbatim}
  \item Mendefinisikan skema \texttt{schema.prisma} berdasarkan struktur tabel dan relasi.
  \item Melakukan migrasi skema ke database Neon:
  \begin{verbatim}
  npx prisma migrate dev
  \end{verbatim}
  \item Menggunakan \texttt{@prisma/client} pada API routes untuk melakukan kueri data.
\end{itemize}

\subsection{Integrasi Langchain dan RAG}

\begin{itemize}
  \item Menginstal dependensi utama:
  \begin{verbatim}
  npm install langchain @langchain/community @langchain/openai
  \end{verbatim}
  \item Menggunakan pipeline \texttt{RetrievalQAChain} dari Langchain untuk menghubungkan retriever dan model LLM (GPT-4).
  \item Langkah-langkah implementasi:
  \begin{itemize}
    \item Mengekstrak teks dari PDF.
    \item Memecah teks menjadi potongan pendek (\texttt{text splitting}).
    \item Membuat embeddings dengan model dari OpenAI (\texttt{text-embedding-ada-002}).
    \item Menyimpan embeddings ke dalam \texttt{vector store} (menggunakan Chroma).
    \item Melakukan similarity search ketika pengguna mengajukan pertanyaan.
    \item Menyediakan jawaban dari LLM yang memperhitungkan hasil retrieval.
  \end{itemize}
\end{itemize}

\subsection{Integrasi OpenAI API}

\begin{itemize}
  \item Mengatur API Key di \texttt{.env} sebagai \texttt{OPENAI\_API\_KEY}.
  \item Menggunakan endpoint \texttt{/api/chat} untuk menerima \texttt{prompt} dan merespons dengan hasil dari GPT-4.
  \item Mendukung konteks percakapan dengan menyimpan riwayat \texttt{messages} pada database.
\end{itemize}

\subsection{Penyimpanan Dokumen PDF dengan Vercel Blob}

\begin{itemize}
  \item Menggunakan \texttt{@vercel/blob} untuk menyimpan file PDF dari pengguna secara langsung ke storage Vercel.
  \item File PDF ini kemudian diakses kembali oleh sistem untuk proses ekstraksi isi dokumen.
  \item URL dokumen disimpan pada tabel \texttt{documents} dalam database.
\end{itemize}

\subsection{Manajemen State dan Cache dengan Redis (Upstash)}
\begin{itemize}
  \item Menggunakan Redis sebagai cache session dan penyimpanan embeddings sementara.
  \item Mendaftar ke layanan \url{https://upstash.com} dan menambahkan variabel \texttt{UPSTASH\_REDIS\_URL} dan \texttt{TOKEN} pada \texttt{.env}.
  \item Menggunakan package \texttt{@upstash/redis} untuk operasi Redis dalam API Routes.
\end{itemize}

\subsection{Preview dan Anotasi PDF dengan PDF.js}
\begin{itemize}
  \item Mengintegrasikan \texttt{pdfjs-dist} untuk menampilkan dokumen PDF secara langsung di browser.
  \item Fitur anotasi dikembangkan dengan menambahkan layer interaktif di atas canvas, yang mencatat teks yang dipilih dan menampilkan input komentar.
  \item Catatan dan posisi anotasi disimpan dalam tabel \texttt{annotations} atau bagian dari \texttt{documents}.
\end{itemize}
\section{Implementasi Fitur Utama}
\subsection{Upload dan Pratinjau PDF}
Pengguna dapat mengunggah file PDF yang akan disimpan pada \texttt{Vercel Blob}. File ini kemudian dimuat dan dipratinjau menggunakan \texttt{PDF.js}.
\begin{lstlisting}[language=TypeScript, caption={Chat dengan AI}]
'use client';

import { ChatHeader } from '@/components/chat-header';
import { useArtifactSelector } from '@/hooks/use-artifact';
import type { Vote } from '@/lib/db/schema';
import { fetcher, generateUUID } from '@/lib/utils';
import { useChat } from '@ai-sdk/react';
import type { Attachment, UIMessage } from 'ai';
import type { Session } from 'next-auth';
import { useRouter, useSearchParams } from 'next/navigation';
import { useEffect, useRef, useState } from 'react';
import useSWR, { useSWRConfig } from 'swr';
import { unstable_serialize } from 'swr/infinite';
import { Artifact } from './artifact';
import { Messages } from './messages';
import { MultimodalInput } from './multimodal-input';
import { getChatHistoryPaginationKey } from './sidebar-history';
import { toast } from './toast';
import type { VisibilityType } from './visibility-selector';

import dynamic from 'next/dynamic';
const PDFViewer = dynamic(() => import('../components/pdf-viewer'), {
  ssr: false,
});

import { Show } from './shared/show';
import { useBoolean } from '@/hooks/use-boolean';
import { useMediaQuery } from 'usehooks-ts';

export function Chat({
  id,
  initialMessages,
  selectedChatModel,
  selectedVisibilityType,
  isReadonly,
  session,
  attachmentUrl,
}: {
  id: string;
  initialMessages: Array<UIMessage>;
  selectedChatModel: string;
  selectedVisibilityType: VisibilityType;
  isReadonly: boolean;
  session: Session;
  attachmentUrl?: string;
}) {
  const { mutate } = useSWRConfig();
  const isPDFSubmitted = useRef(false);
  const router = useRouter();

  const {
    messages,
    setMessages,
    handleSubmit,
    input,
    setInput,
    append,
    status,
    stop,
    reload,
  } = useChat({
    id,
    initialMessages,
    experimental_throttle: 100,
    sendExtraMessageFields: true,
    generateId: generateUUID,
    experimental_prepareRequestBody: (body) => ({
      id,
      message: body.messages.at(-1),
      selectedChatModel,
    }),
    onFinish: () => {
      mutate(unstable_serialize(getChatHistoryPaginationKey));
    },
    onResponse: () => {
      if (isPDFSubmitted.current) return;
      isPDFSubmitted.current = true;
      router.refresh();
      return;
    },
    onError: (error) => {
      toast({
        type: 'error',
        description: error.message,
      });
    },
  });

  const searchParams = useSearchParams();
  const query = searchParams.get('query');

  const [hasAppendedQuery, setHasAppendedQuery] = useState(false);

  useEffect(() => {
    if (query && !hasAppendedQuery) {
      append({
        role: 'user',
        content: query,
      });

      setHasAppendedQuery(true);
      window.history.replaceState({}, '', `/chat/${id}`);
    }
  }, [query, append, hasAppendedQuery, id]);

  const { data: votes } = useSWR<Array<Vote>>(
    messages.length >= 2 ? `/api/vote?chatId=${id}` : null,
    fetcher,
  );

  const [attachments, setAttachments] = useState<Array<Attachment>>([]);
  const { value: isPdfVisible, toggle: togglePdfVisible } = useBoolean(true);
  const isArtifactVisible = useArtifactSelector((state) => state.isVisible);
  const isMobile = useMediaQuery('(max-width: 1024px)');

  return (
    <>
      <div
        className={`flex flex-col min-w-0 h-dvh bg-background ${
          attachmentUrl ? 'md:flex-row' : ''
        }`}
      >
        <div
          className={`flex flex-col flex-1 ${attachmentUrl ? 'md:w-1/2' : 'w-full'}`}
        >
          <ChatHeader
            chatId={id}
            selectedModelId={selectedChatModel}
            selectedVisibilityType={selectedVisibilityType}
            isReadonly={isReadonly}
            session={session}
            isPdfVisible={isPdfVisible}
            onPdfToggle={togglePdfVisible}
            showPdfToggle={Boolean(attachmentUrl)}
          />

          {/* mobile pdf viewer */}
          <Show
            when={
              Boolean(attachmentUrl) &&
              attachmentUrl !== '' &&
              isPdfVisible &&
              isMobile
            }
          >
            <PDFViewer chatId={id} url={attachmentUrl as string} />
          </Show>

          <Messages
            chatId={id}
            status={status}
            votes={votes}
            messages={messages}
            setMessages={setMessages}
            reload={reload}
            isReadonly={isReadonly}
            isArtifactVisible={isArtifactVisible}
          />

          <form className="flex mx-auto px-4 bg-background pb-4 md:pb-6 gap-2 w-full md:max-w-3xl">
            <Show when={!isReadonly}>
              <MultimodalInput
                chatId={id}
                input={input}
                setInput={setInput}
                handleSubmit={handleSubmit}
                status={status}
                stop={stop}
                attachments={attachments}
                setAttachments={setAttachments}
                messages={messages}
                setMessages={setMessages}
                append={append}
              />
            </Show>
          </form>
        </div>

        <Show
          when={
            Boolean(attachmentUrl) &&
            attachmentUrl !== '' &&
            isPdfVisible &&
            !isMobile
          }
        >
          <div className="flex-1 md:w-1/2 overflow-y-auto bg-gray-100">
            <PDFViewer chatId={id} url={attachmentUrl as string} />
          </div>
        </Show>
      </div>

      <Artifact
        chatId={id}
        input={input}
        setInput={setInput}
        handleSubmit={handleSubmit}
        status={status}
        stop={stop}
        attachments={attachments}
        setAttachments={setAttachments}
        append={append}
        messages={messages}
        setMessages={setMessages}
        reload={reload}
        votes={votes}
        isReadonly={isReadonly}
      />
    </>
  );
}
\end{lstlisting}

\subsection{Tanya Jawab Berbasis RAG}
Sistem menggunakan pipeline \texttt{Langchain} untuk menjalankan \textit{Retrieval-Augmented Generation}:
\begin{enumerate}
  \item Ekstraksi teks dari PDF
  \item Pemecahan menjadi chunk dengan overlap
  \item Embedding menggunakan model dari \texttt{OpenAI API}
  \item Penyimpanan dalam vektor menggunakan \texttt{ChromaDB}
  \item Retrieval saat tanya jawab berdasarkan \texttt{cosine similarity}
  \item Prompt otomatis digabung dengan hasil retrieval dan dikirim ke model GPT
\end{enumerate}

\subsection{Retrieval-Augmented Generation}
Agar LLM dapat bekerja dengan dokumen pribadi seperti PDF, konten dokumen harus diproses dan disimpan dengan cara yang dapat diakses secara efisien oleh LLM.\@ Seluruh proses ini dikenal sebagai \emph{ingestion} \citep[p~.84]{oshin2024learning}. Ide intinya adalah mengubah teks menjadi representasi numerik yang disebut \emph{embeddings} dan menyimpannya dalam penyimpanan \emph{vektor store} (sejenis basis data vektor). Hal ini memungkinkan aplikasi untuk menemukan dan mengambil bagian yang paling relevan dari dokumen untuk menjawab pertanyaan spesifik pengguna. Proses ini melibatkan empat langkah utama:
\begin{enumerate}
  \item \emph{Loading}: Mengekstrak teks dari dokumen PDF.
  \item \emph{Splitting}: Memecah teks yang diekstrak menjadi bagian yang lebih kecil dan mudah dikelola.
  \item \emph{Embedding}: Mengubah setiap potongan teks menjadi vektor numerik yang menangkap makna semantiknya.
  \item \emph{Storing}: Menyimpan \emph{embeddings} ini dalam penyimpanan vektor untuk pencarian yang efisien.
\end{enumerate}

\begin{figure}[htbp]
  \centering
  \includegraphics[width=0.85\linewidth]{images/bab-3/embeddings.png}
  \caption{Contoh \emph{RAG framework} yang menggunakan \emph{Vector Database}.}\label{fig:RAG-Framework}\citep{Jing}
\end{figure}

\subsection{Pembuatan Sitasi Otomatis}
Sistem melakukan ekstraksi metadata seperti judul, penulis, tahun, dan DOI dari PDF dan menyusunnya ke dalam format referensi (APA, MLA, IEEE, Harvard) secara otomatis.

\subsection{Anotasi PDF}
Menggunakan integrasi \texttt{PDF.js}, pengguna dapat melakukan:
\begin{itemize}
  \item Highlight teks
  \item Menambahkan catatan
  \item Menyimpan anotasi ke database untuk ditampilkan ulang
\end{itemize}
\section{Teknik Pengumpulan dan Analisis Data}
Data yang dikumpulkan mencakup interaksi pengguna dengan sistem, log respons chatbot, dan hasil uji coba fungsionalitas. Teknik pengumpulan data dilakukan melalui:
\begin{itemize}
  \item \textbf{Observasi langsung} saat pengguna menguji aplikasi.
  \item \textbf{Log sistem} untuk merekam performa model dan hasil pencarian RAG.
  \item \textbf{Kuesioner} untuk mengevaluasi aspek usability dari sisi pengguna.
\end{itemize}

\section{Pengujian dan Evaluasi}
Setiap iterasi diuji dengan pendekatan:
\begin{itemize}
  \item \textbf{Pengujian fungsional} untuk memastikan seluruh fitur seperti upload, tanya jawab, sitasi, dan anotasi bekerja sebagaimana mestinya.
  \item \textbf{Evaluasi usability} melalui uji coba langsung oleh pengguna sasaran (mahasiswa) untuk mendapatkan umpan balik terkait kemudahan dan efektivitas penggunaan sistem.
\end{itemize}
\section{Deployment dan Pengujian Aplikasi}

\begin{itemize}
  \item Aplikasi di-deploy ke \texttt{Vercel} dengan environment variables yang dikonfigurasi langsung di dashboard.
  \item Database Neon dan Redis Upstash beroperasi sebagai layanan terpisah dan dihubungkan melalui koneksi aman.
  \item Pengujian dilakukan melalui \texttt{functional testing} dan \texttt{usability testing} untuk memastikan semua fitur berjalan sesuai tujuan awal.
\end{itemize}

\noindent
Dengan mengikuti alur implementasi di atas, sistem chatbot dapat berjalan sesuai \emph{requirements} dan terintegrasi penuh antara penyimpanan dokumen, pencarian berbasis vektor, dan generasi jawaban kontekstual melalui LLM.
\chapter{HASIL DAN PEMBAHASAN}

\section{Hasil ...}
\chapter{PENUTUP}
\section{Kesimpulan}

Penelitian ini dilakukan untuk menjawab tantangan yang dihadapi mahasiswa, khususnya non-native English speakers, dalam memahami jurnal ilmiah berbahasa Inggris yang kompleks. Kesulitan dalam memahami struktur kalimat akademik, kosakata teknis, dan gaya penulisan formal menjadi hambatan yang signifikan dalam proses belajar dan riset. 
\singlespacing{}

Melalui pengembangan aplikasi chatbot berbasis \textit{Retrieval-Augmented Generation} (RAG) dan integrasi framework \textit{Langchain}, penelitian ini berhasil merancang sebuah solusi yang adaptif dan interaktif untuk membantu mahasiswa memahami konten jurnal, menyusun sitasi otomatis, dan mencatat informasi penting melalui fitur anotasi PDF. Penggunaan RAG memungkinkan sistem untuk memberikan jawaban yang lebih relevan dan akurat berdasarkan isi jurnal yang diunggah, sedangkan Langchain mendukung modularitas dan fleksibilitas sistem dalam mengelola alur interaksi pengguna dengan dokumen.
\singlespacing{}
Dengan demikian, aplikasi yang dikembangkan memiliki potensi untuk meningkatkan efisiensi dan efektivitas mahasiswa dalam melakukan literatur review, memahami isi jurnal secara kontekstual, serta mengelola catatan dan referensi akademik secara terintegrasi.

\section{Saran}

Berdasarkan hasil penelitian dan pengembangan yang telah dilakukan, terdapat beberapa saran yang dapat dipertimbangkan untuk pengembangan lebih lanjut:

\begin{enumerate}
  \item \textbf{Pengayaan fitur pencarian jurnal:} Aplikasi dapat dikembangkan lebih lanjut untuk terintegrasi dengan database jurnal akademik seperti Google Scholar, Semantic Scholar, atau PubMed, guna memudahkan mahasiswa dalam mencari dan mengunggah referensi langsung dari sumber terpercaya.

  \item \textbf{Pengembangan kemampuan multibahasa:} Menambahkan dukungan multibahasa, seperti penerjemahan otomatis dan dukungan terhadap jurnal berbahasa lain, akan membantu lebih banyak mahasiswa dari berbagai latar belakang bahasa.

  \item \textbf{Evaluasi performa teknis:} Di masa mendatang, aspek teknis seperti kecepatan respon chatbot, efisiensi pencarian vektor, serta beban sistem pada saat penggunaan tinggi perlu dievaluasi melalui pengujian performa untuk memastikan aplikasi tetap stabil dan responsif.

  \item \textbf{Fitur kolaborasi:} Pengembangan anotasi multi-user dan sistem komentar dapat mendukung kerja sama tim dalam riset atau diskusi kelompok, menjadikan aplikasi lebih dinamis dalam konteks akademik kolaboratif.

  \item \textbf{Peningkatan aspek keamanan dan privasi:} Mengingat aplikasi mengelola dokumen pribadi pengguna, maka penting untuk terus mengembangkan fitur keamanan seperti enkripsi dokumen, autentikasi ganda, dan penghapusan data otomatis untuk melindungi privasi pengguna.
\end{enumerate}

Dengan mengadopsi saran-saran di atas, diharapkan aplikasi ini dapat berkembang menjadi platform yang lebih komprehensif dalam mendukung proses riset mahasiswa dan memberikan kontribusi nyata terhadap literasi akademik digital.
\clearpage 
\phantomsection 
\renewcommand\bibname{DAFTAR PUSTAKA}\bibliographystyle{apalike}
\addcontentsline{toc}{chapter}{\bibname}\bibliography{refs}


\chapter*{LAMPIRAN}

\setcounter{page}{1}
\appendixpagenumbering
\thispagestyle{appendixstyle}
\pagestyle{appendixstyle}
\addcontentsline{toc}{chapter}{LAMPIRAN}

\section*{Lampiran 1 \emph{Dataset}}

\addcontentsline{app}{appendices}{Lampiran 1 \textit{Dataset}}

\subsection*{Ukuran \emph{Dataset}}
\begin{center}
\includegraphics[width=4.55cm]{images/prop_train} \includegraphics[width=4.55cm]{images/prop_valid}
\includegraphics[width=4.55cm]{images/prop_test}
\par\end{center}

\clearpage
\phantomsection

\section*{Lampiran 2 Jupyter Notebook}

\addcontentsline{app}{appendices}{Lampiran 2 Jupyter Notebook}

\subsection*{Arsitektur Model}
\begin{center}
\includegraphics[width=14cm]{images/model_1}
\par\end{center}

\subsection*{pi24\_train.ipynb}

\begin{lstlisting}[numbers=left,basicstyle={\tiny},breaklines=true,showstringspaces=false,tabsize=5]
import numpy as np 
import tensorflow as tf 
import os import keras 
from tensorflow.keras.preprocessing.image import ImageDataGenerator, img_to_array, load_img 
from tensorflow.keras.applications import EfficientNetV2B0 
from tensorflow.keras.layers import Dense 
from tensorflow.keras.models import Model 
from tensorflow.keras.optimizers import Adam 
import matplotlib.pyplot as plt
\end{lstlisting}

\clearpage
\phantomsection
\end{document}
