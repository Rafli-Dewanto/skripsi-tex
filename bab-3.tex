\chapter{METODE PENELITIAN}

\section{Spesifikasi Perangkat Keras}

Tabel 3.1 menunjukkan informasi lengkap seluruh perangkat keras yang
dipergunakan dalam penelitian ini.

\begin{table}[H]
\caption{Spesifikasi Perangkat Keras yang Digunakan}

\centering{}%
\begin{tabular}{|l|l|}
\hline 
\textbf{Komponen} & \textbf{Nama}\tabularnewline
\hline 
\hline 
CPU & AMD Ryzen 5 7600X\tabularnewline
\hline 
GPU & \tabularnewline
\hline 
RAM & \tabularnewline
\hline 
\emph{Chipset} & \tabularnewline
\hline 
SSD & \tabularnewline
\hline 
\end{tabular}
\end{table}


\section{Spesifikasi Perangkat Lunak}

Tabel 3.2 menunjukkan informasi lengkap mengenai seluruh perangkat
lunak yang dipergunakan dalam penelitian ini.

\begin{table}[H]
\caption{Spesifikasi Perangkat Lunak yang Digunakan}

\centering{}%
\begin{tabular}{|l|l|}
\hline 
\textbf{Aplikasi} & \textbf{Nama}\tabularnewline
\hline 
\hline 
OS & Windows 11 Pro 24H2\tabularnewline
\hline 
\emph{Virtual Environment} & \tabularnewline
\hline 
Bahasa Pemrograman & \tabularnewline
\hline 
Editor Teks & \tabularnewline
\hline 
\emph{Terminal Emulator} & \tabularnewline
\hline 
\emph{Framework} & \tabularnewline
\hline 
Penelusur Web & \tabularnewline
\hline 
\emph{Version Control} & \tabularnewline
\hline 
\end{tabular}
\end{table}