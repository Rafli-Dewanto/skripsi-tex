\chapter{METODE PENELITIAN}

\section{Spesifikasi Perangkat Keras}

Tabel 3.1 menunjukkan informasi lengkap seluruh perangkat keras yang
dipergunakan dalam penelitian ini.

\begin{table}[H]
  \caption{Spesifikasi Perangkat Keras yang Digunakan}

  \centering{}%
  \begin{tabular}{|l|l|}
    \hline
    \textbf{Komponen} & \textbf{Nama}\tabularnewline
    \hline
    \hline
    CPU               & AMD Ryzen 5 7600X\tabularnewline
    \hline
    GPU               & \tabularnewline
    \hline
    RAM               & \tabularnewline
    \hline
    \emph{Chipset}    & \tabularnewline
    \hline
    SSD               & \tabularnewline
    \hline
  \end{tabular}
\end{table}


\section{Spesifikasi Perangkat Lunak}

Tabel 3.2 menunjukkan informasi lengkap mengenai seluruh perangkat
lunak yang dipergunakan dalam penelitian ini.

\begin{table}[H]
  \caption{Spesifikasi Perangkat Lunak yang Digunakan}

  \centering{}%
  \begin{tabular}{|l|l|}
    \hline
    \textbf{Aplikasi}          & \textbf{Nama}\tabularnewline
    \hline
    \hline
    OS                         & Windows 11 Pro 24H2\tabularnewline
    \hline
    \emph{Virtual Environment} & \tabularnewline
    \hline
    Bahasa Pemrograman         & \tabularnewline
    \hline
    Editor Teks                & \tabularnewline
    \hline
    \emph{Terminal Emulator}   & \tabularnewline
    \hline
    \emph{Framework}           & \tabularnewline
    \hline
    Penelusur Web              & \tabularnewline
    \hline
    \emph{Version Control}     & \tabularnewline
    \hline
  \end{tabular}
\end{table}

\section{Metode Penelitian}

Metode penelitian ini dirancang untuk menghasilkan aplikasi chatbot berbasis dokumen jurnal ilmiah yang mampu memberikan jawaban relevan dan kontekstual kepada pengguna, serta menyediakan fitur pendukung seperti sitasi otomatis dan anotasi PDF.\@ Penelitian ini menggunakan pendekatan pengembangan perangkat lunak dengan model \textit{System Development Life Cycle (SDLC)} iteratif, karena pendekatan ini menggabungkan kejelasan alur dari model waterfall dengan fleksibilitas evaluatif dari model prototipe. Dalam setiap iterasi, dilakukan proses analisis kebutuhan, perancangan, implementasi, pengujian, dan evaluasi yang berulang hingga sistem mencapai bentuk final yang stabil dan sesuai kebutuhan pengguna.

\subsubsection*{Prosedur Penelitian}

Langkah-langkah utama dalam penelitian ini dijelaskan sebagai berikut:

\begin{enumerate}
  \item \textbf{Identifikasi Masalah dan Analisis Kebutuhan} \\
        Dilakukan studi literatur dan observasi terhadap hambatan yang dihadapi mahasiswa dalam memahami isi jurnal, membuat sitasi, serta mencatat hasil bacaan. Hasilnya dijadikan dasar dalam merumuskan kebutuhan fungsional dan non-fungsional aplikasi.

  \item \textbf{Perancangan Sistem dan Arsitektur RAG} \\
        Perancangan arsitektur chatbot berbasis \textit{Retrieval-Augmented Generation (RAG)} dilakukan dengan merancang pipeline alur data: unggah jurnal -> ekstraksi konten -> pembuatan vektor -> penyimpanan ke vector store -> pencarian konteks berdasarkan pertanyaan pengguna -> pemanggilan LLM untuk merespons. Peran Langchain dalam tahapan ini adalah sebagai framework untuk mengatur \textit{chainz} interaksi antara dokumen, prompt, dan model AI.

  \item \textbf{Implementasi Aplikasi} \\
        Proses pengembangan frontend dan backend dilakukan secara bertahap. Backend menangani proses parsing PDF, embedding konten ke dalam vektor, penyimpanan, dan retrieval. Frontend mencakup antarmuka chatbot, halaman unggah file, hasil chat, sitasi, dan fitur anotasi PDF.

  \item \textbf{Integrasi Model AI dengan Langchain} \\
        Langchain digunakan untuk menyusun *retrieval chain* dengan LLM. Ini memungkinkan sistem mencari informasi berbasis vektor dan memberikan hasil yang relevan ke model generatif sebagai konteks input. Hal ini mengatasi keterbatasan LLM dalam menjawab berdasarkan dokumen spesifik yang diunggah pengguna.

  \item \textbf{Pengujian dan Evaluasi Fungsionalitas} \\
        Pengujian dilakukan pada setiap iterasi untuk memastikan bahwa fitur utama berjalan dengan baik. Evaluasi mencakup respons chatbot terhadap pertanyaan berbasis jurnal, ketepatan sitasi, dan kenyamanan fitur anotasi PDF. Penilaian dilakukan secara subjektif melalui uji coba dengan pengguna terbatas (student testing).

  \item \textbf{Penyempurnaan dan Dokumentasi} \\
        Berdasarkan hasil evaluasi, dilakukan revisi pada tampilan dan fungsionalitas sistem, serta dokumentasi seluruh proses sebagai bagian dari pelaporan tugas akhir.
\end{enumerate}

\subsubsection*{Teknik Pengumpulan dan Analisis Data}

\begin{itemize}
  \item \textbf{Teknik Pengumpulan Data}:
        \begin{itemize}
          \item Studi literatur untuk merumuskan landasan teori dan permasalahan.
          \item Observasi langsung terhadap pengalaman pengguna dalam memahami jurnal.
          \item Uji coba aplikasi oleh mahasiswa untuk mengumpulkan umpan balik terhadap kegunaan dan fungsionalitas.
        \end{itemize}
  \item \textbf{Teknik Analisis Data}:
        \begin{itemize}
          \item Analisis deskriptif dari hasil pengujian fitur.
          \item Analisis fungsional terhadap ketepatan respons chatbot.
          \item Evaluasi usability berdasarkan kepuasan pengguna dan efisiensi interaksi.
        \end{itemize}
\end{itemize}

\subsection*{Model Pendekatan: RAG dan Langchain}

Metode \textbf{Retrieval-Augmented Generation (RAG)} digunakan sebagai pendekatan utama dalam sistem chatbot. RAG menggabungkan dua tahap utama: tahap \textit{retrieval} untuk mengambil informasi relevan dari basis data dokumen, dan tahap \textit{generation} untuk membentuk respons berdasarkan konteks yang ditemukan. Pendekatan ini dipilih karena LLM konvensional memiliki keterbatasan dalam memberikan jawaban berbasis dokumen tertentu tanpa akses eksplisit terhadap kontennya.

Framework \textbf{Langchain} digunakan untuk mengatur *pipeline* kerja chatbot mulai dari pengelolaan memori, integrasi vector store, hingga pembuatan prompt untuk model LLM. Langchain memungkinkan pembuatan sistem modular yang lebih mudah dirawat dan dikembangkan, serta menyediakan abstraksi yang membantu mengelola kompleksitas dalam integrasi AI.