\chapter{METODE PENELITIAN}

\section{Gambaran Umum Metodologi}

Penelitian ini menggunakan pendekatan \textit{System Development Life Cycle} (SDLC) dengan model \textit{iteratif}. Model ini dipilih karena sesuai untuk proyek pengembangan perangkat lunak yang membutuhkan fleksibilitas dan peningkatan bertahap seiring dengan evaluasi dan umpan balik pengguna. Proses iteratif memungkinkan pengembangan dilakukan dalam beberapa siklus (iterasi), di mana setiap iterasi terdiri dari tahapan analisis, desain, implementasi, dan pengujian. Hasil dari setiap iterasi akan digunakan untuk perbaikan dan penambahan fitur di iterasi selanjutnya.
\singlespacing{}
Gambaran umum tahapan dalam SDLC model iteratif yang diterapkan pada penelitian ini adalah sebagai berikut:

\begin{enumerate}
  \item \textbf{Analisis Kebutuhan:} Mengidentifikasi kebutuhan pengguna dan sistem berdasarkan studi awal serta permasalahan utama yang ingin diselesaikan.
  \item \textbf{Perancangan Sistem:} Mendesain arsitektur sistem, alur data, dan antarmuka pengguna berdasarkan kebutuhan yang telah diidentifikasi.
  \item \textbf{Implementasi:} Mengembangkan fitur-fitur sistem sesuai dengan desain menggunakan teknologi yang telah ditentukan.
  \item \textbf{Pengujian:} Melakukan uji fungsionalitas dan usability pada fitur yang telah dikembangkan untuk memastikan sistem berjalan sesuai harapan.
  \item \textbf{Iterasi:} Berdasarkan hasil pengujian dan umpan balik, dilakukan evaluasi dan perbaikan sistem, kemudian iterasi kembali ke tahap analisis.
\end{enumerate}

\noindent
Proses ini berulang hingga sistem mencapai bentuk akhir yang stabil dan memenuhi seluruh kebutuhan pengguna.
\section{Analisis Kebutuhan}

Tahap awal ini dilakukan untuk mengidentifikasi kebutuhan pengguna dan sistem. Fokus utama adalah permasalahan yang dihadapi mahasiswa dalam memahami jurnal ilmiah, serta fitur yang diperlukan seperti tanya jawab berbasis dokumen, sitasi otomatis, dan anotasi PDF. Dari tahap ini diturunkan kebutuhan fungsional dan non-fungsional sebagai dasar pengembangan sistem.

\subsection{Spesifikasi Perangkat Keras}

Tabel 3.1 menunjukkan informasi lengkap seluruh perangkat keras yang
dipergunakan dalam penelitian ini.

\begin{table}[H]
  \caption{Spesifikasi Perangkat Keras yang Digunakan}

  \centering{}%
  \begin{tabular}{|l|l|}
    \hline
    \textbf{Komponen} & \textbf{Nama}\tabularnewline
    \hline
    \hline
    CPU               & Apple Chip M1\tabularnewline
    \hline
    GPU               & Apple Chip M1\tabularnewline
    \hline
    RAM               & 8GB\tabularnewline
    \hline
    \emph{Chipset}    & M1\tabularnewline
    \hline
    SSD               & 256GB\tabularnewline
    \hline
  \end{tabular}
\end{table}


\subsection{Spesifikasi Perangkat Lunak}

Tabel 3.2 menunjukkan informasi lengkap mengenai seluruh perangkat
lunak yang dipergunakan dalam penelitian ini.

\begin{table}[H]
  \caption{Spesifikasi Perangkat Lunak yang Digunakan}

  \centering{}%
  \begin{tabular}{|l|l|}
    \hline
    \textbf{Aplikasi}          & \textbf{Nama}\tabularnewline
    \hline
    \hline
    OS                         & Macintosh\tabularnewline
    \hline
    \emph{Virtual Environment} & \emph{Node Package Mananger}\tabularnewline
    \hline
    Bahasa Pemrograman         & \emph{TypeScript}\tabularnewline
    \hline
    Editor Teks                & \emph{VS Code}\tabularnewline
    \hline
    \emph{Terminal Emulator}   & \emph{Ghostty}\tabularnewline
    \hline
    \emph{Framework}           & \emph{Next.js}\tabularnewline
    \hline
    Penelusur Web              & \emph{Arc}\tabularnewline
    \hline
    \emph{Version Control}     & \emph{Git}\tabularnewline
    \hline
  \end{tabular}
\end{table}

\section{Perancangan Sistem}
Pada tahap desain, dilakukan:
\begin{itemize}
  \item Perancangan arsitektur sistem yang mengintegrasikan \textit{front-end}, \textit{back-end}, dan layanan AI eksternal.
  \item Perancangan alur proses RAG menggunakan \textit{Langchain}, yang menghubungkan retriever, dokumentasi, dan model LLM.
  \item Desain tampilan antarmuka pengguna (\textit{user interface}) untuk memudahkan mahasiswa dalam mengunggah artikel ilmiah, melakukan interaksi tanya-jawab, membuat sitasi, dan memberi anotasi pada PDF.
  \item Perancangan struktur database menggunakan \emph{PostgreSQL} dan \emph{Redis} untuk menyimpan metadata, histori interaksi, dan catatan pengguna.
\end{itemize}

\subsection{Arsitektur Sistem}
Sistem dirancang dalam arsitektur berbasis layanan terpisah (\textit{modular}) untuk memastikan fleksibilitas dan skalabilitas. Gambar~\ref{fig:arsitektur-sistem} menyajikan arsitektur umum sistem yang dikembangkan.

\begin{figure}[H]
  \centering
  \includegraphics[width=0.9\textwidth]{images/system-arch.png}
  \caption{Arsitektur sistem chatbot dengan RAG dan Langchain}
  \label{fig:arsitektur-sistem}
\end{figure}

\subsection{Use Case Diagram}
Bagian ini menjelaskan fungsionalitas sistem Chatbot Journal dari perspektif pengguna, yang digambarkan melalui use case diagram. Use case diagram ini mengidentifikasi aktor yang berinteraksi dengan sistem dan berbagai fungsi atau layanan yang disediakan oleh sistem.

\begin{figure}[H]
  \centering
  \includegraphics[width=0.9\textwidth]{images/bab-3/usecase.jpg}
  \caption{Use Case Diagram Aplikasi Chatbot Akademik}
  \label{fig:usecase}
\end{figure}
\subsection{Aktor}
Dalam use case diagram ini, terdapat satu aktor utama yaitu \textbf{Pengguna}. Aktor ini merepresentasikan individu yang akan berinteraksi langsung dengan sistem \textit{Journal Chatbot} untuk memenuhi berbagai kebutuhannya terkait manajemen jurnal dan interaksi dengan chatbot.

\subsection{Use Case}
Berikut adalah penjelasan rinci mengenai setiap use case yang terlibat dalam sistem \textit{Journal Chatbot}:

\begin{enumerate}
    \item \textbf{Login}
    \begin{itemize}
        \item \textbf{Deskripsi:} Use case ini memungkinkan pengguna untuk masuk ke dalam sistem dengan menggunakan kredensial yang valid. Ini adalah use case dasar yang harus dilakukan sebelum pengguna dapat mengakses fitur-fitur lain yang memerlukan otentikasi.
        \item \textbf{Relasi:} Use case \textit{Login} di-\textit{include} oleh \textit{Chat dengan Chatbot} dan \textit{Melihat History Chat}, yang berarti setiap kali pengguna ingin melakukan chat atau melihat riwayat chat, mereka harus terlebih dahulu berhasil login.
    \end{itemize}

    \item \textbf{Membuat Sitasi}
    \begin{itemize}
        \item \textbf{Deskripsi:} Use case ini memungkinkan pengguna untuk menghasilkan sitasi dari sumber jurnal atau dokumen yang telah diunggah.
        \item \textbf{Relasi:}
        \begin{itemize}
            \item \textit{Extend Salin Sitasi}: Pengguna dapat memilih untuk menyalin sitasi yang telah dibuat ke clipboard untuk penggunaan lebih lanjut.
            \item \textit{Extend Export Sitasi}: Pengguna dapat memilih untuk mengekspor sitasi ke format lain (misalnya, Bib\TeX, RIS, atau plain text) untuk pengelolaan referensi.
        \end{itemize}
    \end{itemize}

    \item \textbf{Chat dengan Chatbot}
    \begin{itemize}
        \item \textbf{Deskripsi:} Use case ini merupakan inti dari sistem, di mana pengguna dapat berinteraksi secara langsung dengan chatbot untuk mendapatkan informasi, menjawab pertanyaan, atau melakukan tugas-tugas terkait jurnal.
        \item \textbf{Relasi:}
        \begin{itemize}
            \item \textit{Include Login}: Pengguna harus login sebelum dapat berinteraksi dengan chatbot.
            \item \textit{Extend Upload PDF}: Saat berinteraksi dengan chatbot, pengguna memiliki opsi untuk mengunggah dokumen PDF yang kemudian dapat dianalisis atau digunakan oleh chatbot dalam percakapan.
            \item \textit{Extend Anotasi PDF}: Setelah mengunggah PDF, pengguna dapat melakukan anotasi pada dokumen tersebut, seperti menyorot teks, menambahkan catatan, atau menandai bagian-bagian penting. Fitur ini memperkaya interaksi dengan chatbot karena chatbot dapat merujuk pada anotasi pengguna.
        \end{itemize}
    \end{itemize}

    \item \textbf{Melihat History Chat}
    \begin{itemize}
        \item \textbf{Deskripsi:} Use case ini memungkinkan pengguna untuk melihat dan mengelola riwayat percakapan mereka dengan chatbot. Ini sangat penting untuk melacak interaksi sebelumnya dan melanjutkan percakapan.
        \item \textbf{Relasi:}
        \begin{itemize}
            \item \textit{Include Login}: Pengguna harus login untuk dapat mengakses riwayat chat mereka.
            \item \textit{Extend Share Chat}: Pengguna dapat memilih untuk membagikan riwayat chat tertentu dengan pihak lain (misalnya, melalui email atau aplikasi pesan).
            \item \textit{Extend Edit Nama Chat}: Pengguna dapat mengubah nama atau judul dari riwayat chat tertentu untuk mempermudah identifikasi dan pengelolaan.
            \item \textit{Extend Hapus Chat}: Pengguna dapat menghapus riwayat chat yang tidak lagi dibutuhkan untuk menjaga kebersihan data.
        \end{itemize}
    \end{itemize}
\end{enumerate}

\subsection{Relasi Antar Use Case}
\begin{itemize}
    \item \textbf{Relasi \textit{include}}: Menunjukkan bahwa suatu use case menyertakan fungsionalitas dari use case lain secara wajib. Dalam diagram ini, \textit{Login} di-\textit{include} oleh \textit{Chat dengan Chatbot} dan \textit{Melihat History Chat}, yang menekankan bahwa otentikasi adalah prasyarat untuk kedua fungsi tersebut.
    
    \item \textbf{Relasi \textit{extend}}: Menunjukkan bahwa suatu use case dapat memperluas fungsionalitas use case lain dalam kondisi tertentu. Contohnya, \textit{Salin Sitasi} dan \textit{Export Sitasi} adalah opsi tambahan yang tersedia setelah \textit{Membuat Sitasi}. Demikian pula, \textit{Upload PDF} dan \textit{Anotasi PDF} memperluas fungsionalitas \textit{Chat dengan Chatbot}, sementara \textit{Share Chat}, \textit{Edit Nama Chat}, dan \textit{Hapus Chat} memperluas \textit{Melihat History Chat}. Relasi ini menunjukkan fleksibilitas sistem dalam menawarkan fitur tambahan sesuai kebutuhan pengguna.
\end{itemize}

\subsection{Activity Diagram}

Activity Diagram adalah representasi grafis dari alur kerja atau kegiatan di dalam suatu sistem, yang memperlihatkan berbagai langkah yang diperlukan untuk membangun aplikasi chatbot jurnal ini. Berikut merupakan penjelasan lengkap tentang setiap fungsi yang dijalankan sistem ini.

\begin{figure}[H]
  \centering
  \includegraphics[width=0.9\textwidth]{images/bab-3/sitemap.jpg}
  \caption{Activity Diagram Interaksi Chatbot dengan RAG}
  \label{fig:activity}
\end{figure}

Activity diagram di atas menggambarkan alur proses interaksi pengguna dengan aplikasi. Proses dimulai ketika pengguna membuka web app. Sistem kemudian memeriksa apakah pengguna sudah memiliki akun. Jika belum, sistem akan menampilkan halaman sign up; jika sudah, sistem menampilkan halaman login. Pengguna kemudian mengisi kredensial yang dikirimkan ke sistem untuk proses otentikasi melalui database. Jika otentikasi berhasil, sistem akan menampilkan halaman utama. Jika gagal, sistem akan menampilkan pesan kesalahan.
\singlespacing{}
Setelah berhasil masuk, pengguna dapat memilih apakah ingin mengunggah file PDF atau tidak. Jika tidak, pengguna hanya perlu mengirimkan chat. Namun, jika ingin mengunggah PDF, pengguna akan mengirimkan chat bersamaan dengan file tersebut. File PDF akan diunggah ke object storage, kemudian payload yang berisi chat dan file akan diproses menggunakan integrasi OpenAI.\@ Setelah diproses, hasil response dikirimkan kembali ke sistem dan sistem mengirimkannya ke pengguna dalam bentuk stream teks. Di sisi lain, chat juga disimpan ke dalam database sebagai arsip interaksi.

\subsection{Sequence Diagram}

Gambar~\ref{fig:sequence-login-chat} menunjukkan sequence diagram yang menggambarkan alur interaksi antara pengguna, sistem, dan basis data saat menggunakan aplikasi chatbot jurnal. Diagram ini menjelaskan urutan proses dari mulai membuka aplikasi hingga pengguna menerima respons dari sistem berdasarkan dokumen yang telah diunggah.

\begin{figure}[H]
    \centering
    \includegraphics[width=0.9\textwidth]{images/bab-3/sequence.png}
    \caption{Sequence Diagram Interaksi User dan Sistem}
    \label{fig:sequence-login-chat}
\end{figure}

Berdasarkan Gambar~\ref{fig:sequence-login-chat}, berikut adalah tahapan proses yang terjadi:

\begin{enumerate}
  \item \textbf{Pengguna membuka aplikasi web}: Proses dimulai ketika pengguna mengakses aplikasi melalui browser.
  
  \item \textbf{Cek akun dan otentikasi}: Sistem memeriksa apakah pengguna sudah memiliki akun. Jika ya, pengguna mengisi kredensial (email dan password) lalu sistem mengirimkan kredensial tersebut ke backend untuk proses otentikasi.
  
  \item \textbf{Respons otentikasi}: Jika kredensial valid, sistem akan menampilkan halaman utama. Jika tidak, sistem menampilkan pesan error dan meminta pengguna untuk mencoba kembali.
  
  \item \textbf{Registrasi pengguna baru}: Jika pengguna belum memiliki akun, sistem mengarahkan ke halaman sign-up.
  
  \item \textbf{Pengiriman pesan dan dokumen}: Setelah berhasil masuk, pengguna dapat melakukan percakapan dengan chatbot. Jika dibutuhkan, pengguna juga dapat mengunggah file PDF berisi jurnal ilmiah.
  
  \item \textbf{Upload PDF ke Object Storage}: Jika terdapat PDF yang dikirimkan, sistem akan mengunggah file tersebut ke layanan object storage (misalnya Vercel Blob atau storage berbasis cloud lainnya).
  
  \item \textbf{Pemrosesan payload}: Sistem kemudian menggabungkan pertanyaan dan dokumen yang telah diunggah sebagai konteks untuk dikirim ke model LLM (OpenAI GPT-4) melalui integrasi Langchain.
  
  \item \textbf{Penyimpanan riwayat chat}: Setelah mendapatkan hasil respons dari model AI, sistem menyimpan riwayat percakapan (chat) ke basis data untuk kepentingan histori pengguna.
  
  \item \textbf{Streaming respons ke pengguna}: Respons dari model dikirimkan secara bertahap dalam bentuk streaming teks ke pengguna, untuk pengalaman percakapan yang lebih cepat dan real-time.
\end{enumerate}

Diagram ini memvisualisasikan bagaimana sistem menangani proses otentikasi, pengelolaan dokumen, dan integrasi dengan layanan AI secara terstruktur dan efisien. Alur ini juga menunjukkan bahwa sistem bersifat stateless pada sisi server, karena menggunakan pendekatan API tanpa backend server monolitik.

\subsection{Entity Relationship Diagram (ERD)}

Entity Relationship Diagram (ERD) digunakan untuk menggambarkan struktur dan relasi antar tabel dalam basis data aplikasi chatbot berbasis \textit{Retrieval-Augmented Generation (RAG)}. Diagram ini membantu dalam merancang basis data yang efisien dan sesuai dengan kebutuhan fungsional aplikasi.

\begin{figure}[H]
  \centering
  \includegraphics[width=0.95\linewidth]{images/bab-3/erd-skripsi.png}
  \caption{Entity Relationship Diagram Aplikasi}
  \label{fig:erd}
\end{figure}

\noindent Penjelasan masing-masing entitas dalam ERD adalah sebagai berikut:

\begin{itemize}
  \item \textbf{User} \\
  Entitas \texttt{User} menyimpan data pengguna yang terautentikasi. Atribut yang dimiliki:
  \begin{itemize}
    \item \texttt{id}: UUID sebagai \textit{primary key}
    \item \texttt{email}: Alamat email pengguna
    \item \texttt{password}: Kata sandi pengguna (dalam bentuk terenkripsi)
  \end{itemize}
  Relasi: Satu pengguna dapat memiliki banyak \texttt{Chat}, \texttt{Document}, dan \texttt{Citations}.

  \item \textbf{Chat} \\
  Entitas ini merepresentasikan sesi percakapan antara pengguna dan sistem chatbot. Atribut:
  \begin{itemize}
    \item \texttt{id}: UUID sebagai \textit{primary key}
    \item \texttt{createdAt}: Timestamp waktu pembuatan chat
    \item \texttt{userId}: FK mengacu ke \texttt{User}
    \item \texttt{title}: Judul percakapan
    \item \texttt{visibility}: Status visibilitas percakapan (privat/publik)
  \end{itemize}
  Relasi: Satu \texttt{Chat} memiliki banyak \texttt{Message}.

  \item \textbf{Message} \\
  Menyimpan seluruh pesan yang terkandung dalam satu sesi chat. Atribut:
  \begin{itemize}
    \item \texttt{id}: UUID sebagai \textit{primary key}
    \item \texttt{chatId}: FK ke \texttt{Chat}
    \item \texttt{role}: Peran pengirim pesan (user/system)
    \item \texttt{parts}: Konten pesan dalam format JSON
    \item \texttt{attachments}: Lampiran terkait dalam format JSON
    \item \texttt{createdAt}: Timestamp pembuatan pesan
  \end{itemize}

  \item \textbf{Document} \\
  Entitas ini menyimpan data dokumen PDF yang diunggah oleh pengguna. Atribut:
  \begin{itemize}
    \item \texttt{id}: UUID sebagai \textit{primary key}
    \item \texttt{createdAt}: Waktu unggahan dokumen
    \item \texttt{title}: Judul dokumen
    \item \texttt{content}: Isi teks hasil ekstraksi PDF
    \item \texttt{userId}: FK ke \texttt{User}
    \item \texttt{text}: Versi teks tambahan atau metadata (opsional)
  \end{itemize}

  \item \textbf{Citations} \\
  Menyimpan kutipan atau sitasi yang dibuat pengguna berdasarkan dokumen yang telah diunggah. Atribut:
  \begin{itemize}
    \item \texttt{id}: UUID sebagai \textit{primary key}
    \item \texttt{userId}: FK ke \texttt{User}
    \item \texttt{doi}: Digital Object Identifier dari jurnal
    \item \texttt{style}: Gaya kutipan (APA, IEEE, Harvard, dll)
    \item \texttt{content}: Format sitasi lengkap
    \item \texttt{created\_at}: Waktu pembuatan sitasi
  \end{itemize}
\end{itemize}
\section{Implementasi Sistem}

Tahap implementasi merupakan proses penting dalam membangun aplikasi sesuai rancangan dan spesifikasi teknis. Sistem ini dikembangkan menggunakan pendekatan \textit{fullstack} dengan framework Next.js 15 (App Router) dan database PostgreSQL yang dihosting melalui layanan Neon. Tidak ada \textit{back-end} terpisah, seluruh \textit{server logic} ditulis dalam API Routes di dalam struktur Next.js. Implementasi juga melibatkan integrasi Langchain, Retrieval-Augmented Generation (RAG), OpenAI API, serta fitur anotasi PDF menggunakan PDF.js. Berikut adalah langkah-langkah implementasi yang dilakukan:

\subsection{Persiapan Database PostgreSQL (Neon)}

\begin{itemize}
  \item Membuat akun dan proyek baru di \url{https://neon.tech}, kemudian membuat \texttt{branch}, \texttt{database}, dan \texttt{user}.
  \item Menyimpan kredensial \texttt{host}, \texttt{database}, \texttt{user}, dan \texttt{password} ke dalam file \texttt{.env} proyek.
  \item Menyusun skema database berdasarkan ERD menggunakan perintah SQL secara manual atau melalui migration tool seperti Prisma.
  \item Tabel-tabel utama meliputi: \texttt{users}, \texttt{chats}, \texttt{messages}, \texttt{documents}, dan \texttt{citations}.
\end{itemize}

\subsection{Inisialisasi Proyek Next.js 15}

\begin{itemize}
  \item Menginisialisasi proyek dengan perintah:
  \begin{verbatim}
  npx create-next-app@latest ai-journal-assistant --typescript --app
  \end{verbatim}
  \item Mengaktifkan Tailwind CSS dan PostCSS untuk styling antarmuka.
  \item Menyiapkan folder \texttt{app/api/} untuk menyimpan seluruh logika server-side.
\end{itemize}

\subsection{Integrasi PostgreSQL dengan ORM}

\begin{itemize}
  \item Menggunakan Prisma sebagai ORM untuk menghubungkan aplikasi dengan database PostgreSQL.
  \item Menjalankan inisialisasi Prisma:
  \begin{verbatim}
  npx prisma init
  \end{verbatim}
  \item Mendefinisikan skema \texttt{schema.prisma} berdasarkan struktur tabel dan relasi.
  \item Melakukan migrasi skema ke database Neon:
  \begin{verbatim}
  npx prisma migrate dev
  \end{verbatim}
  \item Menggunakan \texttt{@prisma/client} pada API routes untuk melakukan kueri data.
\end{itemize}

\subsection{Integrasi Langchain dan RAG}

\begin{itemize}
  \item Menginstal dependensi utama:
  \begin{verbatim}
  npm install langchain @langchain/community @langchain/openai
  \end{verbatim}
  \item Menggunakan pipeline \texttt{RetrievalQAChain} dari Langchain untuk menghubungkan retriever dan model LLM (GPT-4).
  \item Langkah-langkah implementasi:
  \begin{itemize}
    \item Mengekstrak teks dari PDF.
    \item Memecah teks menjadi potongan pendek (\texttt{text splitting}).
    \item Membuat embeddings dengan model dari OpenAI (\texttt{text-embedding-ada-002}).
    \item Menyimpan embeddings ke dalam \texttt{vector store} (menggunakan Chroma).
    \item Melakukan similarity search ketika pengguna mengajukan pertanyaan.
    \item Menyediakan jawaban dari LLM yang memperhitungkan hasil retrieval.
  \end{itemize}
\end{itemize}

\subsection{Integrasi OpenAI API}

\begin{itemize}
  \item Mengatur API Key di \texttt{.env} sebagai \texttt{OPENAI\_API\_KEY}.
  \item Menggunakan endpoint \texttt{/api/chat} untuk menerima \texttt{prompt} dan merespons dengan hasil dari GPT-4.
  \item Mendukung konteks percakapan dengan menyimpan riwayat \texttt{messages} pada database.
\end{itemize}

\subsection{Penyimpanan Dokumen PDF dengan Vercel Blob}

\begin{itemize}
  \item Menggunakan \texttt{@vercel/blob} untuk menyimpan file PDF dari pengguna secara langsung ke storage Vercel.
  \item File PDF ini kemudian diakses kembali oleh sistem untuk proses ekstraksi isi dokumen.
  \item URL dokumen disimpan pada tabel \texttt{documents} dalam database.
\end{itemize}

\subsection{Manajemen State dan Cache dengan Redis (Upstash)}
\begin{itemize}
  \item Menggunakan Redis sebagai cache session dan penyimpanan embeddings sementara.
  \item Mendaftar ke layanan \url{https://upstash.com} dan menambahkan variabel \texttt{UPSTASH\_REDIS\_URL} dan \texttt{TOKEN} pada \texttt{.env}.
  \item Menggunakan package \texttt{@upstash/redis} untuk operasi Redis dalam API Routes.
\end{itemize}

\subsection{Preview dan Anotasi PDF dengan PDF.js}
\begin{itemize}
  \item Mengintegrasikan \texttt{pdfjs-dist} untuk menampilkan dokumen PDF secara langsung di browser.
  \item Fitur anotasi dikembangkan dengan menambahkan layer interaktif di atas canvas, yang mencatat teks yang dipilih dan menampilkan input komentar.
  \item Catatan dan posisi anotasi disimpan dalam tabel \texttt{annotations} atau bagian dari \texttt{documents}.
\end{itemize}
\section{Implementasi Fitur Utama}
\subsection{Upload dan Pratinjau PDF}
Pengguna dapat mengunggah file PDF yang akan disimpan pada \texttt{Vercel Blob}. File ini kemudian dimuat dan dipratinjau menggunakan \texttt{PDF.js}.
\begin{lstlisting}[language=TypeScript, caption={Chat dengan AI}]
'use client';

import { ChatHeader } from '@/components/chat-header';
import { useArtifactSelector } from '@/hooks/use-artifact';
import type { Vote } from '@/lib/db/schema';
import { fetcher, generateUUID } from '@/lib/utils';
import { useChat } from '@ai-sdk/react';
import type { Attachment, UIMessage } from 'ai';
import type { Session } from 'next-auth';
import { useRouter, useSearchParams } from 'next/navigation';
import { useEffect, useRef, useState } from 'react';
import useSWR, { useSWRConfig } from 'swr';
import { unstable_serialize } from 'swr/infinite';
import { Artifact } from './artifact';
import { Messages } from './messages';
import { MultimodalInput } from './multimodal-input';
import { getChatHistoryPaginationKey } from './sidebar-history';
import { toast } from './toast';
import type { VisibilityType } from './visibility-selector';

import dynamic from 'next/dynamic';
const PDFViewer = dynamic(() => import('../components/pdf-viewer'), {
  ssr: false,
});

import { Show } from './shared/show';
import { useBoolean } from '@/hooks/use-boolean';
import { useMediaQuery } from 'usehooks-ts';

export function Chat({
  id,
  initialMessages,
  selectedChatModel,
  selectedVisibilityType,
  isReadonly,
  session,
  attachmentUrl,
}: {
  id: string;
  initialMessages: Array<UIMessage>;
  selectedChatModel: string;
  selectedVisibilityType: VisibilityType;
  isReadonly: boolean;
  session: Session;
  attachmentUrl?: string;
}) {
  const { mutate } = useSWRConfig();
  const isPDFSubmitted = useRef(false);
  const router = useRouter();

  const {
    messages,
    setMessages,
    handleSubmit,
    input,
    setInput,
    append,
    status,
    stop,
    reload,
  } = useChat({
    id,
    initialMessages,
    experimental_throttle: 100,
    sendExtraMessageFields: true,
    generateId: generateUUID,
    experimental_prepareRequestBody: (body) => ({
      id,
      message: body.messages.at(-1),
      selectedChatModel,
    }),
    onFinish: () => {
      mutate(unstable_serialize(getChatHistoryPaginationKey));
    },
    onResponse: () => {
      if (isPDFSubmitted.current) return;
      isPDFSubmitted.current = true;
      router.refresh();
      return;
    },
    onError: (error) => {
      toast({
        type: 'error',
        description: error.message,
      });
    },
  });

  const searchParams = useSearchParams();
  const query = searchParams.get('query');

  const [hasAppendedQuery, setHasAppendedQuery] = useState(false);

  useEffect(() => {
    if (query && !hasAppendedQuery) {
      append({
        role: 'user',
        content: query,
      });

      setHasAppendedQuery(true);
      window.history.replaceState({}, '', `/chat/${id}`);
    }
  }, [query, append, hasAppendedQuery, id]);

  const { data: votes } = useSWR<Array<Vote>>(
    messages.length >= 2 ? `/api/vote?chatId=${id}` : null,
    fetcher,
  );

  const [attachments, setAttachments] = useState<Array<Attachment>>([]);
  const { value: isPdfVisible, toggle: togglePdfVisible } = useBoolean(true);
  const isArtifactVisible = useArtifactSelector((state) => state.isVisible);
  const isMobile = useMediaQuery('(max-width: 1024px)');

  return (
    <>
      <div
        className={`flex flex-col min-w-0 h-dvh bg-background ${
          attachmentUrl ? 'md:flex-row' : ''
        }`}
      >
        <div
          className={`flex flex-col flex-1 ${attachmentUrl ? 'md:w-1/2' : 'w-full'}`}
        >
          <ChatHeader
            chatId={id}
            selectedModelId={selectedChatModel}
            selectedVisibilityType={selectedVisibilityType}
            isReadonly={isReadonly}
            session={session}
            isPdfVisible={isPdfVisible}
            onPdfToggle={togglePdfVisible}
            showPdfToggle={Boolean(attachmentUrl)}
          />

          {/* mobile pdf viewer */}
          <Show
            when={
              Boolean(attachmentUrl) &&
              attachmentUrl !== '' &&
              isPdfVisible &&
              isMobile
            }
          >
            <PDFViewer chatId={id} url={attachmentUrl as string} />
          </Show>

          <Messages
            chatId={id}
            status={status}
            votes={votes}
            messages={messages}
            setMessages={setMessages}
            reload={reload}
            isReadonly={isReadonly}
            isArtifactVisible={isArtifactVisible}
          />

          <form className="flex mx-auto px-4 bg-background pb-4 md:pb-6 gap-2 w-full md:max-w-3xl">
            <Show when={!isReadonly}>
              <MultimodalInput
                chatId={id}
                input={input}
                setInput={setInput}
                handleSubmit={handleSubmit}
                status={status}
                stop={stop}
                attachments={attachments}
                setAttachments={setAttachments}
                messages={messages}
                setMessages={setMessages}
                append={append}
              />
            </Show>
          </form>
        </div>

        <Show
          when={
            Boolean(attachmentUrl) &&
            attachmentUrl !== '' &&
            isPdfVisible &&
            !isMobile
          }
        >
          <div className="flex-1 md:w-1/2 overflow-y-auto bg-gray-100">
            <PDFViewer chatId={id} url={attachmentUrl as string} />
          </div>
        </Show>
      </div>

      <Artifact
        chatId={id}
        input={input}
        setInput={setInput}
        handleSubmit={handleSubmit}
        status={status}
        stop={stop}
        attachments={attachments}
        setAttachments={setAttachments}
        append={append}
        messages={messages}
        setMessages={setMessages}
        reload={reload}
        votes={votes}
        isReadonly={isReadonly}
      />
    </>
  );
}
\end{lstlisting}

\subsection{Tanya Jawab Berbasis RAG}
Sistem menggunakan pipeline \texttt{Langchain} untuk menjalankan \textit{Retrieval-Augmented Generation}:
\begin{enumerate}
  \item Ekstraksi teks dari PDF
  \item Pemecahan menjadi chunk dengan overlap
  \item Embedding menggunakan model dari \texttt{OpenAI API}
  \item Penyimpanan dalam vektor menggunakan \texttt{ChromaDB}
  \item Retrieval saat tanya jawab berdasarkan \texttt{cosine similarity}
  \item Prompt otomatis digabung dengan hasil retrieval dan dikirim ke model GPT
\end{enumerate}

\subsection{Retrieval-Augmented Generation}
Agar LLM dapat bekerja dengan dokumen pribadi seperti PDF, konten dokumen harus diproses dan disimpan dengan cara yang dapat diakses secara efisien oleh LLM.\@ Seluruh proses ini dikenal sebagai \emph{ingestion} \citep[p~.84]{oshin2024learning}. Ide intinya adalah mengubah teks menjadi representasi numerik yang disebut \emph{embeddings} dan menyimpannya dalam penyimpanan \emph{vektor store} (sejenis basis data vektor). Hal ini memungkinkan aplikasi untuk menemukan dan mengambil bagian yang paling relevan dari dokumen untuk menjawab pertanyaan spesifik pengguna. Proses ini melibatkan empat langkah utama:
\begin{enumerate}
  \item \emph{Loading}: Mengekstrak teks dari dokumen PDF.
  \item \emph{Splitting}: Memecah teks yang diekstrak menjadi bagian yang lebih kecil dan mudah dikelola.
  \item \emph{Embedding}: Mengubah setiap potongan teks menjadi vektor numerik yang menangkap makna semantiknya.
  \item \emph{Storing}: Menyimpan \emph{embeddings} ini dalam penyimpanan vektor untuk pencarian yang efisien.
\end{enumerate}

\begin{figure}[htbp]
  \centering
  \includegraphics[width=0.85\linewidth]{images/bab-3/embeddings.png}
  \caption{Contoh \emph{RAG framework} yang menggunakan \emph{Vector Database}.}\label{fig:RAG-Framework}\citep{Jing}
\end{figure}

\subsection{Pembuatan Sitasi Otomatis}
Sistem melakukan ekstraksi metadata seperti judul, penulis, tahun, dan DOI dari PDF dan menyusunnya ke dalam format referensi (APA, MLA, IEEE, Harvard) secara otomatis.

\subsection{Anotasi PDF}
Menggunakan integrasi \texttt{PDF.js}, pengguna dapat melakukan:
\begin{itemize}
  \item Highlight teks
  \item Menambahkan catatan
  \item Menyimpan anotasi ke database untuk ditampilkan ulang
\end{itemize}
\section{Teknik Pengumpulan dan Analisis Data}
Data yang dikumpulkan mencakup interaksi pengguna dengan sistem, log respons chatbot, dan hasil uji coba fungsionalitas. Teknik pengumpulan data dilakukan melalui:
\begin{itemize}
  \item \textbf{Observasi langsung} saat pengguna menguji aplikasi.
  \item \textbf{Log sistem} untuk merekam performa model dan hasil pencarian RAG.
  \item \textbf{Kuesioner} untuk mengevaluasi aspek usability dari sisi pengguna.
\end{itemize}

\section{Pengujian dan Evaluasi}
Setiap iterasi diuji dengan pendekatan:
\begin{itemize}
  \item \textbf{Pengujian fungsional} untuk memastikan seluruh fitur seperti upload, tanya jawab, sitasi, dan anotasi bekerja sebagaimana mestinya.
  \item \textbf{Evaluasi usability} melalui uji coba langsung oleh pengguna sasaran (mahasiswa) untuk mendapatkan umpan balik terkait kemudahan dan efektivitas penggunaan sistem.
\end{itemize}
\section{Deployment dan Pengujian Aplikasi}

\begin{itemize}
  \item Aplikasi di-deploy ke \texttt{Vercel} dengan environment variables yang dikonfigurasi langsung di dashboard.
  \item Database Neon dan Redis Upstash beroperasi sebagai layanan terpisah dan dihubungkan melalui koneksi aman.
  \item Pengujian dilakukan melalui \texttt{functional testing} dan \texttt{usability testing} untuk memastikan semua fitur berjalan sesuai tujuan awal.
\end{itemize}

\noindent
Dengan mengikuti alur implementasi di atas, sistem chatbot dapat berjalan sesuai \emph{requirements} dan terintegrasi penuh antara penyimpanan dokumen, pencarian berbasis vektor, dan generasi jawaban kontekstual melalui LLM.