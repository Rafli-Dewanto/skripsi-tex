\section{\textit{Deployment}}

Deployment adalah fase penting dalam Software Development Life Cycle (SDLC) yang melibatkan pelepasan perangkat lunak yang dikembangkan ke environment production yang dapat diakses dan digunakan oleh end-user. Fase ini mengikuti pengujian ekstensif untuk memastikan bahwa perangkat lunak memenuhi standar dan fungsi yang diperlukan sebagaimana dimaksud. Strategi penerapan yang efektif sangat penting untuk meminimalkan waktu henti dan memastikan kelancaran transisi dari pengembangan ke produksi \citep{chan2020devops}.
\singlespacing{}
Selama fase deployment, berbagai aktivitas dilakukan, termasuk instalasi perangkat lunak di server, konfigurasi lingkungan, dan migrasi data jika diperlukan. Penting juga untuk memantau proses penerapan untuk mengatasi masalah apa pun yang mungkin timbul dengan cepat, memastikan bahwa perangkat lunak beroperasi dengan benar di lingkungan langsung \citep{chan2020devops}. Chan (2020) berpendapat bahwa deployment yang sukses tidak hanya melibatkan pelaksanaan teknis tetapi juga memerlukan perencanaan dan koordinasi yang cermat di antara anggota tim agar selaras dengan tujuan bisnis dan kebutuhan pengguna \citep{chan2020devops}.

\subsection{\textit{Vercel}}
\textit{Vercel} adalah platform cloud untuk situs statis dan fungsi tanpa server, yang dikenal karena fokusnya pada penerapan cepat dan integrasi tanpa batas dengan alur kerja pengembangan modern. Sebelumnya dikenal sebagai \emph{Zeit}, \emph{Vercel} menyederhanakan proses penerapan aplikasi web, terutama yang dibangun menggunakan kerangka kerja seperti \emph{Next.js}, \emph{React}, \emph{Angular}, dan \emph{Vue.js}. Ia menawarkan jaringan edge global yang memastikan pengiriman konten berkinerja tinggi, mengurangi latensi bagi pengguna di seluruh dunia \citep{vercel2025platform}.
\singlespacing{}
Salah satu fitur menonjol Vercel adalah dukungannya terhadap fungsi tanpa server (serverless), yang memungkinkan pengembang menerapkan logika backend tanpa mengelola infrastruktur server tradisional. Arsitektur tanpa server ini diskalakan secara otomatis berdasarkan permintaan, mengoptimalkan efisiensi biaya, dan menyederhanakan alur kerja penerapan. Vercel juga menyediakan platform terpadu di mana pengembang dapat mengelola penerapan frontend dan backend, menyederhanakan siklus pengembangan mulai dari perubahan kode hingga penerapan produksi \citep{vercel2025platform}.
\singlespacing{}
Selain deployment, Vercel menawarkan fitur kolaborasi yang memfasilitasi alur kerja tim, termasuk pratinjau penerapan untuk berbagi aplikasi yang sedang dalam proses dengan pemangku kepentingan. Ini terintegrasi dengan baik dengan sistem kontrol versi seperti Git, memungkinkan penerapan otomatis setelah perubahan kode, yang meningkatkan produktivitas pengembang dan mempercepat waktu pemasaran aplikasi.\@ \citep{vercel2025platform}