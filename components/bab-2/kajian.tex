\section{Kajian Penelitian}

Kajian penelitian dilakukan untuk menelaah berbagai studi terdahulu yang berkaitan dengan pemanfaatan teknologi \emph{Artificial Intelligence} (AI), khususnya \emph{Large Language Models} (LLM), dalam mendukung kegiatan akademik seperti memahami literatur ilmiah, membuat sitasi, serta melakukan anotasi dokumen. Fokus utama dari kajian ini adalah studi yang membahas penerapan metode \textit{Retrieval-Augmented Generation (RAG)}, framework Langchain, dan integrasi LLM dalam konteks aplikasi berbasis chatbot untuk riset akademik.
\singlespacing{}
Studi-studi terdahulu yang dikaji dalam penelitian ini dipilih karena memiliki keterkaitan erat dengan topik yang diangkat, baik dari sisi teknologi maupun tujuan penggunaannya. Beberapa penelitian menunjukkan bahwa RAG efektif digunakan dalam memperbaiki akurasi jawaban model AI terhadap konteks dokumen tertentu, serta meningkatkan relevansi jawaban terhadap pertanyaan pengguna. Sementara itu, Langchain telah terbukti sebagai framework yang fleksibel dan dapat diintegrasikan dengan berbagai sistem vektor dan LLM untuk membangun aplikasi cerdas yang modular.
\singlespacing{}
Selain itu, terdapat juga penelitian yang membahas pengembangan sistem pendukung pembelajaran atau riset berbasis chatbot, baik dalam konteks pendidikan formal maupun dalam proses penulisan karya ilmiah. Penelitian-penelitian ini menjadi acuan dalam perancangan sistem yang diusulkan, sekaligus sebagai pembanding untuk menilai keunikan serta kontribusi dari aplikasi yang dikembangkan dalam tugas akhir ini.
\singlespacing{}
Tabel \ref{tab:kajian-penelitian} berikut menyajikan ringkasan hasil kajian terhadap beberapa penelitian terdahulu yang relevan, mencakup tujuan, metode, dan teknologi yang digunakan, serta kontribusi masing-masing studi terhadap pengembangan sistem yang diusulkan.

\vspace{-10pt}

\begin{table}[H]
  \caption{Rangkuman Penelitian Terdahulu}
  \label{tab:kajian-penelitian}
  \centering
  \begin{longtable}{|>{\centering\arraybackslash}m{0.5cm}|>{\raggedright\arraybackslash}p{3cm}|>{\raggedright\arraybackslash}p{3.5cm}|>{\raggedright\arraybackslash}p{5cm}|}
    \hline
    \textbf{No} & \textbf{Peneliti} & \textbf{Judul Penelitian} & \textbf{Hasil} \\
    \hline
    \endfirsthead

    \hline
    \textbf{No} & \textbf{Peneliti} & \textbf{Judul Penelitian} & \textbf{Hasil} \\
    \hline
    \endhead

    \hline
    1 & \cite{Dardjito} & \textit{Challenges in reading English academic texts for non-English major students of an Indonesian university} & Hasil utama menunjukkan bahwa tantangan terbesar bagi mahasiswa adalah ketergantungan mereka pada strategi penerjemahan kata per kata. Strategi ini, yang berakar pada keterbatasan penguasaan kosakata dan tata bahasa, justru menjadi penghalang utama dalam memahami teks secara efektif karena terjemahan harfiah sering kali tidak menghasilkan makna yang koheren. \\
    \hline
    2 & \cite{kong2024pedagogical} & \textit{A pedagogical design for self-regulated learning in academic writing using text-based generative artificial intelligence tools: 6-P pedagogy of plan, prompt, preview, produce, peer-review, portfolio-tracking} & Penelitian ini menunjukkan bahwa pendekatan 6-P pedagogy efektif dalam meningkatkan kemampuan berpikir kritis dan regulasi diri mahasiswa dalam penulisan akademik berbasis \textit{Generative AI}. Melalui enam tahap seperti prompting, preview, dan peer-review, mahasiswa didorong untuk menggunakan AI secara etis dan reflektif. Model ini juga memberi pendidik kerangka kerja untuk menjaga keseimbangan antara inovasi dan integritas akademik. \\
    \hline
    3 & \cite{Song2025Interactions} & \textit{Interactions with generative AI chatbots: unveiling dialogic dynamics, students' perceptions, and practical competencies in creative problem-solving} & Chatbot berbasis \textit{Generative AI} terbukti meningkatkan kemampuan pemecahan masalah kreatif mahasiswa, dengan interaksi berbasis pengetahuan yang lebih kompleks dan mendalam dibandingkan dengan diskusi antarmahasiswa. \\
    \hline
  \end{longtable}
\end{table}