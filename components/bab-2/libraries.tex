\section{\textit{Library} yang digunakan}
Dalam pengembangan aplikasi web, penggunaan library sangatlah penting untuk mempercepat proses pengembangan dan meningkatkan produktivitas. Library menyediakan kumpulan fungsi dan metode yang siap digunakan, sehingga pengembang tidak perlu menulis kode dari awal untuk setiap fitur yang ingin ditambahkan.

\subsection{\textit{React.js}}
React.js adalah toolkit JavaScript sumber terbuka terkenal yang unggul dalam mengembangkan single-page application untuk aplikasi web \citep{react2025learn}. React.js menggunakan struktur berbasis komponen untuk membangun antarmuka pengguna aplikasi web secara efisien dan dinamis \citep{oghlukyan2022information}. Library ini memungkinkan pengembang untuk membangun antarmuka pengguna yang dinamis dan adaptif dengan memecah UI menjadi komponen yang dapat digunakan kembali dan dikelola secara mandiri \citep{react2025learn}. React.js memungkinkan pengembang untuk secara efektif memperbarui dan menampilkan komponen seiring dengan perubahan state aplikasi, sehingga meningkatkan efisiensi dan pengalaman pengguna \citep{react2025learn}.
\singlespacing{}
Virtual DOM adalah aspek penting dari React.js karena memungkinkan library ini meningkatkan proses pembaruan dengan secara selektif me-render ulang hanya komponen yang mengalami perubahan, bukan me-render ulang seluruh halaman \citep{oghlukyan2022information}. Strategi ini secara signifikan meningkatkan kecepatan dan efisiensi aplikasi web yang dibangun dengan React.js.
\singlespacing{}
Lebih lanjut, React.js menyederhanakan proses pembangunan aplikasi satu halaman dengan memungkinkan pembuatan konten dinamis yang dapat diperbarui tanpa perlu memuat ulang halaman secara keseluruhan \citep{oghlukyan2022information}. React.js menyederhanakan proses pengembangan dan memungkinkan pembuatan aplikasi web yang kompleks dengan secara efektif menangani komponen UI dan status \citep{react2025learn}.

\subsection{Next.js}
Next.js adalah sebuah framework yang yang secara khusus dibuat untuk membangun aplikasi web full-stack. Framework ini memanfaatkan komponen-komponen React untuk membuat antarmuka pengguna dan menyediakan fitur-fitur tambahan serta pengoptimalan yang lebih dari apa yang ditawarkan oleh React konvensional. Dengan mengabstraksi dan secara otomatis mengkonfigurasi tooling penting untuk React, seperti bundling dan compiling, framework ini merampingkan proses pengembangan. Hal ini memungkinkan para pengembang untuk fokus pada pembuatan aplikasi daripada berurusan dengan tugas-tugas konfigurasi yang membosankan. Framework ini memfasilitasi pengembangan aplikasi React yang interaktif, dinamis, dan berkinerja tinggi baik untuk pengembang individu maupun tim yang lebih besar. Selain itu, Next.js menggabungkan rendering sisi server dan pembuatan situs statis, yang menghasilkan peningkatan efisiensi dan pengoptimalan mesin pencari. Hal ini menjadikannya pilihan serbaguna untuk pengembangan web kontemporer \citep{nextjs2025documentation}.
