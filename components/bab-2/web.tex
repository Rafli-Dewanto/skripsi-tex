\section{\textit{World-Wide Web}}
Sejak awal kemunculannya, internet telah berkembang secara signifikan. Awalnya dirancang untuk berbagi dokumen, internet telah berkembang menjadi jaringan kompleks yang mendukung berbagai aplikasi, layanan, dan interaksi. Franco (2021) mencatat bahwa internet lebih dari sekadar kumpulan halaman web; ini adalah platform dinamis yang memfasilitasi komunikasi, perdagangan, dan pembentukan komunitas. Memahami struktur dan fungsionalitas internet sangat penting bagi pengembang yang ingin menciptakan aplikasi yang efisien dan ramah pengguna.
\singlespacing{}
Evolusi internet dapat dibagi menjadi beberapa fase yang berbeda. Web 1.0 ditandai dengan halaman web statis, sementara Web 2.0 memperkenalkan interaktivitas dan konten yang dibuat oleh pengguna. Peralihan menuju Web 3.0, atau web semantik, bertujuan untuk meningkatkan konektivitas dan kegunaan data. Franco (2021) membahas bagaimana fase-fase ini telah mempengaruhi pendekatan pengembangan web, yang mengarah pada adopsi standar dan teknologi baru yang meningkatkan pengalaman pengguna dan aksesibilitas.
\singlespacing{}
Dalam pengembangan web, HTML, CSS, dan JavaScript adalah teknologi penting. HTML mendefinisikan struktur halaman web, CSS mengatur gaya dan tata letak elemen, dan JavaScript menambah interaktivitas dan memungkinkan pembuatan konten dinamis. Buku ini menyoroti pentingnya HTML5, yang memperkenalkan elemen dan API baru yang memfasilitasi integrasi multimedia dan meningkatkan keterlibatan pengguna. Misalnya, elemen <canvas> memungkinkan pembuatan grafik dinamis, memungkinkan pengembang untuk membangun aplikasi yang kaya secara visual.

\section{\textit{HTML}}
Sejak awal kemunculannya, internet telah berkembang secara signifikan. Awalnya dirancang untuk berbagi dokumen, internet telah berkembang menjadi jaringan kompleks yang mendukung berbagai aplikasi, layanan, dan interaksi.\@ \citet{franco2021html} mencatat bahwa internet lebih dari sekadar kumpulan halaman web; ini adalah platform dinamis yang memfasilitasi komunikasi, perdagangan, dan pembentukan komunitas. Memahami struktur dan fungsionalitas internet sangat penting bagi pengembang yang ingin menciptakan aplikasi yang efisien dan ramah pengguna.
\singlespacing{}
Evolusi internet dapat dibagi menjadi beberapa fase yang berbeda. Web 1.0 ditandai dengan halaman web statis, sementara Web 2.0 memperkenalkan interaktivitas dan konten yang dibuat oleh pengguna. Peralihan menuju Web 3.0, atau web semantik, bertujuan untuk meningkatkan konektivitas dan kegunaan data.\@ \citet{franco2021html} membahas bagaimana fase-fase ini telah mempengaruhi pendekatan pengembangan web, yang mengarah pada adopsi standar dan teknologi baru yang meningkatkan pengalaman pengguna dan aksesibilitas.
\singlespacing{}
Dalam pengembangan web, HTML, CSS, dan JavaScript adalah teknologi penting. HTML mendefinisikan struktur halaman web, CSS mengatur gaya dan tata letak elemen, dan JavaScript menambah interaktivitas dan memungkinkan pembuatan konten dinamis. Buku ini menyoroti pentingnya HTML5, yang memperkenalkan elemen dan API baru yang memfasilitasi integrasi multimedia dan meningkatkan keterlibatan pengguna. Misalnya, elemen <canvas> memungkinkan pembuatan grafik dinamis, memungkinkan pengembang untuk membangun aplikasi yang kaya secara visual.

\section{\textit{CSS (Cascading Style Sheets)}}
CSS, atau Cascading Style Sheets, adalah bahasa stylesheet yang digunakan untuk mendeskripsikan presentasi dokumen yang ditulis dalam HTML atau XML (termasuk dialek XML seperti SVG, MathML, atau XHTML). Menurut MDN Web Docs, CSS adalah teknologi landasan World Wide Web, bersama dengan HTML dan JavaScript, yang memungkinkan pemisahan konten dari desain, memungkinkan pengembang mengontrol tata letak, warna, font, dan keseluruhan presentasi visual halaman web.\@ di berbagai perangkat dan ukuran layar \citep{mozilla2025mdn}.
\singlespacing{}
CSS bekerja dengan mengasosiasikan aturan gaya dengan elemen HTML.\@ Aturan-aturan ini dapat didefinisikan dalam stylesheet eksternal, tertanam dalam dokumen HTML, atau sejajar dengan elemen HTML tertentu. Fleksibilitas ini memungkinkan pendekatan modular terhadap desain web, di mana gaya dapat digunakan kembali dan dipelihara dengan lebih mudah. Selain itu, CSS memperkenalkan konsep cascading, di mana beberapa aturan gaya dapat diterapkan pada elemen yang sama, dengan konflik diselesaikan berdasarkan kekhususan dan urutan sumber. Fitur ini memungkinkan desain canggih dan berlapis yang beradaptasi dengan berbagai konteks dan preferensi pengguna \citep{mozilla2025mdn}.
\singlespacing{}
Selain itu, CSS mendukung berbagai jenis dan fitur media, memungkinkan praktik desain responsif. Kueri media, misalnya, memungkinkan gaya beradaptasi berdasarkan karakteristik perangkat, seperti resolusi layar, orientasi, atau kedalaman warna. Kemampuan ini sangat penting untuk membuat situs web yang memberikan pengalaman pengguna yang konsisten dan optimal di berbagai perangkat, mulai dari ponsel hingga komputer desktop \citep{mozilla2025mdn}.

\subsection{\textit{TailwindCSS}}
Tailwind CSS adalah sistem CSS yang memberikan prioritas pada kelas utilitas untuk secara efisien membangun pengalaman pengguna yang dipersonalisasi. Kelas-kelas utilitas yang disediakan oleh alat ini memungkinkan gaya langsung terhadap elemen-elemen dalam HTML, menawarkan alternatif yang lebih fleksibel dan berkelanjutan dibandingkan dengan kerangka kerja CSS konvensional. Tailwind CSS bertujuan untuk menyediakan seperangkat komponen yang komprehensif untuk membangun kerangka desain. Kerangka kerja ini dapat disesuaikan dan diperluas sesuai dengan kebutuhan unik proyek Anda, semuanya di dalam kode HTML Anda. Pendekatan ini menghilangkan kebutuhan untuk membuat CSS khusus sementara tetap menawarkan fungsionalitas gaya yang kuat \citep{tailwind}.

\section{\textit{JavaScript}}
JavaScript diperkenalkan pada tahun 1995 sebagai cara untuk menambahkan program ke halaman web di browser Netscape Navigator. Bahasa ini sejak itu telah diadopsi oleh semua browser web grafis utama lainnya. Ini telah membuat aplikasi web modern menjadi mungkin, yaitu aplikasi yang dapat Anda interaksikan secara langsung tanpa perlu memuat ulang halaman untuk setiap tindakan. JavaScript juga digunakan di situs web yang lebih tradisional untuk menyediakan berbagai bentuk interaktivitas dan kecerdikan \citep{haverbeke2024eloquent}. Awalnya dirancang untuk scripting sisi klien, kemampuan JavaScript berkembang dengan diperkenalkannya Node.js, yang memungkinkan pengembang membuat aplikasi sisi server yang berbasis event dengan mudah \citep{jartarghar2022react}. Evolusi ini mengubah JavaScript dari bahasa scripting sederhana menjadi alat serbaguna untuk mengembangkan solusi web yang kompleks.
Dalam JavaScript, tipe data sangat penting untuk memahami cara kerja bahasa dan cara menangani berbagai jenis nilai.

\section{\textit{TypeScript}}
TypeScript adalah bahasa pemrograman yang didasarkan pada JavaScript dan memiliki pengetikan yang kuat. Hal ini bertujuan untuk meningkatkan dan mengoptimalkan proses pengembangan aplikasi berskala besar dengan mengkompilasi ke dalam JavaScript biasa. TypeScript menambahkan definisi tipe statis ke JavaScript, sehingga pengembang dapat mengidentifikasi dan memperbaiki masalah tipe selama kompilasi, bukan selama eksekusi. Hasilnya, kode menjadi lebih tangguh dan lebih mudah dipelihara karena sistem pemeriksaan tipe TypeScript dapat mengidentifikasi kesalahan pemrograman yang umum terjadi, seperti tipe data yang tidak konsisten dan penggunaan fungsi yang salah, pada tahap awal pengembangan. Selain itu, TypeScript menyediakan dukungan untuk fungsi JavaScript kontemporer dan dengan mudah diintegrasikan dengan basis kode JavaScript yang sudah ada sebelumnya, sehingga proses adopsi dapat dilakukan secara bertahap tanpa perlu menulis ulang secara total. Fitur ini menjadikannya alat pragmatis untuk meningkatkan skalabilitas dan ketergantungan program yang rumit sambil mempertahankan kemampuan beradaptasi dan keterusterangan yang melekat pada JavaScript \citep{typescript2025handbook}.