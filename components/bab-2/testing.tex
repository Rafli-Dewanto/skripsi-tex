\section{\textit{Agile Testing}}

Pengujian Agile merupakan bagian integral dari proses pengembangan perangkat lunak Agile, yang menekankan pentingnya umpan balik dan kolaborasi berkelanjutan di antara anggota tim. Tidak seperti metode pengujian tradisional, yang sering terjadi di akhir siklus pengembangan, pengujian Agile dilakukan di se-luruh proses pengembangan. Pendekatan ini memungkinkan deteksi dini cacat dan memastikan bahwa produk perangkat lunak selaras dengan persyaratan dan harapan pengguna.
\singlespacing{}
Dalam metodologi Agile, pengujian bukanlah fase terpisah tetapi tertanam dalam siklus pengembangan, yang sering disebut sebagai iterasi atau sprint. Proses iteratif ini memungkinkan tim untuk beradaptasi dengan persyaratan yang berubah dan memasukkan umpan balik pengguna dengan segera. Keterlibatan berbagai pemangku kepentingan, termasuk pengembang, penguji, dan pengguna akhir, men-dorong lingkungan kolaboratif yang meningkatkan kualitas produk akhir \citep{pandit2015agileuat}. \textit{User Acceptance Testing} (UAT) disebutkan sebagai salah satu metode dalam pengujian Agile. Menurut \citet{pandit2015agileuat}, UAT umumnya dilakukan secara manual dan tidak disarankan untuk diotomatisasi. Kerangka kerja UAT tersedia untuk metodologi Agile seperti Scrum.
\singlespacing{}
\textit{User Acceptance Testing} (UAT) memainkan peran penting dalam pengujian Agile, karena ini adalah fase di mana pengguna akhir memvalidasi perangkat lunak terhadap kebutuhan mereka. UAT biasanya dilakukan secara manual dan tidak disarankan untuk diotomatisasi, karena berfokus pada upaya memastikan bahwa perangkat lunak sesuai dengan tujuan dari perspektif pengguna \citep{pandit2015agileuat}. Dengan mengintegrasikan UAT ke dalam kerangka kerja Agile, tim dapat memastikan bahwa perangkat lunak tidak hanya memenuhi spesifikasi teknis tetapi juga memberikan nilai kepada pengguna.

\subsection{\textit{User Acceptance Testing}}
\textit{User Acceptance Testing} (UAT) adalah tahapan penting dalam software development life cycle yang bertujuan untuk memastikan bahwa sistem yang dikembangkan memenuhi kebutuhan dan harapan pengguna akhir sebelum diimplementasikan dalam lingkungan bisnis yang sesungguhnya. UAT sering kali menjadi langkah terakhir dalam proses pengujian perangkat lunak sebelum sistem tersebut dirilis ke pengguna akhir.
\singlespacing{}
Menurut \citet{hambling2013user}, UAT merupakan pengujian yang dilakukan oleh pengguna akhir untuk memverifikasi bahwa sistem informasi baru bekerja sesuai dengan tujuan awal dan memenuhi persyaratan bisnis yang telah ditetapkan.
Dari definisi ini, terdapat tiga aspek penting yang perlu diperhatikan dalam pelaksanaan UAT:\@
\begin{enumerate}
  \item Pengujian Formal: UAT memerlukan pengujian formal, yang berarti bahwa pengujian harus dirancang dan dilaksanakan dengan cara yang terstruktur untuk memberikan bukti objektif mengenai keabsahan sistem. Ini mencakup penyiapan skenario pengujian yang sesuai dengan kebutuhan pengguna dan standar bisnis yang berlaku.
  \item Kebutuhan Pengguna dan Proses Bisnis: UAT tidak hanya fokus pada pengujian berdasarkan spesifikasi teknis, tetapi juga harus memperhatikan kebutuhan pengguna dan proses bisnis yang ada. Pengujian ini bertujuan untuk memastikan bahwa sistem dapat mendukung aktivitas sehari-hari pengguna dan membantu mencapai tujuan bisnis.
  \item Kriteria Penerimaan: Kriteria penerimaan merupakan standar yang harus dipenuhi oleh sistem agar dapat diterima oleh pengguna akhir. Kriteria ini biasanya mencakup aspek fungsionalitas, kinerja, keamanan, dan kemudahan penggunaan. UAT bertujuan untuk memastikan bahwa semua kriteria ini terpenuhi sebelum sistem diimplementasikan.
\end{enumerate}
\singlespacing{}
Pelaksanaan UAT yang efektif melibatkan partisipasi aktif dari pengguna akhir, pemangku kepentingan bisnis, dan tim pengembang. Proses ini meliputi penentuan skenario pengujian, eksekusi pengujian, pencatatan hasil pengujian, dan penyelesaian masalah yang ditemukan. Hasil dari UAT menjadi dasar bagi keputusan apakah sistem siap untuk diterapkan dalam lingkungan produksi atau memerlukan perbaikan lebih lanjut. Dengan demikian, UAT memainkan peran krusial dalam menjamin bahwa sistem yang dikembangkan benar-benar dapat memenuhi kebutuhan pengguna dan mendukung tujuan bisnis secara efektif, sebelum diluncurkan ke lingkungan produksi \citep{hambling2013user}