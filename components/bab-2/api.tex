\subsection{Application Programming Interface}

Application Programming Interface (API) didefinisikan sebagai sekumpulan protokol dan alat yang memungkinkan berbagai aplikasi perangkat lunak untuk berkomunikasi dan berinteraksi satu sama lain. Menurut \citet{richardson2013restful}, API berfungsi sebagai perantara yang memungkinkan pengembang mengakses fungsionalitas suatu layanan atau aplikasi tanpa perlu memahami cara kerja internalnya. Abstraksi ini memfasilitasi integrasi sistem yang beragam, memungkinkan pembuatan aplikasi kompleks yang dapat memanfaatkan layanan yang ada secara efisien.
\singlespacing{}
API dapat dikategorikan ke dalam berbagai jenis, termasuk \textit{web API} yang menggunakan protokol HTTP untuk komunikasi melalui internet, dan \textit{library API} yang dirancang untuk digunakan dalam bahasa pemrograman tertentu. Desain API sangat penting karena harus intuitif dan terdokumentasi dengan baik untuk memastikan kemudahan penggunaan bagi pengembang. Selain itu, API yang dirancang dengan baik dapat meningkatkan interoperabilitas aplikasi, memungkinkan mereka bekerja bersama dengan lancar dan berbagi data secara efektif.
\singlespacing{}
Richardson dan Amundsen menekankan bahwa ``API sangat penting untuk memungkinkan pengembangan aplikasi yang dapat berinteraksi dengan layanan lain, sehingga mendorong inovasi dan kolaborasi dalam pengembangan perangkat lunak'' \citep{richardson2013restful}.