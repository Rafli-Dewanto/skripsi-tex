\chapter{PENDAHULUAN}

\setcounter{page}{1}
\pagenumbering{arabic}
\thispagestyle{plain}
\pagestyle{arabicstyle}

\section{Latar Belakang Masalah}

Mahasiswa, terutama yang bukan penutur asli bahasa Inggris, sering menghadapi tantangan dalam memahami teks akademik berbahasa Inggris. Kesulitan ini disebabkan oleh kompleksitas struktur kalimat, penggunaan kosakata teknis, dan gaya penulisan yang formal. Penelitian oleh~\cite{Dardjito} menunjukkan bahwa mahasiswa di Indonesia sering merasa cemas dan kurang percaya diri saat membaca jurnal akademik dalam bahasa Inggris, yang berdampak negatif pada keterlibatan mereka dalam proses pembelajaran dan penelitian. Kesulitan ini juga dapat menghambat kemampuan mereka untuk mengembangkan pemikiran kritis dan berkontribusi dalam diskusi ilmiah di bidang studi mereka.

Sebagai solusi terhadap tantangan tersebut, pemanfaatan Artificial Intelligence (AI) dapat secara signifikan meningkatkan pengalaman dan hasil mahasiswa dalam penulisan research paper melalui beberapa mekanisme yang bermanfaat. AI-powered writing tools, seperti pemeriksa tata bahasa dan asisten gaya, memberikan feedback real-time tentang kualitas tulisan, membantu mahasiswa menyempurnakan tata bahasa dan sintaksis mereka untuk menghasilkan teks yang lebih jelas dan koheren \citep{kong2024pedagogical,nazari2021application}. Dengan menggabungkan alat tersebut ke dalam proses penulisan mereka, mahasiswa tidak hanya dapat meningkatkan aspek teknis dari pekerjaan mereka tetapi juga meningkatkan kepercayaan diri mereka dalam menyusun teks akademik \citep{nazari2021application}.

Peran AI meluas melampaui mekanisme bahasa. AI juga dapat membantu dalam melakukan tinjauan literatur dengan cepat meringkas penelitian yang ada dan mengidentifikasi sumber-sumber yang relevan, yang menyederhanakan proses penelitian akademik \citep{gupta2024artificial}. Misalnya, alat AI dapat menganalisis volume besar literatur akademik untuk menyajikan ringkasan singkat, membantu mahasiswa dalam mengumpulkan dan mensintesis informasi yang diperlukan untuk paper mereka secara efisien \citep{bulante2024ai}. Integrasi AI ini tidak hanya meningkatkan efisiensi penelitian tetapi juga membantu mahasiswa mengembangkan kemampuan membaca kritis dan analitis dengan mendorong mereka untuk terlibat dengan konten yang telah disintesis.

Penggabungan Retrieval-Augmented Generation dalam aplikasi chatbot berfungsi untuk memperluas batasan yang melekat pada model generasi konvensional. LLM tradisional sering kali kesulitan dengan akurasi faktual, terutama ketika ditanyai tentang informasi yang spesifik atau dinamis \citep{wang2024mememo,lewis2020retrieval}. Teknik RAG secara efektif mengatasi masalah ini dengan memungkinkan model mengakses dan menggabungkan pengetahuan real-time dari database eksternal, memastikan bahwa respons tidak hanya relevan secara kontekstual tetapi juga akurat dan koheren \citep{lewis2020retrieval}.

Dengan adanya aplikasi ini, mahasiswa dapat berinteraksi secara aktif dengan isi jurnal, bukan hanya membaca pasif, sehingga meningkatkan keterlibatan dan pemahaman akademik mereka. Aplikasi ini bukan bertujuan untuk menggantikan proses berpikir kritis, tetapi mendukung mahasiswa menjadi lebih mandiri dan efisien dalam memahami literatur dan mengelola catatan pribadi mereka. Dengan demikian, pengembangan aplikasi chatbot ini diharapkan mampu menjadi alat bantu yang efektif bagi mahasiswa dalam memahami isi jurnal, mencatat poin penting, dan menyusun sitasi dengan lebih efisien

\section{Rumusan Masalah}
Dalam penulisan ini terdapat beberapa rumusan masalah yang penulis akan bahas, diantaranya:
\begin{enumerate}
  \item Bagaimana merancang dan mengembangkan aplikasi chatbot yang dapat membantu mahasiswa memahami isi jurnal akademik berbahasa Inggris yang kompleks?
        \raggedright
  \item Bagaimana pemanfaatan metode Retrieval-Augmented Generation (RAG) dapat meningkatkan akurasi dan relevansi jawaban chatbot terhadap pertanyaan terkait isi jurnal?
  \item Bagaimana integrasi teknologi Langchain dapat mendukung kemampuan chatbot dalam berinteraksi secara kontekstual dan memberikan pengalaman yang lebih adaptif kepada pengguna?
  \item Bagaimana aplikasi ini dapat membantu mahasiswa dalam membuat sitasi otomatis dengan berbagai gaya referensi?
  \item Bagaimana aplikasi ini dapat menyediakan fitur anotasi PDF sebagai catatan pribadi mahasiswa dalam proses riset?
\end{enumerate}

\section{Ruang Lingkup}

Batasan-batasan tersebut adalah sebagai berikut.
\begin{enumerate}
  \item Citra
\end{enumerate}

\section{Tujuan Penelitian}

Tujuan penelitian yang akan dibahas adalah sebagai berikut.
\begin{enumerate}
  \item lorem ipsum
\end{enumerate}

\section{Sistematika Penulisan}

Sistematika yang digunakan dalam penulisan ini mencakup beberapa bagian
penting, di antaranya:
\begin{enumerate}
  \item PENDAHULUAN

        Bab ini berisi pendahuluan yang menguraikan latar belakang masalah
        yang melandasi pemilihan topik penelitian, didukung oleh data relevan
        dan perbandingan dengan penelitian terdahulu, yang menyoroti kelemahan
        atau perbedaan pendekatan. Ruang lingkup penelitian dibatasi secara
        jelas untuk fokus pada persoalan yang dikaji. Tujuan penelitian ini
        adalah untuk menjawab masalah penelitian dan menghasilkan luaran yang
        diharapkan. Sistematika penulisan laporan skripsi ini akan disusun
        secara naratif, dimulai dari bab pendahuluan yang menguraikan konteks
        penelitian, diikuti oleh tinjauan pustaka yang membahas landasan teoretis,
        metodologi penelitian yang menjelaskan pendekatan dan teknik pengumpulan
        data, hasil penelitian dan pembahasan yang menganalisis temuan, hingga
        kesimpulan dan saran yang merangkum hasil penelitian dan implikasinya.
  \item TINJAUAN PUSTAKA

        Bab ini merupakan tinjauan pustaka yang disusun untuk memberikan landasan
        teoretis dan empiris yang kokoh dalam mendukung pendekatan pemecahan
        masalah yang diusulkan. Bab ini menguraikan secara komprehensif hasil
        penelitian sejenis yang relevan dengan tema yang dipilih, dengan menekankan
        pada kontribusi penelitian-penelitian tersebut terhadap pengembangan
        kerangka teoretis dan metodologis penelitian ini. Selain itu, tinjauan
        pustaka ini juga berfungsi untuk memperkuat argumentasi penelitian
        dengan menunjukkan bukti-bukti empiris dari penelitian sebelumnya,
        serta mengidentifikasi celah penelitian yang akan diisi oleh penelitian
        ini. Tinjauan pustaka ini tidak hanya merangkum literatur yang ada,
        tetapi juga menganalisis dan mensintesisnya untuk membangun dasar
        yang kuat bagi penelitian ini.
  \item HASIL DAN PEMBAHASAN

        Bab ini menyajikan hasil penelitian yang diperoleh, disertai dengan
        analisis mendalam untuk menjawab tujuan penelitian yang telah ditetapkan.
        Pada kasus penelitian yang menghasilkan rancangan, bab ini memuat
        deskripsi rinci hasil rancangan tersebut, termasuk evaluasi komprehensif
        terhadap kelebihan dan keterbatasan yang teridentifikasi. Analisis
        ini dilakukan secara sistematis dan objektif, dengan mengacu pada
        kerangka teoretis dan metodologis yang telah ditetapkan sebelumnya.
  \item PENUTUP

        Bagian penutup ini terdiri dari kesimpulan, yang merangkum jawaban
        atas masalah penelitian berdasarkan temuan yang diperoleh, dan saran,
        yang mengusulkan pengembangan lebih lanjut dari hasil penelitian yang
        telah dipaparkan dalam kesimpulan.
\end{enumerate}