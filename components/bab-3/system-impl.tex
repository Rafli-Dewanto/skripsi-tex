\section{Implementasi Sistem}

Tahap implementasi merupakan proses penting dalam membangun aplikasi sesuai rancangan dan spesifikasi teknis. Sistem ini dikembangkan menggunakan pendekatan \textit{fullstack} dengan framework Next.js 15 (App Router) dan database PostgreSQL yang dihosting melalui layanan Neon. Tidak ada \textit{back-end} terpisah, seluruh \textit{server logic} ditulis dalam API Routes di dalam struktur Next.js. Implementasi juga melibatkan integrasi Langchain, Retrieval-Augmented Generation (RAG), OpenAI API, serta fitur anotasi PDF menggunakan PDF.js. Berikut adalah langkah-langkah implementasi yang dilakukan:

\subsection{Persiapan Database PostgreSQL (Neon)}

\begin{itemize}
  \item Membuat akun dan proyek baru di \url{https://neon.tech}, kemudian membuat \texttt{branch}, \texttt{database}, dan \texttt{user}.
  \item Menyimpan kredensial \texttt{host}, \texttt{database}, \texttt{user}, dan \texttt{password} ke dalam file \texttt{.env} proyek.
  \item Menyusun skema database berdasarkan ERD menggunakan perintah SQL secara manual.
  \item Tabel-tabel utama meliputi: \texttt{users}, \texttt{chats}, \texttt{messages}, \texttt{documents}, dan \texttt{citations}.
\end{itemize}

\subsection{Inisialisasi Proyek Next.js 15}

\begin{itemize}
  \item Menginisialisasi proyek dengan perintah:
  \begin{verbatim}
  npx create-next-app@latest ai-journal-assistant --typescript --app
  \end{verbatim}
  \item Mengaktifkan Tailwind CSS dan PostCSS untuk styling antarmuka.
  \item Menyiapkan folder \texttt{app/api/} untuk menyimpan seluruh logika server-side.
\end{itemize}

\subsection{Integrasi Langchain dan RAG}

\begin{itemize}
  \item Menginstal dependensi utama:
  \begin{verbatim}
  npm install langchain @langchain/community @langchain/openai
  \end{verbatim}
  \item Menggunakan pipeline \texttt{RetrievalQAChain} dari Langchain untuk menghubungkan retriever dan model LLM (GPT-4).
  \item Langkah-langkah implementasi:
  \begin{itemize}
    \item Mengekstrak teks dari PDF.\@
    \item Memecah teks menjadi potongan pendek (\texttt{text splitting}).
    \item Membuat embeddings dengan model dari OpenAI (\texttt{text-embedding-ada-002}).
    \item Menyimpan embeddings ke dalam \texttt{vector store} (menggunakan Chroma).
    \item Melakukan similarity search ketika pengguna mengajukan pertanyaan.
    \item Menyediakan jawaban dari LLM yang memperhitungkan hasil retrieval.
  \end{itemize}
\end{itemize}

\subsection{Integrasi OpenAI API}

\begin{itemize}
  \item Mengatur API Key di \texttt{.env} sebagai \texttt{OPENAI\_API\_KEY}.
  \item Menggunakan endpoint \texttt{/api/chat} untuk menerima \texttt{prompt} dan merespons dengan hasil dari GPT-4.
  \item Mendukung konteks percakapan dengan menyimpan riwayat \texttt{messages} pada database.
\end{itemize}

\subsection{Penyimpanan Dokumen PDF dengan Vercel Blob}

\begin{itemize}
  \item Menggunakan \texttt{@vercel/blob} untuk menyimpan file PDF dari pengguna secara langsung ke storage Vercel.
  \item File PDF ini kemudian diakses kembali oleh sistem untuk proses ekstraksi isi dokumen.
  \item URL dokumen disimpan pada tabel \texttt{documents} dalam database.
\end{itemize}

\subsection{Manajemen State dan Cache dengan Redis (Upstash)}
\begin{itemize}
  \item Menggunakan Redis sebagai cache session dan penyimpanan embeddings sementara.
  \item Mendaftar ke layanan \url{https://upstash.com} dan menambahkan variabel \texttt{UPSTASH\_REDIS\_URL} dan \texttt{TOKEN} pada \texttt{.env}.
  \item Menggunakan package \texttt{@upstash/redis} untuk operasi Redis dalam API Routes.
\end{itemize}

\subsection{Preview dan Anotasi PDF dengan PDF.js}
\begin{itemize}
  \item Mengintegrasikan \texttt{pdfjs-dist} untuk menampilkan dokumen PDF secara langsung di browser.
  \item Fitur anotasi dikembangkan dengan menambahkan layer interaktif di atas canvas, yang mencatat teks yang dipilih dan menampilkan input komentar.
  \item Catatan dan posisi anotasi disimpan dalam tabel \texttt{annotations} atau bagian dari \texttt{documents}.
\end{itemize}