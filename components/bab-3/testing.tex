\section{Pengujian}
Pengujian fungsional dilakukan untuk memastikan bahwa setiap fitur aplikasi berjalan sesuai dengan kebutuhan dan spesifikasi yang telah ditentukan. Setiap fungsi diuji berdasarkan skenario penggunaan umum oleh pengguna akhir, dengan fokus pada:

\subsection{Teknik Pengumpulan dan Analisis Data}

Teknik pengumpulan data dalam penelitian ini difokuskan pada evaluasi fungsionalitas dan usability aplikasi melalui pendekatan \textit{User Acceptance Testing} (UAT). Metode ini dipilih untuk memperoleh umpan balik kuantitatif dari pengguna akhir mengenai kinerja dan kegunaan sistem.
\singlespacing{}
Pengujian dilakukan dengan menyebarkan kuesioner UAT dalam bentuk \textit{Google Form} kepada sejumlah pengguna terpilih yang mewakili target pengguna sistem, yaitu mahasiswa yang sering berinteraksi dengan jurnal ilmiah berbahasa Inggris. Kuesioner tersebut terdiri dari sejumlah pernyataan yang dirancang untuk mengukur beberapa aspek utama sistem, antara lain:

\begin{itemize}
  \item Kemudahan dalam mengunggah dan membaca dokumen PDF.\@
  \item Keakuratan dan relevansi jawaban chatbot terhadap isi jurnal.
  \item Kemudahan penggunaan fitur sitasi otomatis.
  \item Kemudahan penggunaan fitur anotasi PDF.\@
  \item Kemudahan navigasi dan tampilan antarmuka secara umum.
\end{itemize}

Responden diminta untuk memberikan penilaian terhadap setiap pernyataan dengan menggunakan skala Likert 1 sampai 5, di mana nilai 1 berarti ``Sangat Tidak Setuju'' dan nilai 5 berarti ``Sangat Setuju''. Data yang terkumpul dianalisis secara kuantitatif dengan menghitung rata-rata skor dari setiap indikator pengujian. Hasil analisis ini akan digunakan untuk menilai tingkat penerimaan pengguna terhadap sistem dan akan dipaparkan lebih lanjut pada Bab 4.