\section{Analisis Kebutuhan}

Tahap awal ini dilakukan untuk mengidentifikasi kebutuhan pengguna dan sistem. Fokus utama adalah permasalahan yang dihadapi mahasiswa dalam memahami jurnal ilmiah, serta fitur yang diperlukan seperti tanya jawab berbasis dokumen, sitasi otomatis, dan anotasi PDF. Dari tahap ini diturunkan kebutuhan fungsional dan non-fungsional sebagai dasar pengembangan sistem.

\subsection{Spesifikasi Perangkat Keras}

Tabel 3.1 menunjukkan informasi lengkap seluruh perangkat keras yang
dipergunakan dalam penelitian ini.

\begin{table}[H]
  \caption{Spesifikasi Perangkat Keras yang Digunakan}

  \centering{}%
  \begin{tabular}{|l|l|}
    \hline
    \textbf{Komponen} & \textbf{Nama}\tabularnewline
    \hline
    \hline
    CPU               & Apple Chip M1\tabularnewline
    \hline
    GPU               & Apple Chip M1\tabularnewline
    \hline
    RAM               & 8GB\tabularnewline
    \hline
    \emph{Chipset}    & M1\tabularnewline
    \hline
    SSD               & 256GB\tabularnewline
    \hline
  \end{tabular}
\end{table}


\subsection{Spesifikasi Perangkat Lunak}

Tabel 3.2 menunjukkan informasi lengkap mengenai seluruh perangkat
lunak yang dipergunakan dalam penelitian ini.

\begin{table}[H]
  \caption{Spesifikasi Perangkat Lunak yang Digunakan}

  \centering{}%
  \begin{tabular}{|l|l|}
    \hline
    \textbf{Aplikasi}          & \textbf{Nama}\tabularnewline
    \hline
    \hline
    OS                         & Macintosh\tabularnewline
    \hline
    \emph{Virtual Environment} & \textit{Node Package Mananger}\tabularnewline
    \hline
    Bahasa Pemrograman         & \textit{TypeScript}\tabularnewline
    \hline
    Editor Teks                & \textit{VS Code}\tabularnewline
    \hline
    \emph{Terminal Emulator}   & \textit{Ghostty}\tabularnewline
    \hline
    \emph{Framework}           & \textit{Next.js}\tabularnewline
    \hline
    Penelusur Web              & \textit{Arc}\tabularnewline
    \hline
    \emph{Version Control}     & \textit{Git}\tabularnewline
    \hline
  \end{tabular}
\end{table}
