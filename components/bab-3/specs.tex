\section{Spesifikasi Perangkat Keras}

Tabel 3.1 menunjukkan informasi lengkap seluruh perangkat keras yang
dipergunakan dalam penelitian ini.

\begin{table}[H]
  \caption{Spesifikasi Perangkat Keras yang Digunakan}

  \centering{}%
  \begin{tabular}{|l|l|}
    \hline
    \textbf{Komponen} & \textbf{Nama}\tabularnewline
    \hline
    \hline
    CPU               & Apple Chip M1\tabularnewline
    \hline
    GPU               & Apple Chip M1\tabularnewline
    \hline
    RAM               & 8GB\tabularnewline
    \hline
    \emph{Chipset}    & M1\tabularnewline
    \hline
    SSD               & 256GB\tabularnewline
    \hline
  \end{tabular}
\end{table}


\section{Spesifikasi Perangkat Lunak}

Tabel 3.2 menunjukkan informasi lengkap mengenai seluruh perangkat
lunak yang dipergunakan dalam penelitian ini.

\begin{table}[H]
  \caption{Spesifikasi Perangkat Lunak yang Digunakan}

  \centering{}%
  \begin{tabular}{|l|l|}
    \hline
    \textbf{Aplikasi}          & \textbf{Nama}\tabularnewline
    \hline
    \hline
    OS                         & Macintosh\tabularnewline
    \hline
    \emph{Virtual Environment} & \emph{Node Package Mananger}\tabularnewline
    \hline
    Bahasa Pemrograman         & \emph{TypeScript}\tabularnewline
    \hline
    Editor Teks                & \emph{VS Code}\tabularnewline
    \hline
    \emph{Terminal Emulator}   & \emph{Ghostty}\tabularnewline
    \hline
    \emph{Framework}           & \emph{Next.js}\tabularnewline
    \hline
    Penelusur Web              & \emph{Arc}\tabularnewline
    \hline
    \emph{Version Control}     & \emph{Git}\tabularnewline
    \hline
  \end{tabular}
\end{table}