\chapter{PENUTUP}
\section{Kesimpulan}

Penelitian ini dilakukan untuk menjawab tantangan yang dihadapi mahasiswa, khususnya non-native English speakers, dalam memahami jurnal ilmiah berbahasa Inggris yang kompleks. Kesulitan dalam memahami struktur kalimat akademik, kosakata teknis, dan gaya penulisan formal menjadi hambatan yang signifikan dalam proses belajar dan riset. 
\singlespacing{}

Melalui pengembangan aplikasi chatbot berbasis \textit{Retrieval-Augmented Generation} (RAG) dan integrasi framework \textit{Langchain}, penelitian ini berhasil merancang sebuah solusi yang adaptif dan interaktif untuk membantu mahasiswa memahami konten jurnal, menyusun sitasi otomatis, dan mencatat informasi penting melalui fitur anotasi PDF. Penggunaan RAG memungkinkan sistem untuk memberikan jawaban yang lebih relevan dan akurat berdasarkan isi jurnal yang diunggah, sedangkan Langchain mendukung modularitas dan fleksibilitas sistem dalam mengelola alur interaksi pengguna dengan dokumen.
\singlespacing{}
Dengan demikian, aplikasi yang dikembangkan memiliki potensi untuk meningkatkan efisiensi dan efektivitas mahasiswa dalam melakukan literatur review, memahami isi jurnal secara kontekstual, serta mengelola catatan dan referensi akademik secara terintegrasi.

\section{Saran}

Berdasarkan hasil penelitian dan pengembangan yang telah dilakukan, terdapat beberapa saran yang dapat dipertimbangkan untuk pengembangan lebih lanjut:

\begin{enumerate}
  \item \textbf{Pengayaan fitur pencarian jurnal:} Aplikasi dapat dikembangkan lebih lanjut untuk terintegrasi dengan database jurnal akademik seperti Google Scholar, Semantic Scholar, atau PubMed, guna memudahkan mahasiswa dalam mencari dan mengunggah referensi langsung dari sumber terpercaya.

  \item \textbf{Pengembangan kemampuan multibahasa:} Menambahkan dukungan multibahasa, seperti penerjemahan otomatis dan dukungan terhadap jurnal berbahasa lain, akan membantu lebih banyak mahasiswa dari berbagai latar belakang bahasa.

  \item \textbf{Evaluasi performa teknis:} Di masa mendatang, aspek teknis seperti kecepatan respon chatbot, efisiensi pencarian vektor, serta beban sistem pada saat penggunaan tinggi perlu dievaluasi melalui pengujian performa untuk memastikan aplikasi tetap stabil dan responsif.

  \item \textbf{Fitur kolaborasi:} Pengembangan anotasi multi-user dan sistem komentar dapat mendukung kerja sama tim dalam riset atau diskusi kelompok, menjadikan aplikasi lebih dinamis dalam konteks akademik kolaboratif.

  \item \textbf{Peningkatan aspek keamanan dan privasi:} Mengingat aplikasi mengelola dokumen pribadi pengguna, maka penting untuk terus mengembangkan fitur keamanan seperti enkripsi dokumen, autentikasi ganda, dan penghapusan data otomatis untuk melindungi privasi pengguna.
\end{enumerate}

Dengan mengadopsi saran-saran di atas, diharapkan aplikasi ini dapat berkembang menjadi platform yang lebih komprehensif dalam mendukung proses riset mahasiswa dan memberikan kontribusi nyata terhadap literasi akademik digital.